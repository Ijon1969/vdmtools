\documentclass[11pt]{article}
\usepackage{a4}
\usepackage{makeidx}
\usepackage{vdmsl-2e}
\usepackage{termref}
\usepackage{longtable}

\newcommand{\StateDef}[1]{{\bf #1}}
\newcommand{\TypeDef}[1]{{\bf #1}}
\newcommand{\TypeOcc}[1]{{\it #1}}
\newcommand{\FuncDef}[1]{{\bf #1}}
\newcommand{\FuncOcc}[1]{#1}
\newcommand{\ModDef}[1]{{\tiny #1}}

\makeindex
%\documentstyle[a4,vdmsl,makeidx,termref,11pt,longtable]{article}

\begin{document}

\title{A Looseness Analysis Experiment}

\author{Peter Gorm Larsen\\ IFAD}

\maketitle

\section{Introduction}

This is an experiment which investigates the complexity of a looseness
analysis tool for an interesting subset of VDM-SL. The subset is
particularly interesting because it illustrates how
underdeterminedness is combined with recursion in VDM-SL.

This paper contains the full specification which have been processed
by the IFAD VDM-SL Toolbox. However, nothing is written here about the
context in which this is to be seen or the tests which have been
carried out. For more information about that see \cite{Larsen94b}.
Finally, it is worth noting here that no particular effort have been spent on
ensuring that erroneous specifications yield something sensible. In
particular this specification will yield bottom if just one of the
models yield bottom.
\newpage

\tableofcontents

\include{as.vdm}

\include{expr.vdm}

\include{pat.vdm}

\include{env.vdm}

\include{auxil.vdm}

\bibliographystyle{newalpha}
\bibliography{ifad}

\newpage
\addcontentsline{toc}{section}{Index}
\printindex

\end{document}
