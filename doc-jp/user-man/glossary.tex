\begin{description}

#ifdef ENG
\item[C++ Code Generator:] The C++ Code Generator can automatically generate
  C++ code from your specification. You need a separate
  license for the \guicmd{C++ Code Generator} in order to access the
  \guicmd{C++ Code Generator} from the \Toolbox.
#endif ENG
  
#ifdef JPN
\item[C++コード生成機能:] 仕様書からC++のコードを自動生成する。
  \Toolbox\ から\guicmd{C++コード生成機能} に
  アクセスするためには別ライセンスが必要である。
#endif JPN
  
#ifdef ENG
\item[Debugger:] With the debugger you can explore the behaviour of
  your specification. The debugger can execute the specification and
  break at function and
  \ifthenelse{\boolean{VDMsl}}{operation}{method} applications. At
  any point in the execution you can explore the local or global state
  and local identifiers in your specification.
#endif ENG

#ifdef JPN
\item[デバッガ:] デバッガを使えば仕様書の振る舞いを調査することができる。
  デバッガは仕様書を実行しアプリケーションの関数や
  \ifthenelse{\boolean{VDMsl}}{操作}{メソッド} でブレークすることもできる。
  実行中いつでも、仕様書内のローカルまたはグローバルな状態、ローカル変数などを調査することができる。\\
#endif JPN

%\ifthenelse{\boolean{VDMsl}}{}{
%\item[Dependency tool:] This tool gives an overview of inheritance and
%association information for a given class.}

#ifdef ENG
\item[Dynamic Semantics:] The dynamic semantics describes the meaning
  of a language. Thus, the dynamic semantics describes
  how the language behaves if it can be executed.
#endif ENG

#ifdef JPN
\item[動的セマンティクス:] 動的セマンティクスは言語の意味を記述する。
  すなわち動的セマンティクスは実行された場合に言語がどう振舞うかを記述する。
#endif JPN

#ifdef ENG
\item[Emacs:] Emacs is an ASCII editor.
#endif ENG

#ifdef JPN
\item[Emacs:] ASCIIエディタ。
#endif JPN

#ifdef ENG
\item[GUI:] Graphical User Interface.
#endif ENG

#ifdef JPN
\item[GUI:] グラフィカルユーザーインターフェース
#endif JPN

%\ifthenelse{\boolean{VDMsl}}{}{
%\item[Inheritance Tool:] The inheritance tool draws the inheritance
%  tree, that is the inheritance structure of your specification.}

#ifdef ENG
\item[Interpreter:] The interpreter can interpret a specification
  according to the dynamic semantics of the language. That is, it can
  execute a program/specification. 
#endif ENG

#ifdef JPN
\item[インタープリタ:] インタープリタは言語の動的動作に従って仕様書を解釈する。
  いわばプログラム/仕様書を実行する。\\
#endif JPN

#ifdef ENG
\ifthenelse{\boolean{VDMsl}}{}{
\item[Java Code Generator:] The Java Code Generator can automatically generate
  Java code from your specification. You need a separate
  license for the \guicmd{Java Code Generator} in order to access the
  \guicmd{Java Code Generator} from the \Toolbox.}
#endif ENG
 
#ifdef JPN
\ifthenelse{\boolean{VDMsl}}{}{
\item[Javaコード生成機能:] 仕様書からJavaのコードを自動生成する。
  \Toolbox\ から \guicmd{Javaコード生成機能} 
  にアクセスするには、別ライセンスが必要である。
}
#endif JPN
 
#ifdef ENG
\item[\LaTeX:] is a generic typesetting system.
#endif ENG

#ifdef JPN
\item[\LaTeX:] 一般的な組版システム
#endif JPN

#ifdef ENG
\item[Pretty Printer:] The pretty printer processes a file
  and produces a pretty printed version of the \vdmslpp\ parts in the
  input \vdmslpp\ file. The output format depends on the input format.
#endif ENG

#ifdef JPN
\item[清書機能:] ファイルを処理して\vdmslpp\ の入力ファイルの
  \vdmslpp\ の箇所の清書版を生成する。
  出力フォーマットは入力フォーマットに依存する。
#endif JPN

#ifdef ENG
\item[Project:] A project is a collection of ASCII file names that make up
  a specification.
#endif ENG

#ifdef JPN
\item[プロジェクト:] 仕様書を構成するASCIIのファイル名の集合
#endif JPN

#ifdef ENG
\item[RTF:] This is an acronym for ``Rich Text Format'' which is one
of the formats which can be used with the Microsoft Word editor.
#endif ENG

#ifdef JPN
\item[RTF:] 「Rich Text Format」の頭字語。
  Microsoft Wordで使用できるフォーマットのひとつ。
#endif JPN

#ifdef ENG
\item[Semantics:] describes the meaning of the language. 
#endif ENG

#ifdef JPN
\item[セマンティクス:] 言語の意味を記述したもの
#endif JPN

#ifdef ENG
\item[Specification:] A specification is a \vdmslpp\ model of a system
  written in one or more files using potentially different input formats.
#endif ENG
  
#ifdef JPN
\item[仕様書:] (おそらく)異なる入力フォーマットを使って書かれた
  1つ以上のファイルからなるシステムの \vdmslpp\ モデル
#endif JPN
  
#ifdef ENG
\item[Static Semantics:] The static semantics describes the
  relationships between the symbols of the language which must be
  obeyed in order for a syntactically correct specification to be
  well-formed (i.e.\ to have a consistent meaning). A well-formed
  specification is also called a type-correct specification.  
#endif ENG

#ifdef JPN
\item[静的セマンティクス:] 構文的に正しい仕様書を適格にするため
  (矛盾のない意味を持たせるため)に従わなくては
  成らない言語の記号間の関係を記述したもの。適格な仕様書とは型的に正しい仕様書とも言える。
#endif JPN

#ifdef ENG
\item[Syntax:] The syntax of a language describes how the symbol
  elements of the language (e.g.\ key words and identifiers) can be related.
  The syntax only describes how the symbols can be ordered in the
  language, not the meaning of the ordering.
#endif ENG

#ifdef JPN
\item[構文:] 言語の構文は、言語の記号要素(キーワード、識別子など)が
  どのように関連しているかを記述したものである。
  構文は言語中で記号がどのように命令されるかを記述しており、命令の意味を記述するものではない。
#endif JPN

#ifdef ENG
\item[Syntax Checker:] A syntax checker verifies if the syntax of a
  specification is correct.
#endif ENG

#ifdef JPN
\item[構文チェック機能:] 仕様書の構文が正しいかどうか確認する。
#endif JPN

#ifdef ENG
\item[Test Coverage Information:] Information about how many times
  each construct in the specification has been executed.
#endif ENG
  
#ifdef JPN
\item[テストカバレッジ情報:] 仕様書の構成物が各々何回実行されたについての情報
#endif JPN
  
#ifdef ENG
\item[Test Coverage File:] The test coverage file contains test
  coverage information.
#endif ENG
  
#ifdef JPN
\item[テストカバレッジファイル:] テストカバレッジ情報を含むファイル。
#endif JPN
  
#ifdef ENG
\item[Type Checker:] The type checker checks the type
  correctness of a specification. A specification can be definitely or
  possibly type correct.
#endif ENG

#ifdef JPN
\item[型チェック機能:] 仕様書の型が正しいかどうかチェックする。
  'def"タイプと'pos"タイプ2種類のチェックがある。
#endif JPN

#ifdef ENG
\item[VDM:] The Vienna Development Method.
#endif ENG

#ifdef JPN
\item[VDM:] ウイーン開発手法
#endif JPN

#ifdef ENG
\item[\vdmsl:] The formal specification language of the {\em Vienna
    Development Method}. \vdmsl\ is an ISO standard
  language‾\cite{ISOVDM96}. 
#endif ENG

#ifdef JPN
\item[\vdmsl:] {\em Vienna Development Method}.の形式仕様言語。
  ISO標準言語である‾\cite{ISOVDM96}。\\
#endif JPN

#ifdef ENG
\item[\vdmpp:] An object-oriented specification language
 that is an extension of ISO VDM-SL.
#endif ENG

#ifdef JPN
\item[\vdmpp:] オブジェクト指向仕様言語。ISO VDM-SLの拡張。\\
#endif JPN

#ifdef ENG
\item[Well-formedness:] A specification can be well-formed with
  respect to the syntax and the static semantics of a language.
#endif ENG

#ifdef JPN
\item[Well-formedness:] 仕様書が言語の構文、静的セマンティクスに関して適格であるということ。
#endif JPN

\end{description}
