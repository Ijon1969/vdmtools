\documentclass[a4paper,10pt]{jsarticle}
%\documentclass[a4paper,10pt]{jsbook}

\usepackage[dvipdfm]{graphicx, color}
%\usepackage{folha}
\graphicspath{{image/}}

\usepackage{color}
\usepackage{array}
\usepackage{longtable}
\usepackage{alltt}
\usepackage{graphics}
\usepackage{vpp-nms}
%\usepackage{vpp}
\usepackage{makeidx}
\makeindex

\usepackage{colortbl}

\usepackage[dvipdfm,bookmarks=true,bookmarksnumbered=true,colorlinks,plainpages=true]{hyperref}

%\AtBeginDvi{\special{pdf:tounicode 90ms-RKSJ-UCS2}}
\AtBeginDvi{\special{pdf:tounicode EUC-UCS2}}


\definecolor{covered}{rgb}{0,0,0}      %black
\definecolor{not-covered}{rgb}{1,0,0}  %red

\setcounter{secnumdepth}{6}
\makeatletter
\renewcommand{\paragraph}{\@startsection{paragraph}{4}{\z@}%
  {1.5\Cvs \@plus.5\Cdp \@minus.2\Cdp}%
  {.5\Cvs \@plus.3\Cdp}%
  {\reset@font\normalsize\bfseries}}
\makeatother

\renewcommand{\sf}{\sffamily \color{blue}}

\newcommand{\syou}{\texttt{<}}
\newcommand{\dai}{\texttt{>}}

%\title{�z�e��}
\author{
�����L\\
�i���jCSK�V�X�e���Y\\
�ʐM�O���[�v\\
VDM�S��\\
}
\date{2006�N10��5��}

%\pagestyle{empty}
\usepackage{fancyhdr}
\usepackage{lastpage} 
  \pagestyle{fancy} 
   \let\origtitle\title 
  \renewcommand{\title}[1]{\lfoot{#1}\origtitle{#1}}

  \rfoot{\today}
  \rhead{[\ \scshape\oldstylenums{\thepage}\ / %
      \scshape\oldstylenums{\pageref{LastPage}}\ ]}
  \cfoot{}


\begin{document}

% the title page
\title{VDM++関数型ライブラリのドキュメント雛形}
\author{佐原 伸\\\\
(株)CSK\\
}
%\institute{\pgldk \and \chessnl}
\date{\mbox{}}
\maketitle

%\TaoReport{ガードコマンド・モデル}{\today}{タオベアーズ}{佐原伸}
%\setlength{\baselineskip}{12pt plus .1pt}
%\tolerance 10000
\tableofcontents
%\thispagestyle{empty} 

%\begin{abstract}
\setlength{\baselineskip}{12pt plus .1pt}
�Q�l����\cite{DJ2006}�t�^E�̃z�e�������̌��|���̗��̌��ׂ��C�����A���{�ꉻ���A���s�”\�d�l�Ƃ����B

\end{abstract}
%\vspace{-1cm}

\tableofcontents
\newpage

\section{���̊T�v}
�z�e���̋q�́A�`�F�b�N�C�����ɁA�t�����g�ŁA�����̌��ƂȂ�J�[�h���󂯎��B

�J�[�h��2�‚̌��������A���̌��͂��̌�ς�邱�Ƃ͂Ȃ��B

�t�����g�́A����܂łɔ��s�����J�[�h�ƁA�g�p�ς݂̃J�[�h���L�^���Ă���B

�q���A�J�[�h���g���ĕ����ɓ���ƁA�O�̋q�̃J�[�h�ł͕����ɓ���Ȃ��Ȃ�B

�q�́A�ēx�`�F�b�N�C�����邱�ƂŁA�V���Ȍ��ƂȂ�J�[�h���󂯎�邱�Ƃ��ł��邪�A�V�����J�[�h���g���ƁA�Â��J�[�h�ł͕����ɓ���Ȃ��Ȃ�B
\section {はじめに}
本ドキュメントは、VDM++関数型ライブラリ・ドキュメントの雛形であり、
まだ、すべてのVDMモジュールを記述しているわけではない。

\section {関数型ライブラリのドキュメント}
\include{AllT.vpp}
% LaTeX 2e Document.
% 
% $Id: BusinessTable.tex,v 1.2 2006/01/10 10:46:35 vdmtools Exp $
% 

%%%%%%%%%%%%%%%%%%%%%%%%%%%%%%%%%%%%%%%%
% PDF compatibility code. 
\makeatletter
\newif\ifpdflatex@
\ifx\pdftexversion\@undefined
\pdflatex@false
%\message{Not using pdf}
\else
\pdflatex@true
%\message{Using pdf}
\fi

\newcommand{\latexorpdf}[2]{
  \ifpdflatex@ #2
  \else #1
  \fi
}

\newcommand{\pformat}{a4paper}

\makeatother
%%%%%%%%%%%%%%%%%%%%%%%%%%%%%%%%%%%%%%%%

%\latexorpdf{
%\documentclass[\pformat,12pt]{article}
%}{
%\documentclass[\pformat,pdftex,12pt]{article}
%}
\documentclass[]{jarticle}

\usepackage[dvips]{color}
\usepackage{array}
\usepackage{longtable}
\usepackage{alltt}
\usepackage{graphics}
\usepackage{vpp}
\usepackage{makeidx}
\makeindex

\definecolor{covered}{rgb}{0,0,0}      %black
%\definecolor{not-covered}{gray}{0.5}   %gray for previewing
%\definecolor{not-covered}{gray}{0.6}   %gray for printing
\definecolor{not-covered}{rgb}{1,0,0}  %red

\newcommand{\InstVarDef}[1]{{\bf #1}}
\newcommand{\TypeDef}[1]{{\bf #1}}
\newcommand{\TypeOcc}[1]{{\it #1}}
\newcommand{\FuncDef}[1]{{\bf #1}}
\newcommand{\FuncOcc}[1]{#1}
\newcommand{\MethodDef}[1]{{\bf #1}}
\newcommand{\MethodOcc}[1]{#1}
\newcommand{\ClassDef}[1]{{\sf #1}}
\newcommand{\ClassOcc}[1]{#1}
\newcommand{\ModDef}[1]{{\sf #1}}
\newcommand{\ModOcc}[1]{#1}

%\nolinenumbering
%\setindent{outer}{\parindent}
%\setindent{inner}{0.0em}

\title{FBusinessTable���C�u�����[}
\author{
�����L
���{�t�B�b�c�������\\
���Z�p������\\
TEL : 03-3623-4683\\
shin.sahara@jfits.co.jp\\
}
%\date{2004�N3��16��}

\begin{document}
\setlength{\baselineskip}{12pt plus .1pt}
\tolerance 10000
\maketitle

\begin{abstract}
\setlength{\baselineskip}{12pt plus .1pt}
�r�W�l�X��̋K���Ɏg�p�����\�Ɋւ��֐���񋟂��郂�W���[���ł���B

% LaTeX 2e Document.
% 
% $Id: BusinessTable.tex,v 1.2 2006/01/10 10:46:35 vdmtools Exp $
% 

%%%%%%%%%%%%%%%%%%%%%%%%%%%%%%%%%%%%%%%%
% PDF compatibility code. 
\makeatletter
\newif\ifpdflatex@
\ifx\pdftexversion\@undefined
\pdflatex@false
%\message{Not using pdf}
\else
\pdflatex@true
%\message{Using pdf}
\fi

\newcommand{\latexorpdf}[2]{
  \ifpdflatex@ #2
  \else #1
  \fi
}

\newcommand{\pformat}{a4paper}

\makeatother
%%%%%%%%%%%%%%%%%%%%%%%%%%%%%%%%%%%%%%%%

%\latexorpdf{
%\documentclass[\pformat,12pt]{article}
%}{
%\documentclass[\pformat,pdftex,12pt]{article}
%}
\documentclass[]{jarticle}

\usepackage[dvips]{color}
\usepackage{array}
\usepackage{longtable}
\usepackage{alltt}
\usepackage{graphics}
\usepackage{vpp}
\usepackage{makeidx}
\makeindex

\definecolor{covered}{rgb}{0,0,0}      %black
%\definecolor{not-covered}{gray}{0.5}   %gray for previewing
%\definecolor{not-covered}{gray}{0.6}   %gray for printing
\definecolor{not-covered}{rgb}{1,0,0}  %red

\newcommand{\InstVarDef}[1]{{\bf #1}}
\newcommand{\TypeDef}[1]{{\bf #1}}
\newcommand{\TypeOcc}[1]{{\it #1}}
\newcommand{\FuncDef}[1]{{\bf #1}}
\newcommand{\FuncOcc}[1]{#1}
\newcommand{\MethodDef}[1]{{\bf #1}}
\newcommand{\MethodOcc}[1]{#1}
\newcommand{\ClassDef}[1]{{\sf #1}}
\newcommand{\ClassOcc}[1]{#1}
\newcommand{\ModDef}[1]{{\sf #1}}
\newcommand{\ModOcc}[1]{#1}

%\nolinenumbering
%\setindent{outer}{\parindent}
%\setindent{inner}{0.0em}

\title{FBusinessTable���C�u�����[}
\author{
�����L
���{�t�B�b�c�������\\
���Z�p������\\
TEL : 03-3623-4683\\
shin.sahara@jfits.co.jp\\
}
%\date{2004�N3��16��}

\begin{document}
\setlength{\baselineskip}{12pt plus .1pt}
\tolerance 10000
\maketitle

\begin{abstract}
\setlength{\baselineskip}{12pt plus .1pt}
�r�W�l�X��̋K���Ɏg�p�����\�Ɋւ��֐���񋟂��郂�W���[���ł���B

% LaTeX 2e Document.
% 
% $Id: BusinessTable.tex,v 1.2 2006/01/10 10:46:35 vdmtools Exp $
% 

%%%%%%%%%%%%%%%%%%%%%%%%%%%%%%%%%%%%%%%%
% PDF compatibility code. 
\makeatletter
\newif\ifpdflatex@
\ifx\pdftexversion\@undefined
\pdflatex@false
%\message{Not using pdf}
\else
\pdflatex@true
%\message{Using pdf}
\fi

\newcommand{\latexorpdf}[2]{
  \ifpdflatex@ #2
  \else #1
  \fi
}

\newcommand{\pformat}{a4paper}

\makeatother
%%%%%%%%%%%%%%%%%%%%%%%%%%%%%%%%%%%%%%%%

%\latexorpdf{
%\documentclass[\pformat,12pt]{article}
%}{
%\documentclass[\pformat,pdftex,12pt]{article}
%}
\documentclass[]{jarticle}

\usepackage[dvips]{color}
\usepackage{array}
\usepackage{longtable}
\usepackage{alltt}
\usepackage{graphics}
\usepackage{vpp}
\usepackage{makeidx}
\makeindex

\definecolor{covered}{rgb}{0,0,0}      %black
%\definecolor{not-covered}{gray}{0.5}   %gray for previewing
%\definecolor{not-covered}{gray}{0.6}   %gray for printing
\definecolor{not-covered}{rgb}{1,0,0}  %red

\newcommand{\InstVarDef}[1]{{\bf #1}}
\newcommand{\TypeDef}[1]{{\bf #1}}
\newcommand{\TypeOcc}[1]{{\it #1}}
\newcommand{\FuncDef}[1]{{\bf #1}}
\newcommand{\FuncOcc}[1]{#1}
\newcommand{\MethodDef}[1]{{\bf #1}}
\newcommand{\MethodOcc}[1]{#1}
\newcommand{\ClassDef}[1]{{\sf #1}}
\newcommand{\ClassOcc}[1]{#1}
\newcommand{\ModDef}[1]{{\sf #1}}
\newcommand{\ModOcc}[1]{#1}

%\nolinenumbering
%\setindent{outer}{\parindent}
%\setindent{inner}{0.0em}

\title{FBusinessTable���C�u�����[}
\author{
�����L
���{�t�B�b�c�������\\
���Z�p������\\
TEL : 03-3623-4683\\
shin.sahara@jfits.co.jp\\
}
%\date{2004�N3��16��}

\begin{document}
\setlength{\baselineskip}{12pt plus .1pt}
\tolerance 10000
\maketitle

\begin{abstract}
\setlength{\baselineskip}{12pt plus .1pt}
�r�W�l�X��̋K���Ɏg�p�����\�Ɋւ��֐���񋟂��郂�W���[���ł���B

\include{test/BusinessTable.vpp}

\include{test/BusinessTableT.vpp}

%\newpage
%\addcontentsline{toc}{section}{Index}
%\printindex

\end{document}


\include{test/BusinessTableT.vpp}

%\newpage
%\addcontentsline{toc}{section}{Index}
%\printindex

\end{document}


\include{test/BusinessTableT.vpp}

%\newpage
%\addcontentsline{toc}{section}{Index}
%\printindex

\end{document}

\include{FBusinessTableT.vpp}
% LaTeX 2e Document.
% 
% $Id: Calendar.tex,v 1.2 2006/01/10 10:46:35 vdmtools Exp $
% 

%%%%%%%%%%%%%%%%%%%%%%%%%%%%%%%%%%%%%%%%
% PDF compatibility code. 
\makeatletter
\newif\ifpdflatex@
\ifx\pdftexversion\@undefined
\pdflatex@false
%\message{Not using pdf}
\else
\pdflatex@true
%\message{Using pdf}
\fi

\newcommand{\latexorpdf}[2]{
  \ifpdflatex@ #2
  \else #1
  \fi
}

\newcommand{\pformat}{a4paper}

\makeatother
%%%%%%%%%%%%%%%%%%%%%%%%%%%%%%%%%%%%%%%%

%\latexorpdf{
%\documentclass[\pformat,12pt]{article}
%}{
%\documentclass[\pformat,pdftex,12pt]{article}
%}
\documentclass[]{jarticle}

\usepackage[dvips]{color}
\usepackage{array}
\usepackage{longtable}
\usepackage{alltt}
\usepackage{graphics}
\usepackage{vpp}
\usepackage{makeidx}
\makeindex

\definecolor{covered}{rgb}{0,0,0}      %black
%\definecolor{not-covered}{gray}{0.5}   %gray for previewing
%\definecolor{not-covered}{gray}{0.6}   %gray for printing
\definecolor{not-covered}{rgb}{1,0,0}  %red

\newcommand{\InstVarDef}[1]{{\bf #1}}
\newcommand{\TypeDef}[1]{{\bf #1}}
\newcommand{\TypeOcc}[1]{{\it #1}}
\newcommand{\FuncDef}[1]{{\bf #1}}
\newcommand{\FuncOcc}[1]{#1}
\newcommand{\MethodDef}[1]{{\bf #1}}
\newcommand{\MethodOcc}[1]{#1}
\newcommand{\ClassDef}[1]{{\sf #1}}
\newcommand{\ClassOcc}[1]{#1}
\newcommand{\ModDef}[1]{{\sf #1}}
\newcommand{\ModOcc}[1]{#1}

%\nolinenumbering
\linenumbering
\setindent{outer}{\parindent}
%\setindent{inner}{0.0em}

\title{FCalendar���C�u����}
\author{
�����L\\
ss@shinsahara.com\\
}
\date{2005�N12��31��}

\begin{document}
\setlength{\baselineskip}{12pt plus .1pt}
\tolerance 10000
\maketitle

\begin{abstract}
\setlength{\baselineskip}{12pt plus .1pt}
��i�J�����_�[�j�Ɋւ��֐���񋟂���A��������t�v�Z�̂��߂̃��W���[���ł���B
�{���W���[���́A�O���S���I��ؑ֓��i1582�N10��15���j�Ȍ�̓O���S���I����A������O�i1582�N10��4���ȑO�j�̓����E�X��̓��t���g�p����B
�e���ł̃O���S���I��ؑ֓��͈قȂ�i���{�̏ꍇ1873�N�j�̂ŁA���j��̗�v�Z�͒��ӂ��Ȃ���΂Ȃ�Ȃ��B

�{���W���[���́A����t���C�������E�X��\footnote{��q�B}�ŕ\���Ă��邽�߁A
���̗L�����͈͓̔��i�I���O4294288353+1�N����A4295646239�N�܂Łj�Ŏg�����Ƃ��ł���͂��ł���B
�����Ƃ��A�O���S���I���4909�N�ɂP�������\��Ȃ̂ŁA���ۂɂ͂��̂�����܂ł��L���ł��낤�B
���Ȃ��Ƃ��A�O���S���I��֑ؑO���i1582�N10��4���j����A2099�N�̏H���܂ł̓e�X�g�Ŋm�F���Ă���B
3000�N�̏t���Ȍ�́A�H���E�H���̌v�Z�͌��݂̓V���ϑ��v�Z�̐��x�ƌ덷�̂��ߓ���ł��Ȃ��̂ŁA�H���E�H���̌v�Z��3000�N�ȍ~�g���Ȃ��B

�Ȃ��A�{���W���[���́A�����E�X��̉[�N�̌v�Z���I���O�ƋI����Ƃœ���ɂ��邱�Ƃ��ł���悤�A�����E�X��̋I���O�P�N��0�N�܂���-0�N�ƕ\���A�I���O4713�N��-4712�N�ƕ\���B���Ȃ킿�A�I���O�̔N��\���Ƃ��́A�ȉ��̎������藧�B
\begin{verbatim}
	�{���W���[���̃����E�X��̔N = �����E�X��̔N + 1
\end{verbatim}

�����Ɋւ��v�Z�̓O���j�b�W�W�������g�p���Ă���̂ŁA�{�N���X�̃T�u�N���X�Ƃ��Ċe���ʁE�W�����ʂ̗�N���X���쐬���A\textbf{�O���j�b�W�W�����Ƃ̍�}�ɑ�������l��ݒ肵�Ȃ���΂Ȃ�Ȃ��B
\end{abstract}
%\vspace{-1cm}

\subsection{�p��̒�`}
�ӂȂ΂����悵���ɂ��p��̒�`\footnote{http://www.funaba.org/}���Q�l�ɋL�q����B

\subsubsection{����t}
����t�́A�ʏ�̔N�����ɂ����t�ł���B

\subsubsection{�����E�X��}
�I���O4713�N1��1���i�����E�X��j���߁i�O���j�b�W���ώ��j���Ƃ����ʓ��i�o�ߓ����j�ł���B

\subsubsection{�C�������E�X��}
1858�N11��17���i�O���S���I��j0���i���萢�E���j���Ƃ����ʓ��i�o�ߓ����j�ł���B���Ƃ��Ƃ̓����E�X�����̌v�Z�ɗp���Ă������A���������ӂ�Ȃ��悤�ɏC�����������E�X���ł���B

\subsubsection{���t}
�{���W���[���œ��t�Ƃ����ꍇ�A�C�������E�X����\��Date�^�i���͎̂����^�j�̒l�������B

\subsubsection{�N���t�i�N�ԒʎZ���j}
�N�̒��̏����ɂ���Ďw�肳�����t�ŁA�Ⴆ�΁A1��10���̔N���t��10�ŁA2��1���̔N���t��32�ł���B

\subsubsection{�����N���t}
���������N�A�����_�ȉ���\textbf{�N���t / �N�ԑ�����}��\���`���̓��t�B���t�v�Z�̉ߒ��Ŏg�p����̂ŁA�ʏ�͋C�ɂ���K�v�͂Ȃ��B
�Ⴆ�΁A����t2001�N7��1���́A�����N���t�ł�2001.5�ƂȂ�B

\subsubsection{�����t�i���ԒʎZ���j}
���̒��̏����ɂ���Ďw�肳�����t�ŁA�Ⴆ�΁A1��31���̌����t��31�ŁA2��1���̌����t��1�ł���B

\subsection{�ӎ�}
�{���W���[���̎Z�@�́ANifty-Serve�V���v�Z�t�H�[�����̕��X�̏������Q�l�Ɏ��������B
�܂��AHowManyDayOfWeekWithin2Days�֐��̎Z�@�́A�R�藘������̃A�C�f�A�ɂ����̂ł���B

\subsection{���j}
�{���W���[���̍ŏ��̔ł�Digitalk Smalltalk�Ŏ��������B���ɁAVisualWorks�iSmalltalk�̖{�Ɓj�ɈڐA���A���̌�֐��^�v���O���~���O����Concurrent Clean�ɈڐA���A�Ō��VDM++�ɈڐA�����B
�{���C�u�����[�́A�iSSLib�Ə̂��郉�C�u�����[�Ɋ܂܂��j�I�u�W�F�N�g�w���ō쐬����VDM++�ł��A����Ɋ֐��^�w���ɈڐA�������̂ł���B
�Ȃ��A�������̃f�B�X�N�g���u�����o�āADigitalk Smalltalk�ł�VisualWorks�ł͌������Ȃ��B�����Ƃ��A�o�b�N�A�b�v�p�̃f�B�X�N�ȂǂɎc���Ă���”\���͂��邪�A�܂��m�F���Ă��Ȃ��B

% LaTeX 2e Document.
% 
% $Id: Calendar.tex,v 1.3 2006/04/19 05:06:48 vdmtools Exp $
% 

%%%%%%%%%%%%%%%%%%%%%%%%%%%%%%%%%%%%%%%%
% PDF compatibility code. 
\makeatletter
\newif\ifpdflatex@
\ifx\pdftexversion\@undefined
\pdflatex@false
%\message{Not using pdf}
\else
\pdflatex@true
%\message{Using pdf}
\fi

\newcommand{\latexorpdf}[2]{
  \ifpdflatex@ #2
  \else #1
  \fi
}

\newcommand{\pformat}{a4paper}

\makeatother
%%%%%%%%%%%%%%%%%%%%%%%%%%%%%%%%%%%%%%%%

%\latexorpdf{
%\documentclass[\pformat,12pt]{article}
%}{
%\documentclass[\pformat,pdftex,12pt]{article}
%}
\documentclass[]{jarticle}

\usepackage[dvips]{color}
\usepackage{array}
\usepackage{longtable}
\usepackage{alltt}
\usepackage{graphics}
\usepackage{vpp}
\usepackage{makeidx}
\makeindex

\definecolor{covered}{rgb}{0,0,0}      %black
%\definecolor{not-covered}{gray}{0.5}   %gray for previewing
%\definecolor{not-covered}{gray}{0.6}   %gray for printing
\definecolor{not-covered}{rgb}{1,0,0}  %red

\newcommand{\InstVarDef}[1]{{\bf #1}}
\newcommand{\TypeDef}[1]{{\bf #1}}
\newcommand{\TypeOcc}[1]{{\it #1}}
\newcommand{\FuncDef}[1]{{\bf #1}}
\newcommand{\FuncOcc}[1]{#1}
\newcommand{\MethodDef}[1]{{\bf #1}}
\newcommand{\MethodOcc}[1]{#1}
\newcommand{\ClassDef}[1]{{\sf #1}}
\newcommand{\ClassOcc}[1]{#1}
\newcommand{\ModDef}[1]{{\sf #1}}
\newcommand{\ModOcc}[1]{#1}

%\nolinenumbering
\linenumbering
\setindent{outer}{\parindent}
%\setindent{inner}{0.0em}

\title{FCalendar���C�u����}
\author{
�����L\\
ss@shinsahara.com\\
}
\date{2005�N12��31��}

\begin{document}
\setlength{\baselineskip}{12pt plus .1pt}
\tolerance 10000
\maketitle

\begin{abstract}
\setlength{\baselineskip}{12pt plus .1pt}
��i�J�����_�[�j�Ɋւ��֐���񋟂���A��������t�v�Z�̂��߂̃��W���[���ł���B
�{���W���[���́A�O���S���I��ؑ֓��i1582�N10��15���j�Ȍ�̓O���S���I����A������O�i1582�N10��4���ȑO�j�̓����E�X��̓��t���g�p����B
�e���ł̃O���S���I��ؑ֓��͈قȂ�i���{�̏ꍇ1873�N�j�̂ŁA���j��̗�v�Z�͒��ӂ��Ȃ���΂Ȃ�Ȃ��B

�{���W���[���́A����t���C�������E�X��\footnote{��q�B}�ŕ\���Ă��邽�߁A
���̗L�����͈͓̔��i�I���O4294288353+1�N����A4295646239�N�܂Łj�Ŏg�����Ƃ��ł���͂��ł���B
�����Ƃ��A�O���S���I���4909�N�ɂP�������\��Ȃ̂ŁA���ۂɂ͂��̂�����܂ł��L���ł��낤�B
���Ȃ��Ƃ��A�O���S���I��֑ؑO���i1582�N10��4���j����A2099�N�̏H���܂ł̓e�X�g�Ŋm�F���Ă���B
3000�N�̏t���Ȍ�́A�H���E�H���̌v�Z�͌��݂̓V���ϑ��v�Z�̐��x�ƌ덷�̂��ߓ���ł��Ȃ��̂ŁA�H���E�H���̌v�Z��3000�N�ȍ~�g���Ȃ��B

�Ȃ��A�{���W���[���́A�����E�X��̉[�N�̌v�Z���I���O�ƋI����Ƃœ���ɂ��邱�Ƃ��ł���悤�A�����E�X��̋I���O�P�N��0�N�܂���-0�N�ƕ\���A�I���O4713�N��-4712�N�ƕ\���B���Ȃ킿�A�I���O�̔N��\���Ƃ��́A�ȉ��̎������藧�B
\begin{verbatim}
	�{���W���[���̃����E�X��̔N = �����E�X��̔N + 1
\end{verbatim}

�����Ɋւ��v�Z�̓O���j�b�W�W�������g�p���Ă���̂ŁA�{�N���X�̃T�u�N���X�Ƃ��Ċe���ʁE�W�����ʂ̗�N���X���쐬���A\textbf{�O���j�b�W�W�����Ƃ̍�}�ɑ�������l��ݒ肵�Ȃ���΂Ȃ�Ȃ��B
\end{abstract}
%\vspace{-1cm}

\subsection{�p��̒�`}
�ӂȂ΂����悵���ɂ��p��̒�`\footnote{http://www.funaba.org/}���Q�l�ɋL�q����B

\subsubsection{����t}
����t�́A�ʏ�̔N�����ɂ����t�ł���B

\subsubsection{�����E�X��}
�I���O4713�N1��1���i�����E�X��j���߁i�O���j�b�W���ώ��j���Ƃ����ʓ��i�o�ߓ����j�ł���B

\subsubsection{�C�������E�X��}
1858�N11��17���i�O���S���I��j0���i���萢�E���j���Ƃ����ʓ��i�o�ߓ����j�ł���B���Ƃ��Ƃ̓����E�X�����̌v�Z�ɗp���Ă������A���������ӂ�Ȃ��悤�ɏC�����������E�X���ł���B

\subsubsection{���t}
�{���W���[���œ��t�Ƃ����ꍇ�A�C�������E�X����\��Date�^�i���͎̂����^�j�̒l�������B

\subsubsection{�N���t�i�N�ԒʎZ���j}
�N�̒��̏����ɂ���Ďw�肳�����t�ŁA�Ⴆ�΁A1��10���̔N���t��10�ŁA2��1���̔N���t��32�ł���B

\subsubsection{�����N���t}
���������N�A�����_�ȉ���\textbf{�N���t / �N�ԑ�����}��\���`���̓��t�B���t�v�Z�̉ߒ��Ŏg�p����̂ŁA�ʏ�͋C�ɂ���K�v�͂Ȃ��B
�Ⴆ�΁A����t2001�N7��1���́A�����N���t�ł�2001.5�ƂȂ�B

\subsubsection{�����t�i���ԒʎZ���j}
���̒��̏����ɂ���Ďw�肳�����t�ŁA�Ⴆ�΁A1��31���̌����t��31�ŁA2��1���̌����t��1�ł���B

\subsection{�ӎ�}
�{���W���[���̎Z�@�́ANifty-Serve�V���v�Z�t�H�[�����̕��X�̏������Q�l�Ɏ��������B
�܂��AHowManyDayOfWeekWithin2Days�֐��̎Z�@�́A�R�藘������̃A�C�f�A�ɂ����̂ł���B

\subsection{���j}
�{���W���[���̍ŏ��̔ł�Digitalk Smalltalk�Ŏ��������B���ɁAVisualWorks�iSmalltalk�̖{�Ɓj�ɈڐA���A���̌�֐��^�v���O���~���O����Concurrent Clean�ɈڐA���A�Ō��VDM++�ɈڐA�����B
�{���C�u�����[�́A�iSSLib�Ə̂��郉�C�u�����[�Ɋ܂܂��j�I�u�W�F�N�g�w���ō쐬����VDM++�ł��A����Ɋ֐��^�w���ɈڐA�������̂ł���B
�Ȃ��A�������̃f�B�X�N�g���u�����o�āADigitalk Smalltalk�ł�VisualWorks�ł͌������Ȃ��B�����Ƃ��A�o�b�N�A�b�v�p�̃f�B�X�N�ȂǂɎc���Ă���”\���͂��邪�A�܂��m�F���Ă��Ȃ��B

% LaTeX 2e Document.
% 
% $Id: Calendar.tex,v 1.3 2006/04/19 05:06:48 vdmtools Exp $
% 

%%%%%%%%%%%%%%%%%%%%%%%%%%%%%%%%%%%%%%%%
% PDF compatibility code. 
\makeatletter
\newif\ifpdflatex@
\ifx\pdftexversion\@undefined
\pdflatex@false
%\message{Not using pdf}
\else
\pdflatex@true
%\message{Using pdf}
\fi

\newcommand{\latexorpdf}[2]{
  \ifpdflatex@ #2
  \else #1
  \fi
}

\newcommand{\pformat}{a4paper}

\makeatother
%%%%%%%%%%%%%%%%%%%%%%%%%%%%%%%%%%%%%%%%

%\latexorpdf{
%\documentclass[\pformat,12pt]{article}
%}{
%\documentclass[\pformat,pdftex,12pt]{article}
%}
\documentclass[]{jarticle}

\usepackage[dvips]{color}
\usepackage{array}
\usepackage{longtable}
\usepackage{alltt}
\usepackage{graphics}
\usepackage{vpp}
\usepackage{makeidx}
\makeindex

\definecolor{covered}{rgb}{0,0,0}      %black
%\definecolor{not-covered}{gray}{0.5}   %gray for previewing
%\definecolor{not-covered}{gray}{0.6}   %gray for printing
\definecolor{not-covered}{rgb}{1,0,0}  %red

\newcommand{\InstVarDef}[1]{{\bf #1}}
\newcommand{\TypeDef}[1]{{\bf #1}}
\newcommand{\TypeOcc}[1]{{\it #1}}
\newcommand{\FuncDef}[1]{{\bf #1}}
\newcommand{\FuncOcc}[1]{#1}
\newcommand{\MethodDef}[1]{{\bf #1}}
\newcommand{\MethodOcc}[1]{#1}
\newcommand{\ClassDef}[1]{{\sf #1}}
\newcommand{\ClassOcc}[1]{#1}
\newcommand{\ModDef}[1]{{\sf #1}}
\newcommand{\ModOcc}[1]{#1}

%\nolinenumbering
\linenumbering
\setindent{outer}{\parindent}
%\setindent{inner}{0.0em}

\title{FCalendar���C�u����}
\author{
�����L\\
ss@shinsahara.com\\
}
\date{2005�N12��31��}

\begin{document}
\setlength{\baselineskip}{12pt plus .1pt}
\tolerance 10000
\maketitle

\begin{abstract}
\setlength{\baselineskip}{12pt plus .1pt}
��i�J�����_�[�j�Ɋւ��֐���񋟂���A��������t�v�Z�̂��߂̃��W���[���ł���B
�{���W���[���́A�O���S���I��ؑ֓��i1582�N10��15���j�Ȍ�̓O���S���I����A������O�i1582�N10��4���ȑO�j�̓����E�X��̓��t���g�p����B
�e���ł̃O���S���I��ؑ֓��͈قȂ�i���{�̏ꍇ1873�N�j�̂ŁA���j��̗�v�Z�͒��ӂ��Ȃ���΂Ȃ�Ȃ��B

�{���W���[���́A����t���C�������E�X��\footnote{��q�B}�ŕ\���Ă��邽�߁A
���̗L�����͈͓̔��i�I���O4294288353+1�N����A4295646239�N�܂Łj�Ŏg�����Ƃ��ł���͂��ł���B
�����Ƃ��A�O���S���I���4909�N�ɂP�������\��Ȃ̂ŁA���ۂɂ͂��̂�����܂ł��L���ł��낤�B
���Ȃ��Ƃ��A�O���S���I��֑ؑO���i1582�N10��4���j����A2099�N�̏H���܂ł̓e�X�g�Ŋm�F���Ă���B
3000�N�̏t���Ȍ�́A�H���E�H���̌v�Z�͌��݂̓V���ϑ��v�Z�̐��x�ƌ덷�̂��ߓ���ł��Ȃ��̂ŁA�H���E�H���̌v�Z��3000�N�ȍ~�g���Ȃ��B

�Ȃ��A�{���W���[���́A�����E�X��̉[�N�̌v�Z���I���O�ƋI����Ƃœ���ɂ��邱�Ƃ��ł���悤�A�����E�X��̋I���O�P�N��0�N�܂���-0�N�ƕ\���A�I���O4713�N��-4712�N�ƕ\���B���Ȃ킿�A�I���O�̔N��\���Ƃ��́A�ȉ��̎������藧�B
\begin{verbatim}
	�{���W���[���̃����E�X��̔N = �����E�X��̔N + 1
\end{verbatim}

�����Ɋւ��v�Z�̓O���j�b�W�W�������g�p���Ă���̂ŁA�{�N���X�̃T�u�N���X�Ƃ��Ċe���ʁE�W�����ʂ̗�N���X���쐬���A\textbf{�O���j�b�W�W�����Ƃ̍�}�ɑ�������l��ݒ肵�Ȃ���΂Ȃ�Ȃ��B
\end{abstract}
%\vspace{-1cm}

\subsection{�p��̒�`}
�ӂȂ΂����悵���ɂ��p��̒�`\footnote{http://www.funaba.org/}���Q�l�ɋL�q����B

\subsubsection{����t}
����t�́A�ʏ�̔N�����ɂ����t�ł���B

\subsubsection{�����E�X��}
�I���O4713�N1��1���i�����E�X��j���߁i�O���j�b�W���ώ��j���Ƃ����ʓ��i�o�ߓ����j�ł���B

\subsubsection{�C�������E�X��}
1858�N11��17���i�O���S���I��j0���i���萢�E���j���Ƃ����ʓ��i�o�ߓ����j�ł���B���Ƃ��Ƃ̓����E�X�����̌v�Z�ɗp���Ă������A���������ӂ�Ȃ��悤�ɏC�����������E�X���ł���B

\subsubsection{���t}
�{���W���[���œ��t�Ƃ����ꍇ�A�C�������E�X����\��Date�^�i���͎̂����^�j�̒l�������B

\subsubsection{�N���t�i�N�ԒʎZ���j}
�N�̒��̏����ɂ���Ďw�肳�����t�ŁA�Ⴆ�΁A1��10���̔N���t��10�ŁA2��1���̔N���t��32�ł���B

\subsubsection{�����N���t}
���������N�A�����_�ȉ���\textbf{�N���t / �N�ԑ�����}��\���`���̓��t�B���t�v�Z�̉ߒ��Ŏg�p����̂ŁA�ʏ�͋C�ɂ���K�v�͂Ȃ��B
�Ⴆ�΁A����t2001�N7��1���́A�����N���t�ł�2001.5�ƂȂ�B

\subsubsection{�����t�i���ԒʎZ���j}
���̒��̏����ɂ���Ďw�肳�����t�ŁA�Ⴆ�΁A1��31���̌����t��31�ŁA2��1���̌����t��1�ł���B

\subsection{�ӎ�}
�{���W���[���̎Z�@�́ANifty-Serve�V���v�Z�t�H�[�����̕��X�̏������Q�l�Ɏ��������B
�܂��AHowManyDayOfWeekWithin2Days�֐��̎Z�@�́A�R�藘������̃A�C�f�A�ɂ����̂ł���B

\subsection{���j}
�{���W���[���̍ŏ��̔ł�Digitalk Smalltalk�Ŏ��������B���ɁAVisualWorks�iSmalltalk�̖{�Ɓj�ɈڐA���A���̌�֐��^�v���O���~���O����Concurrent Clean�ɈڐA���A�Ō��VDM++�ɈڐA�����B
�{���C�u�����[�́A�iSSLib�Ə̂��郉�C�u�����[�Ɋ܂܂��j�I�u�W�F�N�g�w���ō쐬����VDM++�ł��A����Ɋ֐��^�w���ɈڐA�������̂ł���B
�Ȃ��A�������̃f�B�X�N�g���u�����o�āADigitalk Smalltalk�ł�VisualWorks�ł͌������Ȃ��B�����Ƃ��A�o�b�N�A�b�v�p�̃f�B�X�N�ȂǂɎc���Ă���”\���͂��邪�A�܂��m�F���Ă��Ȃ��B

\include{test/Calendar.vpp}

\include{test/CalendarT.vpp}

%\newpage
%\addcontentsline{toc}{section}{Index}
%\printindex

\end{document}


\include{test/CalendarT.vpp}

%\newpage
%\addcontentsline{toc}{section}{Index}
%\printindex

\end{document}


\include{test/CalendarT.vpp}

%\newpage
%\addcontentsline{toc}{section}{Index}
%\printindex

\end{document}
	
%\include{FCalendarT.vpp}
% LaTeX 2e Document.
% 
% $Id: Character.tex,v 1.2 2006/01/10 10:46:35 vdmtools Exp $
% 

%%%%%%%%%%%%%%%%%%%%%%%%%%%%%%%%%%%%%%%%
% PDF compatibility code. 
\makeatletter
\newif\ifpdflatex@
\ifx\pdftexversion\@undefined
\pdflatex@false
%\message{Not using pdf}
\else
\pdflatex@true
%\message{Using pdf}
\fi

\newcommand{\latexorpdf}[2]{
  \ifpdflatex@ #2
  \else #1
  \fi
}

\newcommand{\pformat}{a4paper}

\makeatother
%%%%%%%%%%%%%%%%%%%%%%%%%%%%%%%%%%%%%%%%

%\latexorpdf{
%\documentclass[\pformat,12pt]{article}
%}{
%\documentclass[\pformat,pdftex,12pt]{article}
%}
\documentclass[]{jsarticle}

\usepackage[dvips]{color}
\usepackage{array}
\usepackage{longtable}
\usepackage{alltt}
\usepackage{graphics}
\usepackage{vpp}
\usepackage{makeidx}
\makeindex

\definecolor{covered}{rgb}{0,0,0}      %black
%\definecolor{not-covered}{gray}{0.5}   %gray for previewing
%\definecolor{not-covered}{gray}{0.6}   %gray for printing
\definecolor{not-covered}{rgb}{1,0,0}  %red

\newcommand{\InstVarDef}[1]{{\bf #1}}
\newcommand{\TypeDef}[1]{{\bf #1}}
\newcommand{\TypeOcc}[1]{{\it #1}}
\newcommand{\FuncDef}[1]{{\bf #1}}
\newcommand{\FuncOcc}[1]{#1}
\newcommand{\MethodDef}[1]{{\bf #1}}
\newcommand{\MethodOcc}[1]{#1}
\newcommand{\ClassDef}[1]{{\sf #1}}
\newcommand{\ClassOcc}[1]{#1}
\newcommand{\ModDef}[1]{{\sf #1}}
\newcommand{\ModOcc}[1]{#1}

%\nolinenumbering
%\setindent{outer}{\parindent}
%\setindent{inner}{0.0em}

\title{FCharacter���C�u�����[}
\author{
�����L
���{�t�B�b�c�������\\
���Z�p������\\
TEL : 03-3623-4683\\
shin.sahara@jfits.co.jp\\
}
%\date{2004�N2��16��}

\begin{document}
\setlength{\baselineskip}{12pt plus .1pt}
\tolerance 10000
\maketitle

\begin{abstract}
\setlength{\baselineskip}{12pt plus .1pt}
�����ichar�j�^�Ɋւ��֐���񋟂��郂�W���[���ł���B
\end{abstract}
%\vspace{-1cm}

% LaTeX 2e Document.
% 
% $Id: Character.tex,v 1.2 2006/01/10 10:46:35 vdmtools Exp $
% 

%%%%%%%%%%%%%%%%%%%%%%%%%%%%%%%%%%%%%%%%
% PDF compatibility code. 
\makeatletter
\newif\ifpdflatex@
\ifx\pdftexversion\@undefined
\pdflatex@false
%\message{Not using pdf}
\else
\pdflatex@true
%\message{Using pdf}
\fi

\newcommand{\latexorpdf}[2]{
  \ifpdflatex@ #2
  \else #1
  \fi
}

\newcommand{\pformat}{a4paper}

\makeatother
%%%%%%%%%%%%%%%%%%%%%%%%%%%%%%%%%%%%%%%%

%\latexorpdf{
%\documentclass[\pformat,12pt]{article}
%}{
%\documentclass[\pformat,pdftex,12pt]{article}
%}
\documentclass[]{jsarticle}

\usepackage[dvips]{color}
\usepackage{array}
\usepackage{longtable}
\usepackage{alltt}
\usepackage{graphics}
\usepackage{vpp}
\usepackage{makeidx}
\makeindex

\definecolor{covered}{rgb}{0,0,0}      %black
%\definecolor{not-covered}{gray}{0.5}   %gray for previewing
%\definecolor{not-covered}{gray}{0.6}   %gray for printing
\definecolor{not-covered}{rgb}{1,0,0}  %red

\newcommand{\InstVarDef}[1]{{\bf #1}}
\newcommand{\TypeDef}[1]{{\bf #1}}
\newcommand{\TypeOcc}[1]{{\it #1}}
\newcommand{\FuncDef}[1]{{\bf #1}}
\newcommand{\FuncOcc}[1]{#1}
\newcommand{\MethodDef}[1]{{\bf #1}}
\newcommand{\MethodOcc}[1]{#1}
\newcommand{\ClassDef}[1]{{\sf #1}}
\newcommand{\ClassOcc}[1]{#1}
\newcommand{\ModDef}[1]{{\sf #1}}
\newcommand{\ModOcc}[1]{#1}

%\nolinenumbering
%\setindent{outer}{\parindent}
%\setindent{inner}{0.0em}

\title{FCharacter���C�u�����[}
\author{
�����L
���{�t�B�b�c�������\\
���Z�p������\\
TEL : 03-3623-4683\\
shin.sahara@jfits.co.jp\\
}
%\date{2004�N2��16��}

\begin{document}
\setlength{\baselineskip}{12pt plus .1pt}
\tolerance 10000
\maketitle

\begin{abstract}
\setlength{\baselineskip}{12pt plus .1pt}
�����ichar�j�^�Ɋւ��֐���񋟂��郂�W���[���ł���B
\end{abstract}
%\vspace{-1cm}

% LaTeX 2e Document.
% 
% $Id: Character.tex,v 1.2 2006/01/10 10:46:35 vdmtools Exp $
% 

%%%%%%%%%%%%%%%%%%%%%%%%%%%%%%%%%%%%%%%%
% PDF compatibility code. 
\makeatletter
\newif\ifpdflatex@
\ifx\pdftexversion\@undefined
\pdflatex@false
%\message{Not using pdf}
\else
\pdflatex@true
%\message{Using pdf}
\fi

\newcommand{\latexorpdf}[2]{
  \ifpdflatex@ #2
  \else #1
  \fi
}

\newcommand{\pformat}{a4paper}

\makeatother
%%%%%%%%%%%%%%%%%%%%%%%%%%%%%%%%%%%%%%%%

%\latexorpdf{
%\documentclass[\pformat,12pt]{article}
%}{
%\documentclass[\pformat,pdftex,12pt]{article}
%}
\documentclass[]{jsarticle}

\usepackage[dvips]{color}
\usepackage{array}
\usepackage{longtable}
\usepackage{alltt}
\usepackage{graphics}
\usepackage{vpp}
\usepackage{makeidx}
\makeindex

\definecolor{covered}{rgb}{0,0,0}      %black
%\definecolor{not-covered}{gray}{0.5}   %gray for previewing
%\definecolor{not-covered}{gray}{0.6}   %gray for printing
\definecolor{not-covered}{rgb}{1,0,0}  %red

\newcommand{\InstVarDef}[1]{{\bf #1}}
\newcommand{\TypeDef}[1]{{\bf #1}}
\newcommand{\TypeOcc}[1]{{\it #1}}
\newcommand{\FuncDef}[1]{{\bf #1}}
\newcommand{\FuncOcc}[1]{#1}
\newcommand{\MethodDef}[1]{{\bf #1}}
\newcommand{\MethodOcc}[1]{#1}
\newcommand{\ClassDef}[1]{{\sf #1}}
\newcommand{\ClassOcc}[1]{#1}
\newcommand{\ModDef}[1]{{\sf #1}}
\newcommand{\ModOcc}[1]{#1}

%\nolinenumbering
%\setindent{outer}{\parindent}
%\setindent{inner}{0.0em}

\title{FCharacter���C�u�����[}
\author{
�����L
���{�t�B�b�c�������\\
���Z�p������\\
TEL : 03-3623-4683\\
shin.sahara@jfits.co.jp\\
}
%\date{2004�N2��16��}

\begin{document}
\setlength{\baselineskip}{12pt plus .1pt}
\tolerance 10000
\maketitle

\begin{abstract}
\setlength{\baselineskip}{12pt plus .1pt}
�����ichar�j�^�Ɋւ��֐���񋟂��郂�W���[���ł���B
\end{abstract}
%\vspace{-1cm}

\include{test/Character.vpp}

\include{test/CharT.vpp}

%\newpage
%\addcontentsline{toc}{section}{Index}
%\printindex

\end{document}


\include{test/CharT.vpp}

%\newpage
%\addcontentsline{toc}{section}{Index}
%\printindex

\end{document}


\include{test/CharT.vpp}

%\newpage
%\addcontentsline{toc}{section}{Index}
%\printindex

\end{document}

\include{FCharT.vpp	}
% LaTeX 2e Document.
% 
% $Id: Function.tex,v 1.2 2006/01/10 10:46:35 vdmtools Exp $
% 

%%%%%%%%%%%%%%%%%%%%%%%%%%%%%%%%%%%%%%%%
% PDF compatibility code. 
\makeatletter
\newif\ifpdflatex@
\ifx\pdftexversion\@undefined
\pdflatex@false
%\message{Not using pdf}
\else
\pdflatex@true
%\message{Using pdf}
\fi

\newcommand{\latexorpdf}[2]{
  \ifpdflatex@ #2
  \else #1
  \fi
}

\newcommand{\pformat}{a4paper}

\makeatother
%%%%%%%%%%%%%%%%%%%%%%%%%%%%%%%%%%%%%%%%

%\latexorpdf{
%\documentclass[\pformat,12pt]{article}
%}{
%\documentclass[\pformat,pdftex,12pt]{article}
%}
\documentclass[]{jarticle}

\usepackage[dvips]{color}
\usepackage{array}
\usepackage{longtable}
\usepackage{alltt}
\usepackage{graphics}
\usepackage{vpp}
\usepackage{makeidx}
\makeindex

\definecolor{covered}{rgb}{0,0,0}      %black
%\definecolor{not-covered}{gray}{0.5}   %gray for previewing
%\definecolor{not-covered}{gray}{0.6}   %gray for printing
\definecolor{not-covered}{rgb}{1,0,0}  %red

\newcommand{\InstVarDef}[1]{{\bf #1}}
\newcommand{\TypeDef}[1]{{\bf #1}}
\newcommand{\TypeOcc}[1]{{\it #1}}
\newcommand{\FuncDef}[1]{{\bf #1}}
\newcommand{\FuncOcc}[1]{#1}
\newcommand{\MethodDef}[1]{{\bf #1}}
\newcommand{\MethodOcc}[1]{#1}
\newcommand{\ClassDef}[1]{{\sf #1}}
\newcommand{\ClassOcc}[1]{#1}
\newcommand{\ModDef}[1]{{\sf #1}}
\newcommand{\ModOcc}[1]{#1}

%\nolinenumbering
%\setindent{outer}{\parindent}
%\setindent{inner}{0.0em}

\title{FFunction���C�u�����[}
\author{
�����L
���{�t�B�b�c�������\\
���Z�p������\\
TEL : 03-3623-4683\\
shin.sahara@jfits.co.jp\\
}
%\date{2004�N2��16��}

\begin{document}
\setlength{\baselineskip}{12pt plus .1pt}
\tolerance 10000
\maketitle

\begin{abstract}
\setlength{\baselineskip}{12pt plus .1pt}
�֐��^�Ɋւ��֐���񋟂��郂�W���[���ł���B
\end{abstract}
%\vspace{-1cm}

% LaTeX 2e Document.
% 
% $Id: Function.tex,v 1.2 2006/01/10 10:46:35 vdmtools Exp $
% 

%%%%%%%%%%%%%%%%%%%%%%%%%%%%%%%%%%%%%%%%
% PDF compatibility code. 
\makeatletter
\newif\ifpdflatex@
\ifx\pdftexversion\@undefined
\pdflatex@false
%\message{Not using pdf}
\else
\pdflatex@true
%\message{Using pdf}
\fi

\newcommand{\latexorpdf}[2]{
  \ifpdflatex@ #2
  \else #1
  \fi
}

\newcommand{\pformat}{a4paper}

\makeatother
%%%%%%%%%%%%%%%%%%%%%%%%%%%%%%%%%%%%%%%%

%\latexorpdf{
%\documentclass[\pformat,12pt]{article}
%}{
%\documentclass[\pformat,pdftex,12pt]{article}
%}
\documentclass[]{jarticle}

\usepackage[dvips]{color}
\usepackage{array}
\usepackage{longtable}
\usepackage{alltt}
\usepackage{graphics}
\usepackage{vpp}
\usepackage{makeidx}
\makeindex

\definecolor{covered}{rgb}{0,0,0}      %black
%\definecolor{not-covered}{gray}{0.5}   %gray for previewing
%\definecolor{not-covered}{gray}{0.6}   %gray for printing
\definecolor{not-covered}{rgb}{1,0,0}  %red

\newcommand{\InstVarDef}[1]{{\bf #1}}
\newcommand{\TypeDef}[1]{{\bf #1}}
\newcommand{\TypeOcc}[1]{{\it #1}}
\newcommand{\FuncDef}[1]{{\bf #1}}
\newcommand{\FuncOcc}[1]{#1}
\newcommand{\MethodDef}[1]{{\bf #1}}
\newcommand{\MethodOcc}[1]{#1}
\newcommand{\ClassDef}[1]{{\sf #1}}
\newcommand{\ClassOcc}[1]{#1}
\newcommand{\ModDef}[1]{{\sf #1}}
\newcommand{\ModOcc}[1]{#1}

%\nolinenumbering
%\setindent{outer}{\parindent}
%\setindent{inner}{0.0em}

\title{FFunction���C�u�����[}
\author{
�����L
���{�t�B�b�c�������\\
���Z�p������\\
TEL : 03-3623-4683\\
shin.sahara@jfits.co.jp\\
}
%\date{2004�N2��16��}

\begin{document}
\setlength{\baselineskip}{12pt plus .1pt}
\tolerance 10000
\maketitle

\begin{abstract}
\setlength{\baselineskip}{12pt plus .1pt}
�֐��^�Ɋւ��֐���񋟂��郂�W���[���ł���B
\end{abstract}
%\vspace{-1cm}

% LaTeX 2e Document.
% 
% $Id: Function.tex,v 1.2 2006/01/10 10:46:35 vdmtools Exp $
% 

%%%%%%%%%%%%%%%%%%%%%%%%%%%%%%%%%%%%%%%%
% PDF compatibility code. 
\makeatletter
\newif\ifpdflatex@
\ifx\pdftexversion\@undefined
\pdflatex@false
%\message{Not using pdf}
\else
\pdflatex@true
%\message{Using pdf}
\fi

\newcommand{\latexorpdf}[2]{
  \ifpdflatex@ #2
  \else #1
  \fi
}

\newcommand{\pformat}{a4paper}

\makeatother
%%%%%%%%%%%%%%%%%%%%%%%%%%%%%%%%%%%%%%%%

%\latexorpdf{
%\documentclass[\pformat,12pt]{article}
%}{
%\documentclass[\pformat,pdftex,12pt]{article}
%}
\documentclass[]{jarticle}

\usepackage[dvips]{color}
\usepackage{array}
\usepackage{longtable}
\usepackage{alltt}
\usepackage{graphics}
\usepackage{vpp}
\usepackage{makeidx}
\makeindex

\definecolor{covered}{rgb}{0,0,0}      %black
%\definecolor{not-covered}{gray}{0.5}   %gray for previewing
%\definecolor{not-covered}{gray}{0.6}   %gray for printing
\definecolor{not-covered}{rgb}{1,0,0}  %red

\newcommand{\InstVarDef}[1]{{\bf #1}}
\newcommand{\TypeDef}[1]{{\bf #1}}
\newcommand{\TypeOcc}[1]{{\it #1}}
\newcommand{\FuncDef}[1]{{\bf #1}}
\newcommand{\FuncOcc}[1]{#1}
\newcommand{\MethodDef}[1]{{\bf #1}}
\newcommand{\MethodOcc}[1]{#1}
\newcommand{\ClassDef}[1]{{\sf #1}}
\newcommand{\ClassOcc}[1]{#1}
\newcommand{\ModDef}[1]{{\sf #1}}
\newcommand{\ModOcc}[1]{#1}

%\nolinenumbering
%\setindent{outer}{\parindent}
%\setindent{inner}{0.0em}

\title{FFunction���C�u�����[}
\author{
�����L
���{�t�B�b�c�������\\
���Z�p������\\
TEL : 03-3623-4683\\
shin.sahara@jfits.co.jp\\
}
%\date{2004�N2��16��}

\begin{document}
\setlength{\baselineskip}{12pt plus .1pt}
\tolerance 10000
\maketitle

\begin{abstract}
\setlength{\baselineskip}{12pt plus .1pt}
�֐��^�Ɋւ��֐���񋟂��郂�W���[���ł���B
\end{abstract}
%\vspace{-1cm}

\include{test/Function.vpp}

\include{test/FunctionT.vpp}

%\newpage
%\addcontentsline{toc}{section}{Index}
%\printindex

\end{document}


\include{test/FunctionT.vpp}

%\newpage
%\addcontentsline{toc}{section}{Index}
%\printindex

\end{document}


\include{test/FunctionT.vpp}

%\newpage
%\addcontentsline{toc}{section}{Index}
%\printindex

\end{document}
	
\include{FFunctionT.vpp}
% LaTeX 2e Document.
% 
% $Id: Hashtable.tex,v 1.2 2006/01/10 10:46:35 vdmtools Exp $
% 

%%%%%%%%%%%%%%%%%%%%%%%%%%%%%%%%%%%%%%%%
% PDF compatibility code. 
\makeatletter
\newif\ifpdflatex@
\ifx\pdftexversion\@undefined
\pdflatex@false
%\message{Not using pdf}
\else
\pdflatex@true
%\message{Using pdf}
\fi

\newcommand{\latexorpdf}[2]{
  \ifpdflatex@ #2
  \else #1
  \fi
}

\newcommand{\pformat}{a4paper}

\makeatother
%%%%%%%%%%%%%%%%%%%%%%%%%%%%%%%%%%%%%%%%

%\latexorpdf{
%\documentclass[\pformat,12pt]{article}
%}{
%\documentclass[\pformat,pdftex,12pt]{article}
%}
\documentclass[]{jarticle}

\usepackage[dvips]{color}
\usepackage{array}
\usepackage{longtable}
\usepackage{alltt}
\usepackage{graphics}
\usepackage{vpp}
\usepackage{makeidx}
\makeindex

\definecolor{covered}{rgb}{0,0,0}      %black
%\definecolor{not-covered}{gray}{0.5}   %gray for previewing
%\definecolor{not-covered}{gray}{0.6}   %gray for printing
\definecolor{not-covered}{rgb}{1,0,0}  %red

\newcommand{\InstVarDef}[1]{{\bf #1}}
\newcommand{\TypeDef}[1]{{\bf #1}}
\newcommand{\TypeOcc}[1]{{\it #1}}
\newcommand{\FuncDef}[1]{{\bf #1}}
\newcommand{\FuncOcc}[1]{#1}
\newcommand{\MethodDef}[1]{{\bf #1}}
\newcommand{\MethodOcc}[1]{#1}
\newcommand{\ClassDef}[1]{{\sf #1}}
\newcommand{\ClassOcc}[1]{#1}
\newcommand{\ModDef}[1]{{\sf #1}}
\newcommand{\ModOcc}[1]{#1}

%\nolinenumbering
%\setindent{outer}{\parindent}
%\setindent{inner}{0.0em}

\title{FHashtable���C�u�����[}
\author{
�����L
���{�t�B�b�c�������\\
���Z�p������\\
TEL : 03-3623-4683\\
shin.sahara@jfits.co.jp\\
}
%\date{2004�N2��25��}

\begin{document}
\setlength{\baselineskip}{12pt plus .1pt}
\tolerance 10000
\maketitle

\begin{abstract}
\setlength{\baselineskip}{12pt plus .1pt}
�n�b�V���\�Ɋւ��֐���񋟂��郂�W���[���ł���B
\end{abstract}
%\vspace{-1cm}

% LaTeX 2e Document.
% 
% $Id: Hashtable.tex,v 1.2 2006/01/10 10:46:35 vdmtools Exp $
% 

%%%%%%%%%%%%%%%%%%%%%%%%%%%%%%%%%%%%%%%%
% PDF compatibility code. 
\makeatletter
\newif\ifpdflatex@
\ifx\pdftexversion\@undefined
\pdflatex@false
%\message{Not using pdf}
\else
\pdflatex@true
%\message{Using pdf}
\fi

\newcommand{\latexorpdf}[2]{
  \ifpdflatex@ #2
  \else #1
  \fi
}

\newcommand{\pformat}{a4paper}

\makeatother
%%%%%%%%%%%%%%%%%%%%%%%%%%%%%%%%%%%%%%%%

%\latexorpdf{
%\documentclass[\pformat,12pt]{article}
%}{
%\documentclass[\pformat,pdftex,12pt]{article}
%}
\documentclass[]{jarticle}

\usepackage[dvips]{color}
\usepackage{array}
\usepackage{longtable}
\usepackage{alltt}
\usepackage{graphics}
\usepackage{vpp}
\usepackage{makeidx}
\makeindex

\definecolor{covered}{rgb}{0,0,0}      %black
%\definecolor{not-covered}{gray}{0.5}   %gray for previewing
%\definecolor{not-covered}{gray}{0.6}   %gray for printing
\definecolor{not-covered}{rgb}{1,0,0}  %red

\newcommand{\InstVarDef}[1]{{\bf #1}}
\newcommand{\TypeDef}[1]{{\bf #1}}
\newcommand{\TypeOcc}[1]{{\it #1}}
\newcommand{\FuncDef}[1]{{\bf #1}}
\newcommand{\FuncOcc}[1]{#1}
\newcommand{\MethodDef}[1]{{\bf #1}}
\newcommand{\MethodOcc}[1]{#1}
\newcommand{\ClassDef}[1]{{\sf #1}}
\newcommand{\ClassOcc}[1]{#1}
\newcommand{\ModDef}[1]{{\sf #1}}
\newcommand{\ModOcc}[1]{#1}

%\nolinenumbering
%\setindent{outer}{\parindent}
%\setindent{inner}{0.0em}

\title{FHashtable���C�u�����[}
\author{
�����L
���{�t�B�b�c�������\\
���Z�p������\\
TEL : 03-3623-4683\\
shin.sahara@jfits.co.jp\\
}
%\date{2004�N2��25��}

\begin{document}
\setlength{\baselineskip}{12pt plus .1pt}
\tolerance 10000
\maketitle

\begin{abstract}
\setlength{\baselineskip}{12pt plus .1pt}
�n�b�V���\�Ɋւ��֐���񋟂��郂�W���[���ł���B
\end{abstract}
%\vspace{-1cm}

% LaTeX 2e Document.
% 
% $Id: Hashtable.tex,v 1.2 2006/01/10 10:46:35 vdmtools Exp $
% 

%%%%%%%%%%%%%%%%%%%%%%%%%%%%%%%%%%%%%%%%
% PDF compatibility code. 
\makeatletter
\newif\ifpdflatex@
\ifx\pdftexversion\@undefined
\pdflatex@false
%\message{Not using pdf}
\else
\pdflatex@true
%\message{Using pdf}
\fi

\newcommand{\latexorpdf}[2]{
  \ifpdflatex@ #2
  \else #1
  \fi
}

\newcommand{\pformat}{a4paper}

\makeatother
%%%%%%%%%%%%%%%%%%%%%%%%%%%%%%%%%%%%%%%%

%\latexorpdf{
%\documentclass[\pformat,12pt]{article}
%}{
%\documentclass[\pformat,pdftex,12pt]{article}
%}
\documentclass[]{jarticle}

\usepackage[dvips]{color}
\usepackage{array}
\usepackage{longtable}
\usepackage{alltt}
\usepackage{graphics}
\usepackage{vpp}
\usepackage{makeidx}
\makeindex

\definecolor{covered}{rgb}{0,0,0}      %black
%\definecolor{not-covered}{gray}{0.5}   %gray for previewing
%\definecolor{not-covered}{gray}{0.6}   %gray for printing
\definecolor{not-covered}{rgb}{1,0,0}  %red

\newcommand{\InstVarDef}[1]{{\bf #1}}
\newcommand{\TypeDef}[1]{{\bf #1}}
\newcommand{\TypeOcc}[1]{{\it #1}}
\newcommand{\FuncDef}[1]{{\bf #1}}
\newcommand{\FuncOcc}[1]{#1}
\newcommand{\MethodDef}[1]{{\bf #1}}
\newcommand{\MethodOcc}[1]{#1}
\newcommand{\ClassDef}[1]{{\sf #1}}
\newcommand{\ClassOcc}[1]{#1}
\newcommand{\ModDef}[1]{{\sf #1}}
\newcommand{\ModOcc}[1]{#1}

%\nolinenumbering
%\setindent{outer}{\parindent}
%\setindent{inner}{0.0em}

\title{FHashtable���C�u�����[}
\author{
�����L
���{�t�B�b�c�������\\
���Z�p������\\
TEL : 03-3623-4683\\
shin.sahara@jfits.co.jp\\
}
%\date{2004�N2��25��}

\begin{document}
\setlength{\baselineskip}{12pt plus .1pt}
\tolerance 10000
\maketitle

\begin{abstract}
\setlength{\baselineskip}{12pt plus .1pt}
�n�b�V���\�Ɋւ��֐���񋟂��郂�W���[���ł���B
\end{abstract}
%\vspace{-1cm}

\include{test/Hashtable.vpp}

\include{test/HashtableT.vpp}

%\newpage
%\addcontentsline{toc}{section}{Index}
%\printindex

\end{document}


\include{test/HashtableT.vpp}

%\newpage
%\addcontentsline{toc}{section}{Index}
%\printindex

\end{document}


\include{test/HashtableT.vpp}

%\newpage
%\addcontentsline{toc}{section}{Index}
%\printindex

\end{document}

\include{FHashtableT.vpp}	
% LaTeX 2e Document.
% 
% $Id: Integer.tex,v 1.2 2006/01/10 10:46:35 vdmtools Exp $
% 

%%%%%%%%%%%%%%%%%%%%%%%%%%%%%%%%%%%%%%%%
% PDF compatibility code. 
\makeatletter
\newif\ifpdflatex@
\ifx\pdftexversion\@undefined
\pdflatex@false
%\message{Not using pdf}
\else
\pdflatex@true
%\message{Using pdf}
\fi

\newcommand{\latexorpdf}[2]{
  \ifpdflatex@ #2
  \else #1
  \fi
}

\newcommand{\pformat}{a4paper}

\makeatother
%%%%%%%%%%%%%%%%%%%%%%%%%%%%%%%%%%%%%%%%

%\latexorpdf{
%\documentclass[\pformat,12pt]{article}
%}{
%\documentclass[\pformat,pdftex,12pt]{article}
%}
\documentclass[]{jarticle}

\usepackage[dvips]{color}
\usepackage{array}
\usepackage{longtable}
\usepackage{alltt}
\usepackage{graphics}
\usepackage{vpp}
\usepackage{makeidx}
\makeindex

\definecolor{covered}{rgb}{0,0,0}      %black
%\definecolor{not-covered}{gray}{0.5}   %gray for previewing
%\definecolor{not-covered}{gray}{0.6}   %gray for printing
\definecolor{not-covered}{rgb}{1,0,0}  %red

\newcommand{\InstVarDef}[1]{{\bf #1}}
\newcommand{\TypeDef}[1]{{\bf #1}}
\newcommand{\TypeOcc}[1]{{\it #1}}
\newcommand{\FuncDef}[1]{{\bf #1}}
\newcommand{\FuncOcc}[1]{#1}
\newcommand{\MethodDef}[1]{{\bf #1}}
\newcommand{\MethodOcc}[1]{#1}
\newcommand{\ClassDef}[1]{{\sf #1}}
\newcommand{\ClassOcc}[1]{#1}
\newcommand{\ModDef}[1]{{\sf #1}}
\newcommand{\ModOcc}[1]{#1}

%\nolinenumbering
%\setindent{outer}{\parindent}
%\setindent{inner}{0.0em}

\title{FInteger���C�u�����[}
\author{
�����L
���{�t�B�b�c�������\\
���Z�p������\\
TEL : 03-3623-4683\\
shin.sahara@jfits.co.jp\\
}
%\date{2004�N2��26��}

\begin{document}
\setlength{\baselineskip}{12pt plus .1pt}
\tolerance 10000
\maketitle

\begin{abstract}
\setlength{\baselineskip}{12pt plus .1pt}
�����^�Ɋւ��֐���񋟂��郂�W���[���ł���B
\end{abstract}
%\vspace{-1cm}

% LaTeX 2e Document.
% 
% $Id: Integer.tex,v 1.2 2006/01/10 10:46:35 vdmtools Exp $
% 

%%%%%%%%%%%%%%%%%%%%%%%%%%%%%%%%%%%%%%%%
% PDF compatibility code. 
\makeatletter
\newif\ifpdflatex@
\ifx\pdftexversion\@undefined
\pdflatex@false
%\message{Not using pdf}
\else
\pdflatex@true
%\message{Using pdf}
\fi

\newcommand{\latexorpdf}[2]{
  \ifpdflatex@ #2
  \else #1
  \fi
}

\newcommand{\pformat}{a4paper}

\makeatother
%%%%%%%%%%%%%%%%%%%%%%%%%%%%%%%%%%%%%%%%

%\latexorpdf{
%\documentclass[\pformat,12pt]{article}
%}{
%\documentclass[\pformat,pdftex,12pt]{article}
%}
\documentclass[]{jarticle}

\usepackage[dvips]{color}
\usepackage{array}
\usepackage{longtable}
\usepackage{alltt}
\usepackage{graphics}
\usepackage{vpp}
\usepackage{makeidx}
\makeindex

\definecolor{covered}{rgb}{0,0,0}      %black
%\definecolor{not-covered}{gray}{0.5}   %gray for previewing
%\definecolor{not-covered}{gray}{0.6}   %gray for printing
\definecolor{not-covered}{rgb}{1,0,0}  %red

\newcommand{\InstVarDef}[1]{{\bf #1}}
\newcommand{\TypeDef}[1]{{\bf #1}}
\newcommand{\TypeOcc}[1]{{\it #1}}
\newcommand{\FuncDef}[1]{{\bf #1}}
\newcommand{\FuncOcc}[1]{#1}
\newcommand{\MethodDef}[1]{{\bf #1}}
\newcommand{\MethodOcc}[1]{#1}
\newcommand{\ClassDef}[1]{{\sf #1}}
\newcommand{\ClassOcc}[1]{#1}
\newcommand{\ModDef}[1]{{\sf #1}}
\newcommand{\ModOcc}[1]{#1}

%\nolinenumbering
%\setindent{outer}{\parindent}
%\setindent{inner}{0.0em}

\title{FInteger���C�u�����[}
\author{
�����L
���{�t�B�b�c�������\\
���Z�p������\\
TEL : 03-3623-4683\\
shin.sahara@jfits.co.jp\\
}
%\date{2004�N2��26��}

\begin{document}
\setlength{\baselineskip}{12pt plus .1pt}
\tolerance 10000
\maketitle

\begin{abstract}
\setlength{\baselineskip}{12pt plus .1pt}
�����^�Ɋւ��֐���񋟂��郂�W���[���ł���B
\end{abstract}
%\vspace{-1cm}

% LaTeX 2e Document.
% 
% $Id: Integer.tex,v 1.2 2006/01/10 10:46:35 vdmtools Exp $
% 

%%%%%%%%%%%%%%%%%%%%%%%%%%%%%%%%%%%%%%%%
% PDF compatibility code. 
\makeatletter
\newif\ifpdflatex@
\ifx\pdftexversion\@undefined
\pdflatex@false
%\message{Not using pdf}
\else
\pdflatex@true
%\message{Using pdf}
\fi

\newcommand{\latexorpdf}[2]{
  \ifpdflatex@ #2
  \else #1
  \fi
}

\newcommand{\pformat}{a4paper}

\makeatother
%%%%%%%%%%%%%%%%%%%%%%%%%%%%%%%%%%%%%%%%

%\latexorpdf{
%\documentclass[\pformat,12pt]{article}
%}{
%\documentclass[\pformat,pdftex,12pt]{article}
%}
\documentclass[]{jarticle}

\usepackage[dvips]{color}
\usepackage{array}
\usepackage{longtable}
\usepackage{alltt}
\usepackage{graphics}
\usepackage{vpp}
\usepackage{makeidx}
\makeindex

\definecolor{covered}{rgb}{0,0,0}      %black
%\definecolor{not-covered}{gray}{0.5}   %gray for previewing
%\definecolor{not-covered}{gray}{0.6}   %gray for printing
\definecolor{not-covered}{rgb}{1,0,0}  %red

\newcommand{\InstVarDef}[1]{{\bf #1}}
\newcommand{\TypeDef}[1]{{\bf #1}}
\newcommand{\TypeOcc}[1]{{\it #1}}
\newcommand{\FuncDef}[1]{{\bf #1}}
\newcommand{\FuncOcc}[1]{#1}
\newcommand{\MethodDef}[1]{{\bf #1}}
\newcommand{\MethodOcc}[1]{#1}
\newcommand{\ClassDef}[1]{{\sf #1}}
\newcommand{\ClassOcc}[1]{#1}
\newcommand{\ModDef}[1]{{\sf #1}}
\newcommand{\ModOcc}[1]{#1}

%\nolinenumbering
%\setindent{outer}{\parindent}
%\setindent{inner}{0.0em}

\title{FInteger���C�u�����[}
\author{
�����L
���{�t�B�b�c�������\\
���Z�p������\\
TEL : 03-3623-4683\\
shin.sahara@jfits.co.jp\\
}
%\date{2004�N2��26��}

\begin{document}
\setlength{\baselineskip}{12pt plus .1pt}
\tolerance 10000
\maketitle

\begin{abstract}
\setlength{\baselineskip}{12pt plus .1pt}
�����^�Ɋւ��֐���񋟂��郂�W���[���ł���B
\end{abstract}
%\vspace{-1cm}

\include{test/Integer.vpp}

\include{test/IntegerT.vpp}

%\newpage
%\addcontentsline{toc}{section}{Index}
%\printindex

\end{document}


\include{test/IntegerT.vpp}

%\newpage
%\addcontentsline{toc}{section}{Index}
%\printindex

\end{document}


\include{test/IntegerT.vpp}

%\newpage
%\addcontentsline{toc}{section}{Index}
%\printindex

\end{document}

\include{FIntegerT.vpp}	
% LaTeX 2e Document.
% 
% $Id: JapaneseCalendar.tex,v 1.2 2006/01/10 10:46:35 vdmtools Exp $
% 

%%%%%%%%%%%%%%%%%%%%%%%%%%%%%%%%%%%%%%%%
% PDF compatibility code. 
\makeatletter
\newif\ifpdflatex@
\ifx\pdftexversion\@undefined
\pdflatex@false
%\message{Not using pdf}
\else
\pdflatex@true
%\message{Using pdf}
\fi

\newcommand{\latexorpdf}[2]{
  \ifpdflatex@ #2
  \else #1
  \fi
}

\newcommand{\pformat}{a4paper}

\makeatother
%%%%%%%%%%%%%%%%%%%%%%%%%%%%%%%%%%%%%%%%

%\latexorpdf{
%\documentclass[\pformat,12pt]{article}
%}{
%\documentclass[\pformat,pdftex,12pt]{article}
%}
\documentclass[]{jarticle}

\usepackage[dvips]{color}
\usepackage{array}
\usepackage{longtable}
\usepackage{alltt}
\usepackage{graphics}
\usepackage{vpp}
\usepackage{makeidx}
\makeindex

\definecolor{covered}{rgb}{0,0,0}      %black
%\definecolor{not-covered}{gray}{0.5}   %gray for previewing
%\definecolor{not-covered}{gray}{0.6}   %gray for printing
\definecolor{not-covered}{rgb}{1,0,0}  %red

\newcommand{\InstVarDef}[1]{{\bf #1}}
\newcommand{\TypeDef}[1]{{\bf #1}}
\newcommand{\TypeOcc}[1]{{\it #1}}
\newcommand{\FuncDef}[1]{{\bf #1}}
\newcommand{\FuncOcc}[1]{#1}
\newcommand{\MethodDef}[1]{{\bf #1}}
\newcommand{\MethodOcc}[1]{#1}
\newcommand{\ClassDef}[1]{{\sf #1}}
\newcommand{\ClassOcc}[1]{#1}
\newcommand{\ModDef}[1]{{\sf #1}}
\newcommand{\ModOcc}[1]{#1}

%\nolinenumbering
\linenumbering
\setindent{outer}{\parindent}
%\setindent{inner}{0.0em}

\title{FJapaneseCalendar���C�u�����[}
\author{
�����L
���{�t�B�b�c�������\\
���Z�p������\\
TEL : 03-3623-4683\\
shin.sahara@jfits.co.jp\\
}
%\date{2004�N3��22��}

\begin{document}
\setlength{\baselineskip}{12pt plus .1pt}
\tolerance 10000
\maketitle

\begin{abstract}
\setlength{\baselineskip}{12pt plus .1pt}
���{�̗�i�J�����_�[�j�Ɋւ��֐���񋟂��郂�W���[���ł���B
FCalendar�N���X�̃T�u�N���X�Ƃ��Ď������Ă���B
%\vspace{-1cm}

% LaTeX 2e Document.
% 
% $Id: JapaneseCalendar.tex,v 1.3 2006/04/19 05:06:48 vdmtools Exp $
% 

%%%%%%%%%%%%%%%%%%%%%%%%%%%%%%%%%%%%%%%%
% PDF compatibility code. 
\makeatletter
\newif\ifpdflatex@
\ifx\pdftexversion\@undefined
\pdflatex@false
%\message{Not using pdf}
\else
\pdflatex@true
%\message{Using pdf}
\fi

\newcommand{\latexorpdf}[2]{
  \ifpdflatex@ #2
  \else #1
  \fi
}

\newcommand{\pformat}{a4paper}

\makeatother
%%%%%%%%%%%%%%%%%%%%%%%%%%%%%%%%%%%%%%%%

%\latexorpdf{
%\documentclass[\pformat,12pt]{article}
%}{
%\documentclass[\pformat,pdftex,12pt]{article}
%}
\documentclass[]{jarticle}

\usepackage[dvips]{color}
\usepackage{array}
\usepackage{longtable}
\usepackage{alltt}
\usepackage{graphics}
\usepackage{vpp}
\usepackage{makeidx}
\makeindex

\definecolor{covered}{rgb}{0,0,0}      %black
%\definecolor{not-covered}{gray}{0.5}   %gray for previewing
%\definecolor{not-covered}{gray}{0.6}   %gray for printing
\definecolor{not-covered}{rgb}{1,0,0}  %red

\newcommand{\InstVarDef}[1]{{\bf #1}}
\newcommand{\TypeDef}[1]{{\bf #1}}
\newcommand{\TypeOcc}[1]{{\it #1}}
\newcommand{\FuncDef}[1]{{\bf #1}}
\newcommand{\FuncOcc}[1]{#1}
\newcommand{\MethodDef}[1]{{\bf #1}}
\newcommand{\MethodOcc}[1]{#1}
\newcommand{\ClassDef}[1]{{\sf #1}}
\newcommand{\ClassOcc}[1]{#1}
\newcommand{\ModDef}[1]{{\sf #1}}
\newcommand{\ModOcc}[1]{#1}

%\nolinenumbering
\linenumbering
\setindent{outer}{\parindent}
%\setindent{inner}{0.0em}

\title{FJapaneseCalendar���C�u�����[}
\author{
�����L
���{�t�B�b�c�������\\
���Z�p������\\
TEL : 03-3623-4683\\
shin.sahara@jfits.co.jp\\
}
%\date{2004�N3��22��}

\begin{document}
\setlength{\baselineskip}{12pt plus .1pt}
\tolerance 10000
\maketitle

\begin{abstract}
\setlength{\baselineskip}{12pt plus .1pt}
���{�̗�i�J�����_�[�j�Ɋւ��֐���񋟂��郂�W���[���ł���B
FCalendar�N���X�̃T�u�N���X�Ƃ��Ď������Ă���B
%\vspace{-1cm}

% LaTeX 2e Document.
% 
% $Id: JapaneseCalendar.tex,v 1.3 2006/04/19 05:06:48 vdmtools Exp $
% 

%%%%%%%%%%%%%%%%%%%%%%%%%%%%%%%%%%%%%%%%
% PDF compatibility code. 
\makeatletter
\newif\ifpdflatex@
\ifx\pdftexversion\@undefined
\pdflatex@false
%\message{Not using pdf}
\else
\pdflatex@true
%\message{Using pdf}
\fi

\newcommand{\latexorpdf}[2]{
  \ifpdflatex@ #2
  \else #1
  \fi
}

\newcommand{\pformat}{a4paper}

\makeatother
%%%%%%%%%%%%%%%%%%%%%%%%%%%%%%%%%%%%%%%%

%\latexorpdf{
%\documentclass[\pformat,12pt]{article}
%}{
%\documentclass[\pformat,pdftex,12pt]{article}
%}
\documentclass[]{jarticle}

\usepackage[dvips]{color}
\usepackage{array}
\usepackage{longtable}
\usepackage{alltt}
\usepackage{graphics}
\usepackage{vpp}
\usepackage{makeidx}
\makeindex

\definecolor{covered}{rgb}{0,0,0}      %black
%\definecolor{not-covered}{gray}{0.5}   %gray for previewing
%\definecolor{not-covered}{gray}{0.6}   %gray for printing
\definecolor{not-covered}{rgb}{1,0,0}  %red

\newcommand{\InstVarDef}[1]{{\bf #1}}
\newcommand{\TypeDef}[1]{{\bf #1}}
\newcommand{\TypeOcc}[1]{{\it #1}}
\newcommand{\FuncDef}[1]{{\bf #1}}
\newcommand{\FuncOcc}[1]{#1}
\newcommand{\MethodDef}[1]{{\bf #1}}
\newcommand{\MethodOcc}[1]{#1}
\newcommand{\ClassDef}[1]{{\sf #1}}
\newcommand{\ClassOcc}[1]{#1}
\newcommand{\ModDef}[1]{{\sf #1}}
\newcommand{\ModOcc}[1]{#1}

%\nolinenumbering
\linenumbering
\setindent{outer}{\parindent}
%\setindent{inner}{0.0em}

\title{FJapaneseCalendar���C�u�����[}
\author{
�����L
���{�t�B�b�c�������\\
���Z�p������\\
TEL : 03-3623-4683\\
shin.sahara@jfits.co.jp\\
}
%\date{2004�N3��22��}

\begin{document}
\setlength{\baselineskip}{12pt plus .1pt}
\tolerance 10000
\maketitle

\begin{abstract}
\setlength{\baselineskip}{12pt plus .1pt}
���{�̗�i�J�����_�[�j�Ɋւ��֐���񋟂��郂�W���[���ł���B
FCalendar�N���X�̃T�u�N���X�Ƃ��Ď������Ă���B
%\vspace{-1cm}

\include{test/JapaneseCalendar.vpp}

\include{test/JapaneseCalendarT.vpp}

%\newpage
%\addcontentsline{toc}{section}{Index}
%\printindex

\end{document}


\include{test/JapaneseCalendarT.vpp}

%\newpage
%\addcontentsline{toc}{section}{Index}
%\printindex

\end{document}


\include{test/JapaneseCalendarT.vpp}

%\newpage
%\addcontentsline{toc}{section}{Index}
%\printindex

\end{document}

\include{FJapaneseCalendarT.vpp}
\include{FMap.vpp}
\include{FMapT.vpp}	
% LaTeX 2e Document.
% 
% $Id: Number.tex,v 1.2 2006/01/10 10:46:35 vdmtools Exp $
% 

%%%%%%%%%%%%%%%%%%%%%%%%%%%%%%%%%%%%%%%%
% PDF compatibility code. 
\makeatletter
\newif\ifpdflatex@
\ifx\pdftexversion\@undefined
\pdflatex@false
%\message{Not using pdf}
\else
\pdflatex@true
%\message{Using pdf}
\fi

\newcommand{\latexorpdf}[2]{
  \ifpdflatex@ #2
  \else #1
  \fi
}

\newcommand{\pformat}{a4paper}

\makeatother
%%%%%%%%%%%%%%%%%%%%%%%%%%%%%%%%%%%%%%%%

%\latexorpdf{
%\documentclass[\pformat,12pt]{article}
%}{
%\documentclass[\pformat,pdftex,12pt]{article}
%}
\documentclass[]{jarticle}

\usepackage[dvips]{color}
\usepackage{array}
\usepackage{longtable}
\usepackage{alltt}
\usepackage{graphics}
\usepackage{vpp}
\usepackage{makeidx}
\makeindex

\definecolor{covered}{rgb}{0,0,0}      %black
%\definecolor{not-covered}{gray}{0.5}   %gray for previewing
%\definecolor{not-covered}{gray}{0.6}   %gray for printing
\definecolor{not-covered}{rgb}{1,0,0}  %red

\newcommand{\InstVarDef}[1]{{\bf #1}}
\newcommand{\TypeDef}[1]{{\bf #1}}
\newcommand{\TypeOcc}[1]{{\it #1}}
\newcommand{\FuncDef}[1]{{\bf #1}}
\newcommand{\FuncOcc}[1]{#1}
\newcommand{\MethodDef}[1]{{\bf #1}}
\newcommand{\MethodOcc}[1]{#1}
\newcommand{\ClassDef}[1]{{\sf #1}}
\newcommand{\ClassOcc}[1]{#1}
\newcommand{\ModDef}[1]{{\sf #1}}
\newcommand{\ModOcc}[1]{#1}

%\nolinenumbering
%\setindent{outer}{\parindent}
%\setindent{inner}{0.0em}

\title{FNumber���C�u�����[}
\author{
�����L
���{�t�B�b�c�������\\
���Z�p������\\
TEL : 03-3623-4683\\
shin.sahara@jfits.co.jp\\
}
%\date{2004�N2��25��}

\begin{document}
\setlength{\baselineskip}{12pt plus .1pt}
\tolerance 10000
\maketitle

\begin{abstract}
\setlength{\baselineskip}{12pt plus .1pt}
����������ɋ��ʂȁA���Ɋւ��֐���񋟂��郂�W���[���ł���B
\end{abstract}
%\vspace{-1cm}

% LaTeX 2e Document.
% 
% $Id: Number.tex,v 1.2 2006/01/10 10:46:35 vdmtools Exp $
% 

%%%%%%%%%%%%%%%%%%%%%%%%%%%%%%%%%%%%%%%%
% PDF compatibility code. 
\makeatletter
\newif\ifpdflatex@
\ifx\pdftexversion\@undefined
\pdflatex@false
%\message{Not using pdf}
\else
\pdflatex@true
%\message{Using pdf}
\fi

\newcommand{\latexorpdf}[2]{
  \ifpdflatex@ #2
  \else #1
  \fi
}

\newcommand{\pformat}{a4paper}

\makeatother
%%%%%%%%%%%%%%%%%%%%%%%%%%%%%%%%%%%%%%%%

%\latexorpdf{
%\documentclass[\pformat,12pt]{article}
%}{
%\documentclass[\pformat,pdftex,12pt]{article}
%}
\documentclass[]{jarticle}

\usepackage[dvips]{color}
\usepackage{array}
\usepackage{longtable}
\usepackage{alltt}
\usepackage{graphics}
\usepackage{vpp}
\usepackage{makeidx}
\makeindex

\definecolor{covered}{rgb}{0,0,0}      %black
%\definecolor{not-covered}{gray}{0.5}   %gray for previewing
%\definecolor{not-covered}{gray}{0.6}   %gray for printing
\definecolor{not-covered}{rgb}{1,0,0}  %red

\newcommand{\InstVarDef}[1]{{\bf #1}}
\newcommand{\TypeDef}[1]{{\bf #1}}
\newcommand{\TypeOcc}[1]{{\it #1}}
\newcommand{\FuncDef}[1]{{\bf #1}}
\newcommand{\FuncOcc}[1]{#1}
\newcommand{\MethodDef}[1]{{\bf #1}}
\newcommand{\MethodOcc}[1]{#1}
\newcommand{\ClassDef}[1]{{\sf #1}}
\newcommand{\ClassOcc}[1]{#1}
\newcommand{\ModDef}[1]{{\sf #1}}
\newcommand{\ModOcc}[1]{#1}

%\nolinenumbering
%\setindent{outer}{\parindent}
%\setindent{inner}{0.0em}

\title{FNumber���C�u�����[}
\author{
�����L
���{�t�B�b�c�������\\
���Z�p������\\
TEL : 03-3623-4683\\
shin.sahara@jfits.co.jp\\
}
%\date{2004�N2��25��}

\begin{document}
\setlength{\baselineskip}{12pt plus .1pt}
\tolerance 10000
\maketitle

\begin{abstract}
\setlength{\baselineskip}{12pt plus .1pt}
����������ɋ��ʂȁA���Ɋւ��֐���񋟂��郂�W���[���ł���B
\end{abstract}
%\vspace{-1cm}

% LaTeX 2e Document.
% 
% $Id: Number.tex,v 1.2 2006/01/10 10:46:35 vdmtools Exp $
% 

%%%%%%%%%%%%%%%%%%%%%%%%%%%%%%%%%%%%%%%%
% PDF compatibility code. 
\makeatletter
\newif\ifpdflatex@
\ifx\pdftexversion\@undefined
\pdflatex@false
%\message{Not using pdf}
\else
\pdflatex@true
%\message{Using pdf}
\fi

\newcommand{\latexorpdf}[2]{
  \ifpdflatex@ #2
  \else #1
  \fi
}

\newcommand{\pformat}{a4paper}

\makeatother
%%%%%%%%%%%%%%%%%%%%%%%%%%%%%%%%%%%%%%%%

%\latexorpdf{
%\documentclass[\pformat,12pt]{article}
%}{
%\documentclass[\pformat,pdftex,12pt]{article}
%}
\documentclass[]{jarticle}

\usepackage[dvips]{color}
\usepackage{array}
\usepackage{longtable}
\usepackage{alltt}
\usepackage{graphics}
\usepackage{vpp}
\usepackage{makeidx}
\makeindex

\definecolor{covered}{rgb}{0,0,0}      %black
%\definecolor{not-covered}{gray}{0.5}   %gray for previewing
%\definecolor{not-covered}{gray}{0.6}   %gray for printing
\definecolor{not-covered}{rgb}{1,0,0}  %red

\newcommand{\InstVarDef}[1]{{\bf #1}}
\newcommand{\TypeDef}[1]{{\bf #1}}
\newcommand{\TypeOcc}[1]{{\it #1}}
\newcommand{\FuncDef}[1]{{\bf #1}}
\newcommand{\FuncOcc}[1]{#1}
\newcommand{\MethodDef}[1]{{\bf #1}}
\newcommand{\MethodOcc}[1]{#1}
\newcommand{\ClassDef}[1]{{\sf #1}}
\newcommand{\ClassOcc}[1]{#1}
\newcommand{\ModDef}[1]{{\sf #1}}
\newcommand{\ModOcc}[1]{#1}

%\nolinenumbering
%\setindent{outer}{\parindent}
%\setindent{inner}{0.0em}

\title{FNumber���C�u�����[}
\author{
�����L
���{�t�B�b�c�������\\
���Z�p������\\
TEL : 03-3623-4683\\
shin.sahara@jfits.co.jp\\
}
%\date{2004�N2��25��}

\begin{document}
\setlength{\baselineskip}{12pt plus .1pt}
\tolerance 10000
\maketitle

\begin{abstract}
\setlength{\baselineskip}{12pt plus .1pt}
����������ɋ��ʂȁA���Ɋւ��֐���񋟂��郂�W���[���ł���B
\end{abstract}
%\vspace{-1cm}

\include{test/Number.vpp}

\include{test/NumberT.vpp}

%\newpage
%\addcontentsline{toc}{section}{Index}
%\printindex

\end{document}


\include{test/NumberT.vpp}

%\newpage
%\addcontentsline{toc}{section}{Index}
%\printindex

\end{document}


\include{test/NumberT.vpp}

%\newpage
%\addcontentsline{toc}{section}{Index}
%\printindex

\end{document}

\include{FNumberT.vpp}	
%% LaTeX 2e Document.
% 
% $Id: Queue.tex,v 1.2 2006/01/10 10:46:35 vdmtools Exp $
% 

%%%%%%%%%%%%%%%%%%%%%%%%%%%%%%%%%%%%%%%%
% PDF compatibility code. 
\makeatletter
\newif\ifpdflatex@
\ifx\pdftexversion\@undefined
\pdflatex@false
%\message{Not using pdf}
\else
\pdflatex@true
%\message{Using pdf}
\fi

\newcommand{\latexorpdf}[2]{
  \ifpdflatex@ #2
  \else #1
  \fi
}

\newcommand{\pformat}{a4paper}

\makeatother
%%%%%%%%%%%%%%%%%%%%%%%%%%%%%%%%%%%%%%%%

%\latexorpdf{
%\documentclass[\pformat,12pt]{article}
%}{
%\documentclass[\pformat,pdftex,12pt]{article}
%}
\documentclass[]{jarticle}

\usepackage[dvips]{color}
\usepackage{array}
\usepackage{longtable}
\usepackage{alltt}
\usepackage{graphics}
\usepackage{vpp}
\usepackage{makeidx}
\makeindex

\definecolor{covered}{rgb}{0,0,0}      %black
%\definecolor{not-covered}{gray}{0.5}   %gray for previewing
%\definecolor{not-covered}{gray}{0.6}   %gray for printing
\definecolor{not-covered}{rgb}{1,0,0}  %red

\newcommand{\InstVarDef}[1]{{\bf #1}}
\newcommand{\TypeDef}[1]{{\bf #1}}
\newcommand{\TypeOcc}[1]{{\it #1}}
\newcommand{\FuncDef}[1]{{\bf #1}}
\newcommand{\FuncOcc}[1]{#1}
\newcommand{\MethodDef}[1]{{\bf #1}}
\newcommand{\MethodOcc}[1]{#1}
\newcommand{\ClassDef}[1]{{\sf #1}}
\newcommand{\ClassOcc}[1]{#1}
\newcommand{\ModDef}[1]{{\sf #1}}
\newcommand{\ModOcc}[1]{#1}

%\nolinenumbering
%\setindent{outer}{\parindent}
%\setindent{inner}{0.0em}

\title{FQueue���C�u�����[}
\author{
�����L
���{�t�B�b�c�������\\
���Z�p������\\
TEL : 03-3623-4683\\
shin.sahara@jfits.co.jp\\
}
%\date{2004�N2��25��}

\begin{document}
\setlength{\baselineskip}{12pt plus .1pt}
\tolerance 10000
\maketitle

\begin{abstract}
\setlength{\baselineskip}{12pt plus .1pt}
�҂��s��Ɋւ��֐���񋟂��郂�W���[���ł���B
\end{abstract}
%\vspace{-1cm}

% LaTeX 2e Document.
% 
% $Id: Queue.tex,v 1.2 2006/01/10 10:46:35 vdmtools Exp $
% 

%%%%%%%%%%%%%%%%%%%%%%%%%%%%%%%%%%%%%%%%
% PDF compatibility code. 
\makeatletter
\newif\ifpdflatex@
\ifx\pdftexversion\@undefined
\pdflatex@false
%\message{Not using pdf}
\else
\pdflatex@true
%\message{Using pdf}
\fi

\newcommand{\latexorpdf}[2]{
  \ifpdflatex@ #2
  \else #1
  \fi
}

\newcommand{\pformat}{a4paper}

\makeatother
%%%%%%%%%%%%%%%%%%%%%%%%%%%%%%%%%%%%%%%%

%\latexorpdf{
%\documentclass[\pformat,12pt]{article}
%}{
%\documentclass[\pformat,pdftex,12pt]{article}
%}
\documentclass[]{jarticle}

\usepackage[dvips]{color}
\usepackage{array}
\usepackage{longtable}
\usepackage{alltt}
\usepackage{graphics}
\usepackage{vpp}
\usepackage{makeidx}
\makeindex

\definecolor{covered}{rgb}{0,0,0}      %black
%\definecolor{not-covered}{gray}{0.5}   %gray for previewing
%\definecolor{not-covered}{gray}{0.6}   %gray for printing
\definecolor{not-covered}{rgb}{1,0,0}  %red

\newcommand{\InstVarDef}[1]{{\bf #1}}
\newcommand{\TypeDef}[1]{{\bf #1}}
\newcommand{\TypeOcc}[1]{{\it #1}}
\newcommand{\FuncDef}[1]{{\bf #1}}
\newcommand{\FuncOcc}[1]{#1}
\newcommand{\MethodDef}[1]{{\bf #1}}
\newcommand{\MethodOcc}[1]{#1}
\newcommand{\ClassDef}[1]{{\sf #1}}
\newcommand{\ClassOcc}[1]{#1}
\newcommand{\ModDef}[1]{{\sf #1}}
\newcommand{\ModOcc}[1]{#1}

%\nolinenumbering
%\setindent{outer}{\parindent}
%\setindent{inner}{0.0em}

\title{FQueue���C�u�����[}
\author{
�����L
���{�t�B�b�c�������\\
���Z�p������\\
TEL : 03-3623-4683\\
shin.sahara@jfits.co.jp\\
}
%\date{2004�N2��25��}

\begin{document}
\setlength{\baselineskip}{12pt plus .1pt}
\tolerance 10000
\maketitle

\begin{abstract}
\setlength{\baselineskip}{12pt plus .1pt}
�҂��s��Ɋւ��֐���񋟂��郂�W���[���ł���B
\end{abstract}
%\vspace{-1cm}

% LaTeX 2e Document.
% 
% $Id: Queue.tex,v 1.2 2006/01/10 10:46:35 vdmtools Exp $
% 

%%%%%%%%%%%%%%%%%%%%%%%%%%%%%%%%%%%%%%%%
% PDF compatibility code. 
\makeatletter
\newif\ifpdflatex@
\ifx\pdftexversion\@undefined
\pdflatex@false
%\message{Not using pdf}
\else
\pdflatex@true
%\message{Using pdf}
\fi

\newcommand{\latexorpdf}[2]{
  \ifpdflatex@ #2
  \else #1
  \fi
}

\newcommand{\pformat}{a4paper}

\makeatother
%%%%%%%%%%%%%%%%%%%%%%%%%%%%%%%%%%%%%%%%

%\latexorpdf{
%\documentclass[\pformat,12pt]{article}
%}{
%\documentclass[\pformat,pdftex,12pt]{article}
%}
\documentclass[]{jarticle}

\usepackage[dvips]{color}
\usepackage{array}
\usepackage{longtable}
\usepackage{alltt}
\usepackage{graphics}
\usepackage{vpp}
\usepackage{makeidx}
\makeindex

\definecolor{covered}{rgb}{0,0,0}      %black
%\definecolor{not-covered}{gray}{0.5}   %gray for previewing
%\definecolor{not-covered}{gray}{0.6}   %gray for printing
\definecolor{not-covered}{rgb}{1,0,0}  %red

\newcommand{\InstVarDef}[1]{{\bf #1}}
\newcommand{\TypeDef}[1]{{\bf #1}}
\newcommand{\TypeOcc}[1]{{\it #1}}
\newcommand{\FuncDef}[1]{{\bf #1}}
\newcommand{\FuncOcc}[1]{#1}
\newcommand{\MethodDef}[1]{{\bf #1}}
\newcommand{\MethodOcc}[1]{#1}
\newcommand{\ClassDef}[1]{{\sf #1}}
\newcommand{\ClassOcc}[1]{#1}
\newcommand{\ModDef}[1]{{\sf #1}}
\newcommand{\ModOcc}[1]{#1}

%\nolinenumbering
%\setindent{outer}{\parindent}
%\setindent{inner}{0.0em}

\title{FQueue���C�u�����[}
\author{
�����L
���{�t�B�b�c�������\\
���Z�p������\\
TEL : 03-3623-4683\\
shin.sahara@jfits.co.jp\\
}
%\date{2004�N2��25��}

\begin{document}
\setlength{\baselineskip}{12pt plus .1pt}
\tolerance 10000
\maketitle

\begin{abstract}
\setlength{\baselineskip}{12pt plus .1pt}
�҂��s��Ɋւ��֐���񋟂��郂�W���[���ł���B
\end{abstract}
%\vspace{-1cm}

\include{test/Queue.vpp}

\include{test/QueueT.vpp}

%\newpage
%\addcontentsline{toc}{section}{Index}
%\printindex

\end{document}


\include{test/QueueT.vpp}

%\newpage
%\addcontentsline{toc}{section}{Index}
%\printindex

\end{document}


\include{test/QueueT.vpp}

%\newpage
%\addcontentsline{toc}{section}{Index}
%\printindex

\end{document}

\include{FQueueT.vpp}
% LaTeX 2e Document.
% 
% $Id: Product.tex,v 1.2 2006/01/10 10:46:35 vdmtools Exp $
% 

%%%%%%%%%%%%%%%%%%%%%%%%%%%%%%%%%%%%%%%%
% PDF compatibility code. 
\makeatletter
\newif\ifpdflatex@
\ifx\pdftexversion\@undefined
\pdflatex@false
%\message{Not using pdf}
\else
\pdflatex@true
%\message{Using pdf}
\fi

\newcommand{\latexorpdf}[2]{
  \ifpdflatex@ #2
  \else #1
  \fi
}

\newcommand{\pformat}{a4paper}

\makeatother
%%%%%%%%%%%%%%%%%%%%%%%%%%%%%%%%%%%%%%%%

%\latexorpdf{
%\documentclass[\pformat,12pt]{article}
%}{
%\documentclass[\pformat,pdftex,12pt]{article}
%}
\documentclass[]{jarticle}

\usepackage[dvips]{color}
\usepackage{array}
\usepackage{longtable}
\usepackage{alltt}
\usepackage{graphics}
\usepackage{vpp}
\usepackage{makeidx}
\makeindex

\definecolor{covered}{rgb}{0,0,0}      %black
%\definecolor{not-covered}{gray}{0.5}   %gray for previewing
%\definecolor{not-covered}{gray}{0.6}   %gray for printing
\definecolor{not-covered}{rgb}{1,0,0}  %red

\newcommand{\InstVarDef}[1]{{\bf #1}}
\newcommand{\TypeDef}[1]{{\bf #1}}
\newcommand{\TypeOcc}[1]{{\it #1}}
\newcommand{\FuncDef}[1]{{\bf #1}}
\newcommand{\FuncOcc}[1]{#1}
\newcommand{\MethodDef}[1]{{\bf #1}}
\newcommand{\MethodOcc}[1]{#1}
\newcommand{\ClassDef}[1]{{\sf #1}}
\newcommand{\ClassOcc}[1]{#1}
\newcommand{\ModDef}[1]{{\sf #1}}
\newcommand{\ModOcc}[1]{#1}

%\nolinenumbering
%\setindent{outer}{\parindent}
%\setindent{inner}{0.0em}

\title{FProduct���C�u�����[}
\author{
�����L
���{�t�B�b�c�������\\
���Z�p������\\
TEL : 03-3623-4683\\
shin.sahara@jfits.co.jp\\
}
%\date{2004�N2��25��}

\begin{document}
\setlength{\baselineskip}{12pt plus .1pt}
\tolerance 10000
\maketitle

\begin{abstract}
\setlength{\baselineskip}{12pt plus .1pt}
�g�Ɋւ��֐���񋟂��郂�W���[���ł���B
\end{abstract}
%\vspace{-1cm}

% LaTeX 2e Document.
% 
% $Id: Product.tex,v 1.2 2006/01/10 10:46:35 vdmtools Exp $
% 

%%%%%%%%%%%%%%%%%%%%%%%%%%%%%%%%%%%%%%%%
% PDF compatibility code. 
\makeatletter
\newif\ifpdflatex@
\ifx\pdftexversion\@undefined
\pdflatex@false
%\message{Not using pdf}
\else
\pdflatex@true
%\message{Using pdf}
\fi

\newcommand{\latexorpdf}[2]{
  \ifpdflatex@ #2
  \else #1
  \fi
}

\newcommand{\pformat}{a4paper}

\makeatother
%%%%%%%%%%%%%%%%%%%%%%%%%%%%%%%%%%%%%%%%

%\latexorpdf{
%\documentclass[\pformat,12pt]{article}
%}{
%\documentclass[\pformat,pdftex,12pt]{article}
%}
\documentclass[]{jarticle}

\usepackage[dvips]{color}
\usepackage{array}
\usepackage{longtable}
\usepackage{alltt}
\usepackage{graphics}
\usepackage{vpp}
\usepackage{makeidx}
\makeindex

\definecolor{covered}{rgb}{0,0,0}      %black
%\definecolor{not-covered}{gray}{0.5}   %gray for previewing
%\definecolor{not-covered}{gray}{0.6}   %gray for printing
\definecolor{not-covered}{rgb}{1,0,0}  %red

\newcommand{\InstVarDef}[1]{{\bf #1}}
\newcommand{\TypeDef}[1]{{\bf #1}}
\newcommand{\TypeOcc}[1]{{\it #1}}
\newcommand{\FuncDef}[1]{{\bf #1}}
\newcommand{\FuncOcc}[1]{#1}
\newcommand{\MethodDef}[1]{{\bf #1}}
\newcommand{\MethodOcc}[1]{#1}
\newcommand{\ClassDef}[1]{{\sf #1}}
\newcommand{\ClassOcc}[1]{#1}
\newcommand{\ModDef}[1]{{\sf #1}}
\newcommand{\ModOcc}[1]{#1}

%\nolinenumbering
%\setindent{outer}{\parindent}
%\setindent{inner}{0.0em}

\title{FProduct���C�u�����[}
\author{
�����L
���{�t�B�b�c�������\\
���Z�p������\\
TEL : 03-3623-4683\\
shin.sahara@jfits.co.jp\\
}
%\date{2004�N2��25��}

\begin{document}
\setlength{\baselineskip}{12pt plus .1pt}
\tolerance 10000
\maketitle

\begin{abstract}
\setlength{\baselineskip}{12pt plus .1pt}
�g�Ɋւ��֐���񋟂��郂�W���[���ł���B
\end{abstract}
%\vspace{-1cm}

% LaTeX 2e Document.
% 
% $Id: Product.tex,v 1.2 2006/01/10 10:46:35 vdmtools Exp $
% 

%%%%%%%%%%%%%%%%%%%%%%%%%%%%%%%%%%%%%%%%
% PDF compatibility code. 
\makeatletter
\newif\ifpdflatex@
\ifx\pdftexversion\@undefined
\pdflatex@false
%\message{Not using pdf}
\else
\pdflatex@true
%\message{Using pdf}
\fi

\newcommand{\latexorpdf}[2]{
  \ifpdflatex@ #2
  \else #1
  \fi
}

\newcommand{\pformat}{a4paper}

\makeatother
%%%%%%%%%%%%%%%%%%%%%%%%%%%%%%%%%%%%%%%%

%\latexorpdf{
%\documentclass[\pformat,12pt]{article}
%}{
%\documentclass[\pformat,pdftex,12pt]{article}
%}
\documentclass[]{jarticle}

\usepackage[dvips]{color}
\usepackage{array}
\usepackage{longtable}
\usepackage{alltt}
\usepackage{graphics}
\usepackage{vpp}
\usepackage{makeidx}
\makeindex

\definecolor{covered}{rgb}{0,0,0}      %black
%\definecolor{not-covered}{gray}{0.5}   %gray for previewing
%\definecolor{not-covered}{gray}{0.6}   %gray for printing
\definecolor{not-covered}{rgb}{1,0,0}  %red

\newcommand{\InstVarDef}[1]{{\bf #1}}
\newcommand{\TypeDef}[1]{{\bf #1}}
\newcommand{\TypeOcc}[1]{{\it #1}}
\newcommand{\FuncDef}[1]{{\bf #1}}
\newcommand{\FuncOcc}[1]{#1}
\newcommand{\MethodDef}[1]{{\bf #1}}
\newcommand{\MethodOcc}[1]{#1}
\newcommand{\ClassDef}[1]{{\sf #1}}
\newcommand{\ClassOcc}[1]{#1}
\newcommand{\ModDef}[1]{{\sf #1}}
\newcommand{\ModOcc}[1]{#1}

%\nolinenumbering
%\setindent{outer}{\parindent}
%\setindent{inner}{0.0em}

\title{FProduct���C�u�����[}
\author{
�����L
���{�t�B�b�c�������\\
���Z�p������\\
TEL : 03-3623-4683\\
shin.sahara@jfits.co.jp\\
}
%\date{2004�N2��25��}

\begin{document}
\setlength{\baselineskip}{12pt plus .1pt}
\tolerance 10000
\maketitle

\begin{abstract}
\setlength{\baselineskip}{12pt plus .1pt}
�g�Ɋւ��֐���񋟂��郂�W���[���ł���B
\end{abstract}
%\vspace{-1cm}

\include{test/Product.vpp}

\include{test/ProductT.vpp}

%\newpage
%\addcontentsline{toc}{section}{Index}
%\printindex

\end{document}


\include{test/ProductT.vpp}

%\newpage
%\addcontentsline{toc}{section}{Index}
%\printindex

\end{document}


\include{test/ProductT.vpp}

%\newpage
%\addcontentsline{toc}{section}{Index}
%\printindex

\end{document}
	
\include{FProductT.vpp}	
% LaTeX 2e Document.
% 
% $Id: Real.tex,v 1.2 2006/01/10 10:46:35 vdmtools Exp $
% 

%%%%%%%%%%%%%%%%%%%%%%%%%%%%%%%%%%%%%%%%
% PDF compatibility code. 
\makeatletter
\newif\ifpdflatex@
\ifx\pdftexversion\@undefined
\pdflatex@false
%\message{Not using pdf}
\else
\pdflatex@true
%\message{Using pdf}
\fi

\newcommand{\latexorpdf}[2]{
  \ifpdflatex@ #2
  \else #1
  \fi
}

\newcommand{\pformat}{a4paper}

\makeatother
%%%%%%%%%%%%%%%%%%%%%%%%%%%%%%%%%%%%%%%%

%\latexorpdf{
%\documentclass[\pformat,12pt]{article}
%}{
%\documentclass[\pformat,pdftex,12pt]{article}
%}
\documentclass[]{jarticle}

\usepackage[dvips]{color}
\usepackage{array}
\usepackage{longtable}
\usepackage{alltt}
\usepackage{graphics}
\usepackage{vpp}
\usepackage{makeidx}
\makeindex

\definecolor{covered}{rgb}{0,0,0}      %black
%\definecolor{not-covered}{gray}{0.5}   %gray for previewing
%\definecolor{not-covered}{gray}{0.6}   %gray for printing
\definecolor{not-covered}{rgb}{1,0,0}  %red

\newcommand{\InstVarDef}[1]{{\bf #1}}
\newcommand{\TypeDef}[1]{{\bf #1}}
\newcommand{\TypeOcc}[1]{{\it #1}}
\newcommand{\FuncDef}[1]{{\bf #1}}
\newcommand{\FuncOcc}[1]{#1}
\newcommand{\MethodDef}[1]{{\bf #1}}
\newcommand{\MethodOcc}[1]{#1}
\newcommand{\ClassDef}[1]{{\sf #1}}
\newcommand{\ClassOcc}[1]{#1}
\newcommand{\ModDef}[1]{{\sf #1}}
\newcommand{\ModOcc}[1]{#1}

%\nolinenumbering
%\setindent{outer}{\parindent}
%\setindent{inner}{0.0em}

\title{FReal���C�u�����[}
\author{
�����L
���{�t�B�b�c�������\\
���Z�p������\\
TEL : 03-3623-4683\\
shin.sahara@jfits.co.jp\\
}
%\date{2004�N2��16��}

\begin{document}
\setlength{\baselineskip}{12pt plus .1pt}
\tolerance 10000
\maketitle

\begin{abstract}
\setlength{\baselineskip}{12pt plus .1pt}
�����^�Ɋւ��֐���񋟂��郂�W���[���ł���B
\end{abstract}
%\vspace{-1cm}

% LaTeX 2e Document.
% 
% $Id: Real.tex,v 1.2 2006/01/10 10:46:35 vdmtools Exp $
% 

%%%%%%%%%%%%%%%%%%%%%%%%%%%%%%%%%%%%%%%%
% PDF compatibility code. 
\makeatletter
\newif\ifpdflatex@
\ifx\pdftexversion\@undefined
\pdflatex@false
%\message{Not using pdf}
\else
\pdflatex@true
%\message{Using pdf}
\fi

\newcommand{\latexorpdf}[2]{
  \ifpdflatex@ #2
  \else #1
  \fi
}

\newcommand{\pformat}{a4paper}

\makeatother
%%%%%%%%%%%%%%%%%%%%%%%%%%%%%%%%%%%%%%%%

%\latexorpdf{
%\documentclass[\pformat,12pt]{article}
%}{
%\documentclass[\pformat,pdftex,12pt]{article}
%}
\documentclass[]{jarticle}

\usepackage[dvips]{color}
\usepackage{array}
\usepackage{longtable}
\usepackage{alltt}
\usepackage{graphics}
\usepackage{vpp}
\usepackage{makeidx}
\makeindex

\definecolor{covered}{rgb}{0,0,0}      %black
%\definecolor{not-covered}{gray}{0.5}   %gray for previewing
%\definecolor{not-covered}{gray}{0.6}   %gray for printing
\definecolor{not-covered}{rgb}{1,0,0}  %red

\newcommand{\InstVarDef}[1]{{\bf #1}}
\newcommand{\TypeDef}[1]{{\bf #1}}
\newcommand{\TypeOcc}[1]{{\it #1}}
\newcommand{\FuncDef}[1]{{\bf #1}}
\newcommand{\FuncOcc}[1]{#1}
\newcommand{\MethodDef}[1]{{\bf #1}}
\newcommand{\MethodOcc}[1]{#1}
\newcommand{\ClassDef}[1]{{\sf #1}}
\newcommand{\ClassOcc}[1]{#1}
\newcommand{\ModDef}[1]{{\sf #1}}
\newcommand{\ModOcc}[1]{#1}

%\nolinenumbering
%\setindent{outer}{\parindent}
%\setindent{inner}{0.0em}

\title{FReal���C�u�����[}
\author{
�����L
���{�t�B�b�c�������\\
���Z�p������\\
TEL : 03-3623-4683\\
shin.sahara@jfits.co.jp\\
}
%\date{2004�N2��16��}

\begin{document}
\setlength{\baselineskip}{12pt plus .1pt}
\tolerance 10000
\maketitle

\begin{abstract}
\setlength{\baselineskip}{12pt plus .1pt}
�����^�Ɋւ��֐���񋟂��郂�W���[���ł���B
\end{abstract}
%\vspace{-1cm}

% LaTeX 2e Document.
% 
% $Id: Real.tex,v 1.2 2006/01/10 10:46:35 vdmtools Exp $
% 

%%%%%%%%%%%%%%%%%%%%%%%%%%%%%%%%%%%%%%%%
% PDF compatibility code. 
\makeatletter
\newif\ifpdflatex@
\ifx\pdftexversion\@undefined
\pdflatex@false
%\message{Not using pdf}
\else
\pdflatex@true
%\message{Using pdf}
\fi

\newcommand{\latexorpdf}[2]{
  \ifpdflatex@ #2
  \else #1
  \fi
}

\newcommand{\pformat}{a4paper}

\makeatother
%%%%%%%%%%%%%%%%%%%%%%%%%%%%%%%%%%%%%%%%

%\latexorpdf{
%\documentclass[\pformat,12pt]{article}
%}{
%\documentclass[\pformat,pdftex,12pt]{article}
%}
\documentclass[]{jarticle}

\usepackage[dvips]{color}
\usepackage{array}
\usepackage{longtable}
\usepackage{alltt}
\usepackage{graphics}
\usepackage{vpp}
\usepackage{makeidx}
\makeindex

\definecolor{covered}{rgb}{0,0,0}      %black
%\definecolor{not-covered}{gray}{0.5}   %gray for previewing
%\definecolor{not-covered}{gray}{0.6}   %gray for printing
\definecolor{not-covered}{rgb}{1,0,0}  %red

\newcommand{\InstVarDef}[1]{{\bf #1}}
\newcommand{\TypeDef}[1]{{\bf #1}}
\newcommand{\TypeOcc}[1]{{\it #1}}
\newcommand{\FuncDef}[1]{{\bf #1}}
\newcommand{\FuncOcc}[1]{#1}
\newcommand{\MethodDef}[1]{{\bf #1}}
\newcommand{\MethodOcc}[1]{#1}
\newcommand{\ClassDef}[1]{{\sf #1}}
\newcommand{\ClassOcc}[1]{#1}
\newcommand{\ModDef}[1]{{\sf #1}}
\newcommand{\ModOcc}[1]{#1}

%\nolinenumbering
%\setindent{outer}{\parindent}
%\setindent{inner}{0.0em}

\title{FReal���C�u�����[}
\author{
�����L
���{�t�B�b�c�������\\
���Z�p������\\
TEL : 03-3623-4683\\
shin.sahara@jfits.co.jp\\
}
%\date{2004�N2��16��}

\begin{document}
\setlength{\baselineskip}{12pt plus .1pt}
\tolerance 10000
\maketitle

\begin{abstract}
\setlength{\baselineskip}{12pt plus .1pt}
�����^�Ɋւ��֐���񋟂��郂�W���[���ł���B
\end{abstract}
%\vspace{-1cm}

\include{test/Real.vpp}

\include{test/RealT.vpp}

%\newpage
%\addcontentsline{toc}{section}{Index}
%\printindex

\end{document}


\include{test/RealT.vpp}

%\newpage
%\addcontentsline{toc}{section}{Index}
%\printindex

\end{document}


\include{test/RealT.vpp}

%\newpage
%\addcontentsline{toc}{section}{Index}
%\printindex

\end{document}

\include{FRealT.vpp}
% LaTeX 2e Document.
% 
% $Id: Set.tex,v 1.2 2006/01/10 10:46:35 vdmtools Exp $
% 

%%%%%%%%%%%%%%%%%%%%%%%%%%%%%%%%%%%%%%%%
% PDF compatibility code. 
\makeatletter
\newif\ifpdflatex@
\ifx\pdftexversion\@undefined
\pdflatex@false
%\message{Not using pdf}
\else
\pdflatex@true
%\message{Using pdf}
\fi

\newcommand{\latexorpdf}[2]{
  \ifpdflatex@ #2
  \else #1
  \fi
}

\newcommand{\pformat}{a4paper}

\makeatother
%%%%%%%%%%%%%%%%%%%%%%%%%%%%%%%%%%%%%%%%

%\latexorpdf{
%\documentclass[\pformat,12pt]{article}
%}{
%\documentclass[\pformat,pdftex,12pt]{article}
%}
\documentclass[]{jarticle}

\usepackage[dvips]{color}
\usepackage{array}
\usepackage{longtable}
\usepackage{alltt}
\usepackage{graphics}
\usepackage{vpp}
\usepackage{makeidx}
\makeindex

\definecolor{covered}{rgb}{0,0,0}      %black
%\definecolor{not-covered}{gray}{0.5}   %gray for previewing
%\definecolor{not-covered}{gray}{0.6}   %gray for printing
\definecolor{not-covered}{rgb}{1,0,0}  %red

\newcommand{\InstVarDef}[1]{{\bf #1}}
\newcommand{\TypeDef}[1]{{\bf #1}}
\newcommand{\TypeOcc}[1]{{\it #1}}
\newcommand{\FuncDef}[1]{{\bf #1}}
\newcommand{\FuncOcc}[1]{#1}
\newcommand{\MethodDef}[1]{{\bf #1}}
\newcommand{\MethodOcc}[1]{#1}
\newcommand{\ClassDef}[1]{{\sf #1}}
\newcommand{\ClassOcc}[1]{#1}
\newcommand{\ModDef}[1]{{\sf #1}}
\newcommand{\ModOcc}[1]{#1}

%\nolinenumbering
%\setindent{outer}{\parindent}
%\setindent{inner}{0.0em}

\title{FSet���C�u�����[}
\author{
�����L
���{�t�B�b�c�������\\
���Z�p������\\
TEL : 03-3623-4683\\
shin.sahara@jfits.co.jp\\
}
%\date{2004�N2��16��}

\begin{document}
\setlength{\baselineskip}{12pt plus .1pt}
\tolerance 10000
\maketitle

\begin{abstract}
\setlength{\baselineskip}{12pt plus .1pt}
�W���^�Ɋւ��֐���񋟂��郂�W���[���ł���B
\end{abstract}
%\vspace{-1cm}

% LaTeX 2e Document.
% 
% $Id: Set.tex,v 1.2 2006/01/10 10:46:35 vdmtools Exp $
% 

%%%%%%%%%%%%%%%%%%%%%%%%%%%%%%%%%%%%%%%%
% PDF compatibility code. 
\makeatletter
\newif\ifpdflatex@
\ifx\pdftexversion\@undefined
\pdflatex@false
%\message{Not using pdf}
\else
\pdflatex@true
%\message{Using pdf}
\fi

\newcommand{\latexorpdf}[2]{
  \ifpdflatex@ #2
  \else #1
  \fi
}

\newcommand{\pformat}{a4paper}

\makeatother
%%%%%%%%%%%%%%%%%%%%%%%%%%%%%%%%%%%%%%%%

%\latexorpdf{
%\documentclass[\pformat,12pt]{article}
%}{
%\documentclass[\pformat,pdftex,12pt]{article}
%}
\documentclass[]{jarticle}

\usepackage[dvips]{color}
\usepackage{array}
\usepackage{longtable}
\usepackage{alltt}
\usepackage{graphics}
\usepackage{vpp}
\usepackage{makeidx}
\makeindex

\definecolor{covered}{rgb}{0,0,0}      %black
%\definecolor{not-covered}{gray}{0.5}   %gray for previewing
%\definecolor{not-covered}{gray}{0.6}   %gray for printing
\definecolor{not-covered}{rgb}{1,0,0}  %red

\newcommand{\InstVarDef}[1]{{\bf #1}}
\newcommand{\TypeDef}[1]{{\bf #1}}
\newcommand{\TypeOcc}[1]{{\it #1}}
\newcommand{\FuncDef}[1]{{\bf #1}}
\newcommand{\FuncOcc}[1]{#1}
\newcommand{\MethodDef}[1]{{\bf #1}}
\newcommand{\MethodOcc}[1]{#1}
\newcommand{\ClassDef}[1]{{\sf #1}}
\newcommand{\ClassOcc}[1]{#1}
\newcommand{\ModDef}[1]{{\sf #1}}
\newcommand{\ModOcc}[1]{#1}

%\nolinenumbering
%\setindent{outer}{\parindent}
%\setindent{inner}{0.0em}

\title{FSet���C�u�����[}
\author{
�����L
���{�t�B�b�c�������\\
���Z�p������\\
TEL : 03-3623-4683\\
shin.sahara@jfits.co.jp\\
}
%\date{2004�N2��16��}

\begin{document}
\setlength{\baselineskip}{12pt plus .1pt}
\tolerance 10000
\maketitle

\begin{abstract}
\setlength{\baselineskip}{12pt plus .1pt}
�W���^�Ɋւ��֐���񋟂��郂�W���[���ł���B
\end{abstract}
%\vspace{-1cm}

% LaTeX 2e Document.
% 
% $Id: Set.tex,v 1.2 2006/01/10 10:46:35 vdmtools Exp $
% 

%%%%%%%%%%%%%%%%%%%%%%%%%%%%%%%%%%%%%%%%
% PDF compatibility code. 
\makeatletter
\newif\ifpdflatex@
\ifx\pdftexversion\@undefined
\pdflatex@false
%\message{Not using pdf}
\else
\pdflatex@true
%\message{Using pdf}
\fi

\newcommand{\latexorpdf}[2]{
  \ifpdflatex@ #2
  \else #1
  \fi
}

\newcommand{\pformat}{a4paper}

\makeatother
%%%%%%%%%%%%%%%%%%%%%%%%%%%%%%%%%%%%%%%%

%\latexorpdf{
%\documentclass[\pformat,12pt]{article}
%}{
%\documentclass[\pformat,pdftex,12pt]{article}
%}
\documentclass[]{jarticle}

\usepackage[dvips]{color}
\usepackage{array}
\usepackage{longtable}
\usepackage{alltt}
\usepackage{graphics}
\usepackage{vpp}
\usepackage{makeidx}
\makeindex

\definecolor{covered}{rgb}{0,0,0}      %black
%\definecolor{not-covered}{gray}{0.5}   %gray for previewing
%\definecolor{not-covered}{gray}{0.6}   %gray for printing
\definecolor{not-covered}{rgb}{1,0,0}  %red

\newcommand{\InstVarDef}[1]{{\bf #1}}
\newcommand{\TypeDef}[1]{{\bf #1}}
\newcommand{\TypeOcc}[1]{{\it #1}}
\newcommand{\FuncDef}[1]{{\bf #1}}
\newcommand{\FuncOcc}[1]{#1}
\newcommand{\MethodDef}[1]{{\bf #1}}
\newcommand{\MethodOcc}[1]{#1}
\newcommand{\ClassDef}[1]{{\sf #1}}
\newcommand{\ClassOcc}[1]{#1}
\newcommand{\ModDef}[1]{{\sf #1}}
\newcommand{\ModOcc}[1]{#1}

%\nolinenumbering
%\setindent{outer}{\parindent}
%\setindent{inner}{0.0em}

\title{FSet���C�u�����[}
\author{
�����L
���{�t�B�b�c�������\\
���Z�p������\\
TEL : 03-3623-4683\\
shin.sahara@jfits.co.jp\\
}
%\date{2004�N2��16��}

\begin{document}
\setlength{\baselineskip}{12pt plus .1pt}
\tolerance 10000
\maketitle

\begin{abstract}
\setlength{\baselineskip}{12pt plus .1pt}
�W���^�Ɋւ��֐���񋟂��郂�W���[���ł���B
\end{abstract}
%\vspace{-1cm}

\include{test/Set.vpp}

\include{test/SetT.vpp}

%\newpage
%\addcontentsline{toc}{section}{Index}
%\printindex

\end{document}


\include{test/SetT.vpp}

%\newpage
%\addcontentsline{toc}{section}{Index}
%\printindex

\end{document}


\include{test/SetT.vpp}

%\newpage
%\addcontentsline{toc}{section}{Index}
%\printindex

\end{document}

\include{FSetT.vpp}	
% LaTeX 2e Document.
% 
% $Id: String.tex,v 1.2 2006/01/10 10:46:35 vdmtools Exp $
% 

%%%%%%%%%%%%%%%%%%%%%%%%%%%%%%%%%%%%%%%%
% PDF compatibility code. 
\makeatletter
\newif\ifpdflatex@
\ifx\pdftexversion\@undefined
\pdflatex@false
%\message{Not using pdf}
\else
\pdflatex@true
%\message{Using pdf}
\fi

\newcommand{\latexorpdf}[2]{
  \ifpdflatex@ #2
  \else #1
  \fi
}

\newcommand{\pformat}{a4paper}

\makeatother
%%%%%%%%%%%%%%%%%%%%%%%%%%%%%%%%%%%%%%%%

%\latexorpdf{
%\documentclass[\pformat,12pt]{article}
%}{
%\documentclass[\pformat,pdftex,12pt]{article}
%}
\documentclass[]{jarticle}

\usepackage[dvips]{color}
\usepackage{array}
\usepackage{longtable}
\usepackage{alltt}
\usepackage{graphics}
\usepackage{vpp}
\usepackage{makeidx}
\makeindex

\definecolor{covered}{rgb}{0,0,0}      %black
%\definecolor{not-covered}{gray}{0.5}   %gray for previewing
%\definecolor{not-covered}{gray}{0.6}   %gray for printing
\definecolor{not-covered}{rgb}{1,0,0}  %red

\newcommand{\InstVarDef}[1]{{\bf #1}}
\newcommand{\TypeDef}[1]{{\bf #1}}
\newcommand{\TypeOcc}[1]{{\it #1}}
\newcommand{\FuncDef}[1]{{\bf #1}}
\newcommand{\FuncOcc}[1]{#1}
\newcommand{\MethodDef}[1]{{\bf #1}}
\newcommand{\MethodOcc}[1]{#1}
\newcommand{\ClassDef}[1]{{\sf #1}}
\newcommand{\ClassOcc}[1]{#1}
\newcommand{\ModDef}[1]{{\sf #1}}
\newcommand{\ModOcc}[1]{#1}

%\nolinenumbering
%\setindent{outer}{\parindent}
%\setindent{inner}{0.0em}

\title{FString���C�u�����[}
\author{
�����L
���{�t�B�b�c�������\\
���Z�p������\\
TEL : 03-3623-4683\\
shin.sahara@jfits.co.jp\\
}
%\date{2004�N2��16��}

\begin{document}
\setlength{\baselineskip}{12pt plus .1pt}
\tolerance 10000
\maketitle

\begin{abstract}
\setlength{\baselineskip}{12pt plus .1pt}
������iseq of char�j�Ɋւ��֐���񋟂��郂�W���[���ł���B
\end{abstract}
%\vspace{-1cm}

% LaTeX 2e Document.
% 
% $Id: String.tex,v 1.2 2006/01/10 10:46:35 vdmtools Exp $
% 

%%%%%%%%%%%%%%%%%%%%%%%%%%%%%%%%%%%%%%%%
% PDF compatibility code. 
\makeatletter
\newif\ifpdflatex@
\ifx\pdftexversion\@undefined
\pdflatex@false
%\message{Not using pdf}
\else
\pdflatex@true
%\message{Using pdf}
\fi

\newcommand{\latexorpdf}[2]{
  \ifpdflatex@ #2
  \else #1
  \fi
}

\newcommand{\pformat}{a4paper}

\makeatother
%%%%%%%%%%%%%%%%%%%%%%%%%%%%%%%%%%%%%%%%

%\latexorpdf{
%\documentclass[\pformat,12pt]{article}
%}{
%\documentclass[\pformat,pdftex,12pt]{article}
%}
\documentclass[]{jarticle}

\usepackage[dvips]{color}
\usepackage{array}
\usepackage{longtable}
\usepackage{alltt}
\usepackage{graphics}
\usepackage{vpp}
\usepackage{makeidx}
\makeindex

\definecolor{covered}{rgb}{0,0,0}      %black
%\definecolor{not-covered}{gray}{0.5}   %gray for previewing
%\definecolor{not-covered}{gray}{0.6}   %gray for printing
\definecolor{not-covered}{rgb}{1,0,0}  %red

\newcommand{\InstVarDef}[1]{{\bf #1}}
\newcommand{\TypeDef}[1]{{\bf #1}}
\newcommand{\TypeOcc}[1]{{\it #1}}
\newcommand{\FuncDef}[1]{{\bf #1}}
\newcommand{\FuncOcc}[1]{#1}
\newcommand{\MethodDef}[1]{{\bf #1}}
\newcommand{\MethodOcc}[1]{#1}
\newcommand{\ClassDef}[1]{{\sf #1}}
\newcommand{\ClassOcc}[1]{#1}
\newcommand{\ModDef}[1]{{\sf #1}}
\newcommand{\ModOcc}[1]{#1}

%\nolinenumbering
%\setindent{outer}{\parindent}
%\setindent{inner}{0.0em}

\title{FString���C�u�����[}
\author{
�����L
���{�t�B�b�c�������\\
���Z�p������\\
TEL : 03-3623-4683\\
shin.sahara@jfits.co.jp\\
}
%\date{2004�N2��16��}

\begin{document}
\setlength{\baselineskip}{12pt plus .1pt}
\tolerance 10000
\maketitle

\begin{abstract}
\setlength{\baselineskip}{12pt plus .1pt}
������iseq of char�j�Ɋւ��֐���񋟂��郂�W���[���ł���B
\end{abstract}
%\vspace{-1cm}

% LaTeX 2e Document.
% 
% $Id: String.tex,v 1.2 2006/01/10 10:46:35 vdmtools Exp $
% 

%%%%%%%%%%%%%%%%%%%%%%%%%%%%%%%%%%%%%%%%
% PDF compatibility code. 
\makeatletter
\newif\ifpdflatex@
\ifx\pdftexversion\@undefined
\pdflatex@false
%\message{Not using pdf}
\else
\pdflatex@true
%\message{Using pdf}
\fi

\newcommand{\latexorpdf}[2]{
  \ifpdflatex@ #2
  \else #1
  \fi
}

\newcommand{\pformat}{a4paper}

\makeatother
%%%%%%%%%%%%%%%%%%%%%%%%%%%%%%%%%%%%%%%%

%\latexorpdf{
%\documentclass[\pformat,12pt]{article}
%}{
%\documentclass[\pformat,pdftex,12pt]{article}
%}
\documentclass[]{jarticle}

\usepackage[dvips]{color}
\usepackage{array}
\usepackage{longtable}
\usepackage{alltt}
\usepackage{graphics}
\usepackage{vpp}
\usepackage{makeidx}
\makeindex

\definecolor{covered}{rgb}{0,0,0}      %black
%\definecolor{not-covered}{gray}{0.5}   %gray for previewing
%\definecolor{not-covered}{gray}{0.6}   %gray for printing
\definecolor{not-covered}{rgb}{1,0,0}  %red

\newcommand{\InstVarDef}[1]{{\bf #1}}
\newcommand{\TypeDef}[1]{{\bf #1}}
\newcommand{\TypeOcc}[1]{{\it #1}}
\newcommand{\FuncDef}[1]{{\bf #1}}
\newcommand{\FuncOcc}[1]{#1}
\newcommand{\MethodDef}[1]{{\bf #1}}
\newcommand{\MethodOcc}[1]{#1}
\newcommand{\ClassDef}[1]{{\sf #1}}
\newcommand{\ClassOcc}[1]{#1}
\newcommand{\ModDef}[1]{{\sf #1}}
\newcommand{\ModOcc}[1]{#1}

%\nolinenumbering
%\setindent{outer}{\parindent}
%\setindent{inner}{0.0em}

\title{FString���C�u�����[}
\author{
�����L
���{�t�B�b�c�������\\
���Z�p������\\
TEL : 03-3623-4683\\
shin.sahara@jfits.co.jp\\
}
%\date{2004�N2��16��}

\begin{document}
\setlength{\baselineskip}{12pt plus .1pt}
\tolerance 10000
\maketitle

\begin{abstract}
\setlength{\baselineskip}{12pt plus .1pt}
������iseq of char�j�Ɋւ��֐���񋟂��郂�W���[���ł���B
\end{abstract}
%\vspace{-1cm}

\include{test/String.vpp}

\include{test/StringT.vpp}

%\newpage
%\addcontentsline{toc}{section}{Index}
%\printindex

\end{document}


\include{test/StringT.vpp}

%\newpage
%\addcontentsline{toc}{section}{Index}
%\printindex

\end{document}


\include{test/StringT.vpp}

%\newpage
%\addcontentsline{toc}{section}{Index}
%\printindex

\end{document}

\include{FStringT.vpp}	
% LaTeX 2e Document.
% 
% $Id: Sequence.tex,v 1.2 2006/01/10 10:46:35 vdmtools Exp $
% 

%%%%%%%%%%%%%%%%%%%%%%%%%%%%%%%%%%%%%%%%
% PDF compatibility code. 
\makeatletter
\newif\ifpdflatex@
\ifx\pdftexversion\@undefined
\pdflatex@false
%\message{Not using pdf}
\else
\pdflatex@true
%\message{Using pdf}
\fi

\newcommand{\latexorpdf}[2]{
  \ifpdflatex@ #2
  \else #1
  \fi
}

\newcommand{\pformat}{a4paper}

\makeatother
%%%%%%%%%%%%%%%%%%%%%%%%%%%%%%%%%%%%%%%%

%\latexorpdf{
%\documentclass[\pformat,12pt]{article}
%}{
%\documentclass[\pformat,pdftex,12pt]{article}
%}
\documentclass[]{jarticle}

\usepackage[dvips]{color}
\usepackage{array}
\usepackage{longtable}
\usepackage{alltt}
\usepackage{graphics}
\usepackage{vpp}
\usepackage{makeidx}
\makeindex

\definecolor{covered}{rgb}{0,0,0}      %black
%\definecolor{not-covered}{gray}{0.5}   %gray for previewing
%\definecolor{not-covered}{gray}{0.6}   %gray for printing
\definecolor{not-covered}{rgb}{1,0,0}  %red

\newcommand{\InstVarDef}[1]{{\bf #1}}
\newcommand{\TypeDef}[1]{{\bf #1}}
\newcommand{\TypeOcc}[1]{{\it #1}}
\newcommand{\FuncDef}[1]{{\bf #1}}
\newcommand{\FuncOcc}[1]{#1}
\newcommand{\MethodDef}[1]{{\bf #1}}
\newcommand{\MethodOcc}[1]{#1}
\newcommand{\ClassDef}[1]{{\sf #1}}
\newcommand{\ClassOcc}[1]{#1}
\newcommand{\ModDef}[1]{{\sf #1}}
\newcommand{\ModOcc}[1]{#1}

%\nolinenumbering
%\setindent{outer}{\parindent}
%\setindent{inner}{0.0em}

\title{FSequence���C�u�����[}
\author{
�����L
���{�t�B�b�c�������\\
���Z�p������\\
TEL : 03-3623-4683\\
shin.sahara@jfits.co.jp\\
}
%\date{2004�N2��16��}

\begin{document}
\setlength{\baselineskip}{12pt plus .1pt}
\tolerance 10000
\maketitle

\begin{abstract}
\setlength{\baselineskip}{12pt plus .1pt}
��^�Ɋւ��֐���񋟂��郂�W���[���ł���B
\end{abstract}
%\vspace{-1cm}

% LaTeX 2e Document.
% 
% $Id: Sequence.tex,v 1.2 2006/01/10 10:45:26 vdmtools Exp $
% 

%%%%%%%%%%%%%%%%%%%%%%%%%%%%%%%%%%%%%%%%
% PDF compatibility code. 
\makeatletter
\newif\ifpdflatex@
\ifx\pdftexversion\@undefined
\pdflatex@false
%\message{Not using pdf}
\else
\pdflatex@true
%\message{Using pdf}
\fi

\newcommand{\latexorpdf}[2]{
  \ifpdflatex@ #2
  \else #1
  \fi
}

\newcommand{\pformat}{a4paper}

\makeatother
%%%%%%%%%%%%%%%%%%%%%%%%%%%%%%%%%%%%%%%%

\latexorpdf{
\documentclass[\pformat,12pt]{jarticle}
}{
\documentclass[\pformat,pdftex,12pt]{jarticle}
}


\usepackage[dvips]{color}
\usepackage{longtable}
\usepackage{alltt}
\usepackage{graphics}
\usepackage{vpp}
\usepackage{makeidx}
\makeindex

\definecolor{covered}{rgb}{0,0,0}      %black
%\definecolor{not-covered}{gray}{0.5}   %gray for previewing
%\definecolor{not-covered}{gray}{0.6}   %gray for printing
\definecolor{not-covered}{rgb}{1,0,0}  %red

\newcommand{\InstVarDef}[1]{{\bf #1}}
\newcommand{\TypeDef}[1]{{\bf #1}}
\newcommand{\TypeOcc}[1]{{\it #1}}
\newcommand{\FuncDef}[1]{{\bf #1}}
\newcommand{\FuncOcc}[1]{#1}
\newcommand{\MethodDef}[1]{{\bf #1}}
\newcommand{\MethodOcc}[1]{#1}
\newcommand{\ClassDef}[1]{{\sf #1}}
\newcommand{\ClassOcc}[1]{#1}
\newcommand{\ModDef}[1]{{\sf #1}}
\newcommand{\ModOcc}[1]{#1}


\title{Sequence}
\author{�����L}
\date{�Q�O�O�T�N�R���Q�Q��}

\begin{document}
\maketitle

\section{Introduction}

Sequence���C�u�����B

% LaTeX 2e Document.
% 
% $Id: Sequence.tex,v 1.2 2006/01/10 10:45:26 vdmtools Exp $
% 

%%%%%%%%%%%%%%%%%%%%%%%%%%%%%%%%%%%%%%%%
% PDF compatibility code. 
\makeatletter
\newif\ifpdflatex@
\ifx\pdftexversion\@undefined
\pdflatex@false
%\message{Not using pdf}
\else
\pdflatex@true
%\message{Using pdf}
\fi

\newcommand{\latexorpdf}[2]{
  \ifpdflatex@ #2
  \else #1
  \fi
}

\newcommand{\pformat}{a4paper}

\makeatother
%%%%%%%%%%%%%%%%%%%%%%%%%%%%%%%%%%%%%%%%

\latexorpdf{
\documentclass[\pformat,12pt]{jarticle}
}{
\documentclass[\pformat,pdftex,12pt]{jarticle}
}


\usepackage[dvips]{color}
\usepackage{longtable}
\usepackage{alltt}
\usepackage{graphics}
\usepackage{vpp}
\usepackage{makeidx}
\makeindex

\definecolor{covered}{rgb}{0,0,0}      %black
%\definecolor{not-covered}{gray}{0.5}   %gray for previewing
%\definecolor{not-covered}{gray}{0.6}   %gray for printing
\definecolor{not-covered}{rgb}{1,0,0}  %red

\newcommand{\InstVarDef}[1]{{\bf #1}}
\newcommand{\TypeDef}[1]{{\bf #1}}
\newcommand{\TypeOcc}[1]{{\it #1}}
\newcommand{\FuncDef}[1]{{\bf #1}}
\newcommand{\FuncOcc}[1]{#1}
\newcommand{\MethodDef}[1]{{\bf #1}}
\newcommand{\MethodOcc}[1]{#1}
\newcommand{\ClassDef}[1]{{\sf #1}}
\newcommand{\ClassOcc}[1]{#1}
\newcommand{\ModDef}[1]{{\sf #1}}
\newcommand{\ModOcc}[1]{#1}


\title{Sequence}
\author{�����L}
\date{�Q�O�O�T�N�R���Q�Q��}

\begin{document}
\maketitle

\section{Introduction}

Sequence���C�u�����B

\include{Sequence.vpp}


\newpage
\addcontentsline{toc}{section}{Index}
\printindex


\end{document}



\newpage
\addcontentsline{toc}{section}{Index}
\printindex


\end{document}


\include{test/SequenceT.vpp}

%\newpage
%\addcontentsline{toc}{section}{Index}
%\printindex

\end{document}

\include{FSequenceT.vpp}
\include{FTestDriver.vpp}
\include{FTestLogger.vpp}

%\begin{thebibliography}{9}
\section{参考文献等}
VDM++\cite{CSK2007PP}は、
1970年代中頃にIBMウィーン研究所で開発されたVDM-SL\cite{CSK2007SL}を拡張し、
さらにオブジェクト指向拡張したオープンソース
\footnote{使用に際しては、(株)CSKシステムズとの契約締結が必要になる。}の形式仕様記述言語である。
\bibliographystyle{jplain}
%\bibliography{/Users/sahara/svnw/sahara}
\bibliography{/Users/sahara/bib/saharaUTF8}
%\bibliography{/Users/ssahara/svnwork/sahara}

%\end{thebibliography}

%\newpage
%\addcontentsline{toc}{section}{Index}
\printindex

\end{document}
