\documentclass[a4paper,dvips]{article}
\usepackage[dvips]{color}
\usepackage{vdmsl-2e}
\usepackage{longtable}
\usepackage{makeidx}
\usepackage{ifad}
\usepackage{colortbl}
\usepackage{multicol}
\usepackage{alltt}

\newenvironment{formalparameters}{
\vspace{2ex}\hrule\vspace{1ex}
\noindent\textit{Formal Parameters}\\[2ex]
\begin{tabular}{|>{\ttfamily}l|p{7cm}|}\hline}{%
\hline\hline\end{tabular}\vspace{1ex}}
 
\newcommand{\methodresult}[2]{%
\vspace{2ex}\hrule\vspace{1ex}
\noindent\textit{The Results}\\[2ex]
\begin{tabular}{|>{\ttfamily}l|p{7cm}|}\hline
#1 & #2 \\
\hline\hline\end{tabular}\vspace{1ex}}

\newcommand{\TBW}{\fbox{To be written}}
\newcommand{\TBD}{\fbox{To be discussed}}

\newcommand{\VDMTools}{\textsf{\textbf{VDMTools\raisebox{1ex}{\Pisymbol{psy}{226}}}}} 


% VDMgray environment
\newcolumntype{G}{>{\columncolor[gray]{0.8}}p{.98\columnwidth}}
\newenvironment{VDMgray}%
{\small\begin{alltt}\begin{tabular}{G}}%
{\end{tabular}\end{alltt}\normalsize} 

\usepackage{vdmsl-2e}
\usepackage{vpp}
\newcommand{\vdmstyle}[1]{\texttt{#1}}

\definecolor{covered}{rgb}{0,0,0}      %black
%\definecolor{not_covered}{gray}{0.5}   %gray for previewing
\definecolor{not_covered}{gray}{0.6}   %gray for printing
%\definecolor{not_covered}{rgb}{1,0,0}  %read

%\documentstyle[vdmsl_v1.1.31,alltt,makeidx,ifad,a4]{article}

\newcommand{\VDM}{VDM++}

\newcommand{\StateDef}[1]{{\bf #1}}
\newcommand{\TypeDef}[1]{{\bf #1}}
\newcommand{\TypeOcc}[1]{{\it #1}}
\newcommand{\FuncDef}[1]{{\bf #1}}
\newcommand{\FuncOcc}[1]{#1}
\newcommand{\ModDef}[1]{{\tiny #1}}


\newcommand{\MCL}{Meta-IV Class Library}
\newcommand{\NYI}{Notice, that this function/operation is not fully
  specified in order to capture a possible integration with another
  library than \MCL{}.}
\newcommand{\nfs}{{\em not fully specified\/}}

\makeindex

\begin{document}

\special{!userdict begin /bop-hook{gsave 220 30 translate
0 rotate /Times-Roman findfont 21 scalefont setfont
0 0 moveto (CONFIDENTIAL) show grestore}def end}

\docdef{Specification of the Java Static Semantics}
         {IFAD VDM Tool Group \\
          IFAD A/S}
         {\today,}
         {IFAD-Legacy-2}
         {Report}
         {Under Development}
         {Confidential}
         {}
         {\copyright IFAD}
         {\item[V1.0] First version.}
         {}

\renewcommand{\thepage}{\roman{page}}

\tableofcontents
\newpage
\renewcommand{\thepage}{\arabic{page}}
\setcounter{page}{1}

\parskip12pt
\parindent0pt


\section{Introduction}

This document contains the specification of the Java Static Semantics.
The Java language is described in \cite{Gosling&00}. This
specification will act as a pre-condition to the Java to VDM++
specification. 


\section{Overall Structure}

This section describes the {\bf design} of the specification, the {\bf
strategy} used in the Java to VDM++ translator. In addition to these
VDM modules there must a a module with a backend for VDM++ (converting
the abstract syntax three for VDM++ to concrete ysntax inside RTF? files).

\subsection{Design}

The specification is divided into several modules:

\begin{description} 
\item[JSSDEF: ] (JSSDEF is an abbreviation of Java Static Semantics
for DEFinitions).
  Providing functions/operations for translating top-level Java to VDM++
  definitions.

\item[JSSEXPR: ] (EXPR is an abbreviation of EXPRession). Providing
  functions/operations  for translating Java expressions corresponding
  to VDM++ statements/expressions.

\item[JSSSTMT: ] (STMT is an abbreviation of STateMenT). Providing
  functions/operations for translating Java statements corresponding
  to VDM++ statements.

\item[JSSENV:] (ENV is an abbreviation for ENVironment). Provides
  functions/operations for setting and looking up information about
  the environment in which a construct is being translated in the
  context of.�This includes things such as imports from packages.

\item[JSSAUX:] (AUX is an abbreviation of AUXiliary). Provides auxiliary
  functions and operations to the rest of the specification.

\item[JSSVCM1:] (VCM1 is an abbreviation for Version Control
Management First Pass).
This module is responsible for building data structures holding deferent kind of
information needed by the Static Sematics Checks.

%\item[JSSVCM2:] (VCM2 is an abbreviation for Version Control
%Management Second Pass).
%This module is responsible for building data structures holding deferent kind of
%information needed by the Static Sematics Checks.
%
%\item[JSSVCM2EXPR:] JR?
%
%\item[JSSVCM2STMT:] JR?

\item[JSSERR:] (ERR is an abbreviation for ERRor). This module is
responsible for logging all error messages/warnings produced during
the static semantics check of a collection of Java files.

\item[JSSERRMSG:] (ERRMSG is an abbreviation for ERRor MeSsaGes). This
module is responsible for providing the unique identification of the
different errors which can be reported by the Java Static Semantics
analysis. Note that the source file for this is a file with the
extension \texttt{.txt} used commonly for the specification and the code.

\item[CPP:] (CPP is an abbreviation of C Plus Plus). Describing the
  abstract syntax of Java and C++.

\item[CI:] (CI is an abbreviation fo Context Information). It is used
  to be able to refer back to errors in a concrete syntax such that
  the user can understand error messages.

\item[REP: ] Type Representations Module. Provides the type
  definitions for the semantic domains which are used internally in
  the static semantics to model the types from the abstract syntax and
  the derived types.

\item[AS: ] (AS is an abbreviation of Abstract Syntax). Describing the
  abstract syntax of the VDM-SL/VDM++.

\item[UTIL:] (UTIL is an abbreviation for UTILities). This module
contains a number of operations which serve as common operations to
different VDM specifications inside the specifications for VDMTools
components. 
\end{description}

\include{jssdef.vdm}

\include{jssstmt.vdm}

\include{jssexpr.vdm}

\include{jssenv.vdm}

\include{jssvcm1.vdm}

\include{jssrep.vdm}

\include{jssaux.vdm}

\include{javaerr.vdm}

\include{jsserr.vdm}

\include{jsserrmsg.vdm}

\appendix
\bibliographystyle{iptes}
\bibliography{/home/peter/bib/dan}

\newpage
\addcontentsline{toc}{section}{Index}
\printindex

\end{document}



