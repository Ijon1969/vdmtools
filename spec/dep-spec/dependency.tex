% WHAT
%    Specification of a data structure and corresponding operations
%    for representing the dependencies between VDM++ classes.
%    
% FILE
%    $Source: /home/vdmtools/cvsroot/toolbox/spec/dep-spec/dependency.tex,v $
% VERSION
%    $Revision: 1.2 $
% DATE
%    $Date: 1998/12/03 13:23:51 $
% FORMAT
%    $State: Exp $
% PROJECT
%    INFORMA
% STATUS
%    Under development.
% AUTHOR
%    $Author: jeppe $
% COPYRIGHT
%    (C) 1997 IFAD, Denmark

\documentclass[a4paper,dvips]{article}
\usepackage[dvips]{color}
\usepackage{vdmsl-2e}
\usepackage{longtable}
\usepackage{alltt}
\usepackage{makeidx}
\usepackage{ifad}

\definecolor{covered}{rgb}{0,0,0}      %black
%\definecolor{not_covered}{gray}{0.5}   %gray for previewing
\definecolor{not_covered}{gray}{0.6}   %gray for printing
%\definecolor{not_covered}{rgb}{1,0,0}  %read


%\documentstyle[vdmsl_v1.1.31,alltt,makeidx,ifad,a4]{article}


\newcommand{\StateDef}[1]{{\bf #1}}
\newcommand{\TypeDef}[1]{{\bf #1}}
\newcommand{\TypeOcc}[1]{{\it #1}}
\newcommand{\FuncDef}[1]{{\bf #1}}
\newcommand{\FuncOcc}[1]{#1}
\newcommand{\ModDef}[1]{{\tiny #1}}


\newcommand{\MCL}{Meta-IV Class Library}
\newcommand{\TBW}{TO BE WRITTEN}
\newcommand{\NYI}{Notice, that this function/operation is not fully
  specified in order to capture a possible integration with another
  library than \MCL{}.}
\newcommand{\nfs}{{\em not fully specified\/}}
\newcommand{\vpp}{VDM$^{++}$}
\makeindex

\begin{document}

\section{The Dependency Abstract Syntax} 

This module defines a data structure for the dependency (inheritance
and client ship relations) between classes in a \vpp\ specification.
The information contained in this data structure will be used by the
specification manager while determining the optimal sequence for type
checking the classes in a \vpp\ specification, and by the UML-mapper
while translating a \vpp\ AST into an abstract UML syntax. Furthermore
the structure could also be used to model the dependency between
different modules in a VDM-SL specification.

The dependency is modelled as a directed graph, in which we have
decided to distinguish between three different types of dependency:
\begin{description}
\item[Inheritance:] If class $A$ is subclass of class $B$ this appears
  in the graph as an edge from $A$ to $B$.
\item[Static associations:] If an  instance variable contains an object
  reference (at any level in the AST), this is considered as an
  association from the client to the server class. Consequently this
  will show in the graph as an edge from the client to the server.
\item[Uses associations:] Whenever a method from a foreign class is
  invoked through statements like \verb+a_ref.a_op()+ or
  \verb+new A().a_op()+, with \verb+a_ref+ being an object reference,
  this introduces a dependency between the client and the
  server. However, this dependency, especially in the seccond example
  above, is of a more temporary nature, for which reason we will
  distinguish these references from the static ones. 
\end{description}
Note that the static and uses associations are overlapping in some
way; an instance variable containing an object reference will give
rise to a static association. Later, if the instance variable is used
in a method invocation in the server object, this will also show as an
uses association. On the other hand uses associations need not be
visible as static associations --- consider the statement
\verb+new A().a_op()+ for instance.

\input{dep.vdm.tex}

\end{document}
