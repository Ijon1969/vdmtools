% WHAT
%   The main file of the document describing the specification of 
%   the VDM++ part of the Code Generator Front-End.
%
% FILE
%    $Source: /home/vdmtools/cvsroot/toolbox/spec/cg-spec/vdm++gen.tex,v $
% VERSION
%    $Revision: 1.7 $
% DATE
%    $Date: 1996/09/24 08:07:36 $
% FORMAT
%    $State: Exp $
% PROJECT
%    Afrodite/IDERS
% STATUS
%    Under development.
% AUTHOR
%    $Author: jeppe $
% COPYRIGHT
%    (C) 1993 IFAD, Denmark

\documentclass[a4paper,dvips]{article}
\usepackage[dvips]{color}
\usepackage{vdmsl-2e}
\usepackage{longtable}
\usepackage{alltt}
\usepackage{makeidx}
\usepackage{ifad}

\definecolor{covered}{rgb}{0,0,0}      %black
%\definecolor{not_covered}{gray}{0.5}   %gray for previewing
\definecolor{not_covered}{gray}{0.6}   %gray for printing
%\definecolor{not_covered}{rgb}{1,0,0}  %read


\newcommand{\StateDef}[1]{{\bf #1}}
\newcommand{\TypeDef}[1]{{\bf #1}}
\newcommand{\TypeOcc}[1]{{\it #1}}
\newcommand{\FuncDef}[1]{{\bf #1}}
\newcommand{\FuncOcc}[1]{#1}
\newcommand{\ModDef}[1]{{\tiny #1}}

\newcommand{\MCL}{Meta-IV Class Library }
\newcommand{\vinput}[1]{
\begin{verbatim}
\input #1
\end{verbatim}
}

\newcommand{\nfs}{{\em not fully specified\/}}
\newcommand{\TBW}{To be written}
\newcommand{\NYI}{Notice, that this function/operation is not fully
  specified in order to capture a possible integration with another
  library than \MCL.}

\makeindex

\begin{document}

\special{!userdict begin /bop-hook{gsave 220 30 translate
0 rotate /Times-Roman findfont 21 scalefont setfont
0 0 moveto (CONFIDENTIAL) show grestore}def end}

\docdef{Specification of the VDM++ Code Generator}
         {IFAD VDM Tool Group \\
          The Institute of Applied Computer Science}
         {\today,}
         {IFAD-VDM-29}
         {Report}
         {Under Development}
         {Confidential}
         {}
         {\copyright IFAD}
         {\item[V1.0] First version.}
         {}


\tableofcontents
\newpage


\section{Introduction}

This document contains the first draft of the specification
of the VDM++ part of the code generator front-end. In addition
it describes the overall structure of the specification and
the used strategies. 

The specification in this document supports class definitions
containing the following constructs:

\begin{itemize}
\item Inheritance:
  \begin{itemize}
  \item Representation.
  \item Indexed.
  \item Behavioral.
  \end{itemize}
\item Instance Variables ({\em invariants are not supported}).
\item Explicit member function definitions.
\item Value definitions.
\item Full and preliminary method specifications.
\item Statements:
  \begin{itemize}
  \item Invoke statements.
  \item New statement.
  \item Specification and Topology statements.
  \end{itemize}
\item Self Expressions
 \end{itemize}



\section{Overall Structure}

The specification is divided into the following modules:

\begin{description}
\item[CLASS:] This module provides functions generating code
  corresponding to VDM++ class definitions.

\item[FVD:] This module provides functions generating code
  corresponding to function and value definitions inside VDM++ class
  definitions.

\item[MD:] This module provides functions generating code
  corresponding to method definitions, methods modeling behavioral
  inheritance, and argument and type declarations/definitions.

\item[STMT:] This module is defined [IFAD-CG-4]\footnote{This is a
    doc. id.} bud is extended with functions generating code
  corresponding to VDM++ statements. That is, {\em Call Statement\/}
  has been changed into {\em Invoke Statement\/} and {\em New
    Statement\/}, and where {\em Specification} and {\em Topology
    Statements} have been added.

\item[EXPR:] This module is defined [IFAD-CG-4] and is extended with
  functions generation code corresponding to VDM++ expressions. That
  is, {\em isOfBaseClass Expression},
  {\em isOfClass Expression} and {\em Self Expression\/}. Note that
  only {\em Self Expression\/} is supported in the first version of
  the code generator.

\item[TI:] This module provides operations keeping track of type
  information collected by the type checker. This type information is
  used to determine methods class membership and dependencies between
  classes.

\end{description}

In addition to these modules, the specification uses of the modules
$CPP$, $BC$, $AUX$, $DS$, $PM$, $VD$ and $VD$. These are all described
in [IFAD-CG-4]. The VDM++ code generator uses the abstract syntax
(module $AS$) and the type representation (module $REP$) described in
\cite{Lassen&93a}.


Strategies, such as temporary variables, renaming, specification
style etc.\ used in this document will follow the same guidelines as
described in [IFAD-CG-4]. Classes will be renamed using
the following strings:

\begin{itemize}
\item {\em vdm\_\/} --- classes corresponding to VDM++ classes.

\item {\em iclass\_index\_super\_\/} --- classes modeling indexed
  inheritance from class {\em super\/} with the index {\em index\/}
  (see section \ref{sec:class}).

\end{itemize}


%\subsection{The Meta-IV C++ Library}
%
%
%The first version of the code generator will use the Meta-IV C++
%Library. This raises two problems in the VDM++ part of the code
%generator.
%
%In addition to the VDM-SL type set, VDM++ includes an object type,
%which can be combined with the other types in the usual way. This
%implies that the generated code can contain non Meta-IV objects which
%might be inserted in e.g.\ a Meta-IV sequence. Unfortunately this is
%not possible with out extending the Meta-IV library. The specification
%in this document do not concern with this problem, and is to be
%modified when the consequences to the code are clarified.
%
%Second, the way that indexed inheritance is modeled in the generated
%code can not be compiled with version 2.2.2 of the GNU C++ compiler
%(due to a bug). This bug has been fix in 2.5.8. The code generator
%generates code which is supported by  version 2.5.8 of the GNU C++
%compiler, and the Meta-IV library is to be ported to this version in
%order to be used with the generated code.
%
%The remaining part of this document will not be concerned with the
%former problems. This implies that parts of the specification are to
%modified then the Meta-IV library is extended and ported to GNU C++
%version 2.5.8.
%

\subsection{Run-Time Type Information}

In order to code generate the expressions {\em isOfBaseClass} and {\em
  isOfClass}, some run-time type information most be available (C++
does not provide a standard way of doing such type inquires). However,
the minimal information that a C++ class corresponding to a VDM++
class need, are a unique identification of the class (e.g.\ the class
name) and a unique identification of the root base classes (e.g.\ 
their names). The information can be stored in a member
variable in the class. The first version of the code generator will
not support {\em isOfBaseClass} and {\em isOfClass} and run-time type
information will therefore not be available for C++ classes generated
by the specification in this document.

Guidelines for adding run-time type information are described in
\cite{Stroustrup91}. 





%%%Specification.

%%module CLASS
\input{mod_class.vdm.tex}

%%module FVD
\input{mod_fvd.vdm.tex}

%%module MD
\input{mod_md.vdm.tex}

\section{Function Definitions}\label{fctdef}
\input{mod_fctdef.vdmpp.tex}


\section{Expressions}\label{expr}
\input{mod_expr.vdmpp.tex}


\section{Pattern Match}\label{PM}
\input{mod_patmat.vdm.tex}


\section{Statements}
\input{mod_stmt.vdmpp.tex}

\section{Type Definitions}
\label{TD}
\input{mod_typedef.vdmpp.tex}

\section{Value Definitions}
\input{mod_valdef.vdmpp.tex}

\appendix

%\section{The Definition of the Abstract Syntax of VDM}
%\label{AS}
%\input{mod_as.vdm.tex}

%\label{AS}
%\input{rep.vdmpp.tex}

%\section{The Definition of the Abstract Syntax of C++}
%\input{mod_cppast.vdm.tex}

%\section{Building the Abstract Syntax Tree of C++}
%This module provides functions for building the abstract syntax tree of C++. 
%\input{mod_bcppast.vdmpp.tex}

%\section{Code Corresponding to Functions on VDM Data Structures}\label{DS}
\input{mod_vdm_ds.vdmpp.tex}


%\section{Auxiliary Module}\label{AUX}
\input{mod_aux.vdmpp.tex}

%%%module TI
%\input{mod_ti.vdm.tex}


%\newpage
%\addcontentsline{toc}{section}{Index}
%\printindex

\end{document}
