\documentclass[11pt]{article}
\usepackage{toolbox,vdmsl-2e,alltt,graphics,makeidx}
\usepackage{ifthen}
\usepackage{verbatimfiles}
\usepackage[dvips]{color}
\usepackage{longtable}
\usepackage{float}


\newcommand{\vdmslpp}[2]{%
#1
}



\newcommand{\Lit}[1]{`{\tt #1}\Quote}
\newcommand{\Rule}[2]{
  \begin{quote}\begin{tabbing}
    #1\index{#1}\ \ \= = \ \ \= #2  ; %    Adds production rule to index
    
  \end{tabbing}\end{quote}
  }
\newcommand{\SeqPt}[1]{\{\ #1\ \}}
\newcommand{\lfeed}{\\ \> \>}
\newcommand{\dsepl}{\ $|$\ }
\newcommand{\dsep}{\\ \> $|$ \>}
\newcommand{\Lop}[1]{`{\sf #1}\Quote}
\newcommand{\blankline}{\vspace{\baselineskip}}
\newcommand{\Brack}[1]{(\ #1\ )}
\newcommand{\nmk}{\footnotemark}
\newcommand{\ntext}[1]{\footnotetext{{\bf Note: } #1}}
\newlength{\kwlen}
\newcommand{\Keyw}[1]{\settowidth{\kwlen}{\tt #1}\makebox[\kwlen][l]{\sf
    #1}}
\newcommand{\keyw}[1]{{\sf #1}}
\newcommand{\id}[1]{{\tt #1}}
\newcommand{\metaiv}[1]{\begin{alltt}\input{#1}\end{alltt}}

\newcommand{\OptPt}[1]{[\ #1\ ]}
\newcommand{\MAP}[2]{\kw{map }#1\kw{ to }#2}
\newcommand{\INMAP}[2]{\kw{inmap }#1\kw{ to }#2}
\newcommand{\SEQ}[1]{\kw{seq of }#1}
\newcommand{\NSEQ}[1]{\kw{seq1 of }#1}
\newcommand{\SET}[1]{\kw{set of }#1}
\newcommand{\PROD}[2]{#1 * #2}
\newcommand{\TO}[2]{$#1 \To #2$}
\newcommand{\FUN}[2]{#1 \To #2}


\newcommand{\MYEQUIV}{$\equiv$}
\newlength{\nonstandlen}
\newcommand{\nonstandard}[1]{%
\setlength{\nonstandlen}{#1\baselineskip}%
  \marginpar{\hspace*{-3mm}\raisebox{\nonstandlen}[0pt][0pt]{\fbox{{\footnotesize Non standard}}}}%
}
\newenvironment{TypeSemantics}{\begin{longtable}[r]{|p{3.5cm}|p{9cm}|}\hline%
  Operator Name & Semantics Description \\ \hline\hline \endhead}% 
  {\hline\end{longtable}}









\makeindex

\newcommand{\Index}[1]{#1\index{#1}}


\parindent0mm

\newcommand{\xfigpicture}[4]{
\begin{figure}[hbt]
\setlength{\unitlength}{1mm}
\begin{center}
\mbox{
\begin{picture}(#1,#2)
\put(0,0){\special{psfile=#3 hscale=70 vscale=55}}
\end{picture} }
\end{center}
\caption{#4}
\end{figure}
}

\newcommand{\dest}[1]{\hfill{\fbox{\bf#1}}}

\begin{document}
\docdef{The IFAD VDM-SL Lite Toolbox}
         {The VDM Tool Group\\
          IFAD}
         {November, 1997}
         {IFAD-VDM-55}
         {Report}
         {Draft}
         {Confidential}
         {\copyright IFAD 1997}
         {\item[V1.0] First version.}

\pagenumbering{roman}
\tableofcontents
%\newpage
%\mbox{}
\newpage
\pagenumbering{arabic}
\parskip2mm


\newcommand{\lite}{{\em Lite Toolbox}}

\section{Introduction} \label{introduction}

This document describes the restrictions of the {\em IFAD VDM-SL Lite
  Toolbox}, hereafter called the {\em Lite Toolbox}. This document
serves as documentation of the Lite Toolbox for IFAD internally.


The Lite Toolbox is a restricted version of the IFAD VDM-SL Toolbox.
The restrictions are listed below:
\begin{itemize}
\item It only supports a subset of the VDM Specification Language. The
  subset is described in Section~\ref{sec:subset}.
\item There is a limit on specification size it can handle. This
  restriction is described in Section~\ref{sec:size}.
\item Restrictions on the tools:
  \begin{itemize}
    \item Interpreter: No restrictions.
    \item Type Checker: The {\em def} option is not supported.
    \item Code generator: Not supported.
    \item Dynamic Link: Not supported.
    \item Pretty Printer: Not supported.
  \end{itemize}
\end{itemize}

\section{Limitation on Specification Size}\label{sec:size}

The \lite\ only supports specifications with less than:
\begin{itemize}
\item 21 function definitions, and
\item 11 implicit operation definitions.
\end{itemize}


If one parses a file that exceeds the limitations, the file is read
and a warning is given. If this happens it will not be possible to
interprete or type check the specification (error messages that the
limititation has been exceeded are written to the log window).

\section{The VDM-SL subset supported}\label{sec:subset}

The supported subset of the VDM-SL is described in
Appendix~\ref{subset_bnf}.

Below is given a list of main constructs that are not supported:
\begin{itemize}
\item modules,
\item explicit operations, extended explicit operations,
\item all statements,
\item extended explicit functions,
\item expression as: iota, lambda, function type instantiations,
  undefined, def, local definitions, floor, power, map inverse, rem,
  mod, proper subset, num exp, compose, record modifier.
\end{itemize}


\appendix

\section{The Syntax of the VDM-SL subset supported}\label{subset_bnf}


\Rule{document}{
   definition block, \SeqPt{definition block}
  }


\subsection{Definitions}

\Rule{definition block}{
  type definitions \dsep
  state definition \dsep
  value definitions \dsep
  function definitions
  \dsep
  operation definitions
  }


\subsubsection{Type Definitions}
\label{mathCSTypeDefs}

\Rule{type definitions}{
  \Lop{types}, type definition, \lfeed
  \SeqPt{\Lit{;}, type definition}, \OptPt{\Lit{;}}
  }%
  \nonstandard{1}

\Rule{type definition}{
  identifier, \Lit{=}, type, \OptPt{invariant} \dsep
  identifier, \Lit{::}, field list, \OptPt{invariant}
  }

\Rule{type}{
  bracketed type \dsep
  basic type \dsep
  quote type \dsep
  union type \dsep
  product type \dsep
  optional type \dsep
  set type \dsep
  seq type \dsep
  map type \dsep
  partial function type \dsep
  type name
  }

\Rule{bracketed type}{
  \Lit{(}, type, \Lit{)}
  }

\Rule{basic type}{
  \Lop{bool} \dsepl
  \Lop{nat} \dsepl
  \Lop{nat1} \dsepl
  \Lop{int} \dsepl
  \Lop{real} \dsepl
  \Lop{char} \dsepl
  \Lop{token} 
  }

\Rule{quote type}{
  quote literal
  }


\Rule{field list}{
  \SeqPt{field}
  }

\Rule{field}{
  \OptPt{identifier, \Lit{:}}, type
  }

\Rule{union type}{
  type, \Lit{|}, type, \SeqPt{\Lit{|}, type}
  }

\Rule{product type}{
  type, \Lit{*}, type, \SeqPt{\Lit{*}, type}
  }

\Rule{optional type}{
  \Lit{[}, type, \Lit{]}
  }

\Rule{set type}{
  \Lop{set of}, type
  }

\Rule{seq type}{
  seq0 type 
  }

\Rule{seq0 type}{
  \Lop{seq of}, type
  }

\Rule{seq1 type}{
  \Lop{seq1 of}, type
  }

\Rule{map type}{
  general map type
  }

\Rule{general map type}{
  \Lop{map}, type, \Lop{to}, type
  }


\Rule{function type}{
  partial function type 
  }

\Rule{partial function type}{
  discretionary type, \Lit{->}, type
  }

\Rule{discretionary type}{
  type \dsep
  \Lit{(}, \Lit{)}
  }

\Rule{type name}{
  name
  }

\subsubsection{The State Definition}

\Rule{state definition}{
  \Lop{state}, identifier, \Lop{of}, field list, \lfeed
  \OptPt{invariant}, \OptPt{initialisation}, \Lop{end}, \OptPt{\Lit{;}}
  }\nonstandard{1}
\Rule{invariant}{
  \Lop{inv}, invariant initial function
  }

\Rule{initialisation}{
  \Lop{init}, invariant initial function
  }

\Rule{invariant initial function}{
  pattern, \Lit{==}, expression
  }


\subsubsection{Function Definitions}\label{functiondef2}

\Rule{function definitions}{
  \Lop{functions}, function definition, \lfeed
  \SeqPt{\Lit{;}, function definition}, \OptPt{\Lit{;}}
  }
  \nonstandard{1}

\Rule{function definition}{
  explicit function definition \dsep
  implicit function definition
  }%

  \nonstandard{1}


\Rule{explicit function definition}{
  identifier,
  \Lit{:}, function type, \lfeed
  identifier, parameters list, \lfeed
  \Lit{==}, expression, \lfeed
  \OptPt{\Lop{pre}, expression}
    , \lfeed \OptPt{\Lop{post}, expression}
  }
  \nonstandard{1}

\Rule{implicit function definition}{
  identifier, \OptPt{type variable \vdmslpp{list}{}}, \lfeed 
  parameter types , identifier type pair list, \lfeed
  \OptPt{\Lop{pre}, expression}, \lfeed
  \Lop{post}, expression
  }



\Rule{identifier type pair}{
  identifier, \Lit{:}, type
  }

\Rule{parameter types}{
  \Lit{(}, \OptPt{pattern type pair list}, \Lit{)}
  }

\Rule{identifier type pair list}{
  identifier, \Lit{:}, type, \lfeed
  \SeqPt{\Lit{,}, identifier, \Lit{:}, type}
}

\Rule{pattern type pair list}{
  pattern list, \Lit{:}, type, \SeqPt{\Lit{,}, pattern list,\Lit{:}, type}
  }

\Rule{parameters list}{
  parameters
  }

\Rule{parameters}{
  \Lit{(}, \OptPt{pattern list}, \Lit{)}
  }




\subsubsection{Operation Definitions}\label{op-def2}

\Rule{operation definitions}{
  \Lop{operations}, operation definition, \lfeed
  \SeqPt{\Lit{;}, operation definition}, \OptPt{\Lit{;}}
  }\nonstandard{1}

\Rule{operation definition}{
  implicit operation definition 
  }\nonstandard{1}

\Rule{implicit operation definition}{
  identifier, parameter types,
  \OptPt{identifier type pair list}, \lfeed
  \OptPt{externals}, \lfeed
  \OptPt{\Lop{pre}, expression}, \lfeed
  \Lop{post}, expression, \lfeed
  \OptPt{exceptions}
  }


\Rule{externals}{
  \Lop{ext}, var information, \SeqPt{var information}
  }

\Rule{var information}{
  mode, name list, \OptPt{\Lit{:}, type}
  }

\Rule{mode}{
  \Lop{rd} \dsepl \Lop{wr}
  }



\subsection{Expressions}

\Rule{expression list}{
  expression, \SeqPt{\Lit{,}, expression}
  }

\Rule{expression}{
  bracketed expression \dsep
  let expression \dsep
  let be expression \dsep
  if expression \dsep
  cases expression \dsep
  unary expression \dsep
  binary expression \dsep
  quantified expression \dsep
  set enumeration \dsep
  set comprehension \dsep
  set range expression \dsep
  sequence enumeration \dsep
  sequence comprehension \dsep
  subsequence \dsep
  map enumeration \dsep
  map comprehension \dsep
  tuple constructor \dsep
  record constructor \dsep 
  apply \dsep
  field select \dsep
  is expression \dsep
  name \dsep
  old name  \dsep
  symbolic literal % \dsep
%  input expression
}%
\nonstandard{4}
%\marginpar{\vspace*{-53mm}\hspace*{-3mm}\fbox{{\footnotesize Non standard}}\vspace*{53mm}}%
%\marginpar{\vspace*{-25mm}\hspace*{-3mm}\fbox{{\footnotesize Non standard}}\vspace*{25mm}}%
%\marginpar{\vspace*{-23mm}\hspace*{-3mm}\fbox{{\footnotesize Non standard}}\vspace*{23mm}}


\subsubsection{Bracketed Expressions}

\Rule{bracketed expression}{
  \Lit{(}, expression, \Lit{)}
  }

\subsubsection{Local Binding Expressions}

\Rule{let expression}{
  \Lop{let}, local definition,
   \SeqPt{\Lit{,}, local definition}, \lfeed
  \Lop{in}, expression
  }

\Rule{local definition}{
  value definition 
  }


\Rule{let be expression}{
  \Lop{let}, bind,
  \OptPt{\Lop{be}, \Lop{st}, expression}, \Lop{in}, expression
  }


\subsubsection{Conditional Expressions}

\Rule{if expression}{
  \Lop{if}, expression, \Lop{then}, expression,
  \SeqPt{elseif expression}, \lfeed
  \Lop{else}, expression
  }

\Rule{elseif expression}{
  \Lop{elseif}, expression, \Lop{then}, expression
  }

\Rule{cases expression}{
  \Lop{cases}, expression, \Lit{:}, 
  cases expression alternatives, \lfeed
  \OptPt{\Lit{,}, others expression}, \Lop{end}
  }

\Rule{cases expression alternatives}{
  cases expression alternative, \lfeed
  \SeqPt{\Lit{,}, cases expression alternative}
  }

\Rule{cases expression alternative}{
  pattern list, \Lit{->}, expression
  }

\Rule{others expression}{
  \Lop{others}, \Lit{->}, expression
  }

\subsubsection{Unary Expressions}

\Rule{unary expression}{
  prefix expression
  }

\Rule{prefix expression}{
  unary operator, expression
  }

\Rule{unary operator}{
  unary plus \dsep
  unary minus \dsep
  arithmetic abs \dsep
  not \dsep
  set cardinality \dsep
  finite power set \dsep
  distributed set union \dsep
  distributed set intersection \dsep
  sequence head \dsep
  sequence tail \dsep
  sequence length \dsep
  sequence elements \dsep
  sequence indices \dsep
  distributed sequence concatenation \dsep
  map domain \dsep
  map range \dsep
  distributed map merge
  }

\Rule{unary plus}{
  \Lit{+}
  }

\Rule{unary minus}{
  \Lit{-}
  }

\Rule{arithmetic abs}{
  \Lop{abs}
  }

\Rule{floor}{
  \Lop{floor}
  }

\Rule{not}{
  \Lop{not}
  }

\Rule{set cardinality}{
  \Lop{card}
  }


\Rule{distributed set union}{
  \Lop{dunion}
  }

\Rule{distributed set intersection}{
  \Lop{dinter}
  }

\Rule{sequence head}{
  \Lop{hd}
  }

\Rule{sequence tail}{
  \Lop{tl}
  }

\Rule{sequence length}{
  \Lop{len}
  }

\Rule{sequence elements}{
  \Lop{elems}
  }

\Rule{sequence indices}{
  \Lop{inds}
  }

\Rule{distributed sequence concatenation}{
  \Lop{conc}
  }

\Rule{map domain}{
  \Lop{dom}
  }

\Rule{map range}{
  \Lop{rng}
  }

\Rule{distributed map merge}{
  \Lop{merge}
  }

\Rule{map inverse}{
  \Lop{inverse}, expression
  }

\subsubsection{Binary Expressions}

\Rule{binary expression}{
  expression, binary operator, expression
  }

\Rule{binary operator}{
  arithmetic plus \dsep
  arithmetic minus \dsep
  arithmetic multiplication \dsep
  arithmetic divide \dsep
  arithmetic integer division \dsep
  arithmetic mod \dsep
  less than \dsep
  less than or equal \dsep
  greater than \dsep
  greater than or equal \dsep
  equal \dsep
  not equal \dsep
  or \dsep
  and \dsep
  imply \dsep
  logical equivalence \dsep
  in set \dsep
  not in set \dsep
  subset \dsep
  set union \dsep
  set difference \dsep
  set intersection \dsep
  sequence concatenate \dsep
  map or sequence modify \dsep
  map merge \dsep
  map domain restrict to \dsep
  map domain restrict by \dsep
  map range restrict to \dsep
  map range restrict by 
}

\Rule{arithmetic plus}{
  \Lit{+}
  }

\Rule{arithmetic minus}{
  \Lit{-}
  }

\Rule{arithmetic multiplication}{
  \Lit{*}
  }

\Rule{arithmetic divide}{
  \Lit{/}
  }

\Rule{arithmetic integer division}{
  \Lop{div}
  }


\Rule{arithmetic mod}{
  \Lop{mod}
  }

\Rule{less than}{
  \Lit{<}
  }

\Rule{less than or equal}{
  \Lit{<=}
  }

\Rule{greater than}{
  \Lit{>}
  }

\Rule{greater than or equal}{
  \Lit{>=}
  }

\Rule{equal}{
  \Lit{=}
  }

\Rule{not equal}{
  \Lit{<>}
  }


\Rule{or}{
  \Lop{or}
  }

\Rule{and}{
  \Lop{and}
  }

\Rule{imply}{
  \Lit{=>}
  }

\Rule{logical equivalence}{
  \Lit{<=>}
  }

\Rule{in set}{
  \Lop{in set}
  }

\Rule{not in set}{
  \Lop{not in set}
  }

\Rule{subset}{
  \Lop{subset}
  }

\Rule{proper subset}{
  \Lop{psubset}
  }

\Rule{set union}{
  \Lop{union}
  }

\Rule{set difference}{
  \Lit{\char'134}
  }

\Rule{set intersection}{
  \Lop{inter}
  }

\Rule{sequence concatenate}{
  \Lit{\char'136}
  }

\Rule{map or sequence modify}{
  \Lit{++}
  }

\Rule{map merge}{
  \Lop{munion}
  }

\Rule{map domain restrict to}{
  \Lit{<:}
  }

\Rule{map domain restrict by}{
  \Lit{<-:}
  }

\Rule{map range restrict to}{
  \Lit{:>}
  }

\Rule{map range restrict by}{
  \Lit{:->}
  }

\Rule{composition}{
  \Lop{comp}
  }

\Rule{iterate}{
  \Lit{**}
  }

\subsubsection{Quantified Expressions}

\Rule{quantified expression}{
  all expression \dsep
  exists expression \dsep
  exists unique expression
  }

\Rule{all expression}{
  \Lop{forall}, bind list, \Lit{\&}, expression
  }

\Rule{exists expression}{
  \Lop{exists}, bind list, \Lit{\&}, expression
  }

\Rule{exists unique expression}{
  \Lop{exists1}, bind, \Lit{\&}, expression
  }


\subsubsection{Set Expressions}

\Rule{set enumeration}{
  \Lit{\{}, \OptPt{expression list}, \Lit{\}}
  }

\Rule{set comprehension}{
  \Lit{\{}, expression, \Lit{|}, bind list, \OptPt{\Lit{\&}, expression},
  \Lit{\}}
  }

\Rule{set range expression}{
  \Lit{\{}, expression, \Lit{,}, \Lit{\Range}, \Lit{,}, expression,
  \Lit{\}}
  }

\subsubsection{Sequence Expressions}

\Rule{sequence enumeration}{
  \Lit{[}, \OptPt{expression list}, \Lit{]}
  }

\Rule{sequence comprehension}{
  \Lit{[}, expression, \Lit{|}, set bind, \OptPt{\Lit{\&}, expression},
  \Lit{]}
  }

\Rule{subsequence}{
  expression, \Lit{(}, expression, \Lit{,}, \Lit{\Range}, \Lit{,},
  expression, \Lit{)} 
  }

\subsubsection{Map Expressions}

\Rule{map enumeration}{
  \Lit{\{}, maplet, \SeqPt{\Lit{,}, maplet}, \Lit{\}} \dsep
  \Lit{\{}, \Lit{|->}, \Lit{\}}
  }

\Rule{maplet}{
  expression, \Lit{|->}, expression
  }

\Rule{map comprehension}{
  \Lit{\{}, maplet, \Lit{|}, bind list, \OptPt{\Lit{\&}, expression},
  \Lit{\}}
  }

\subsubsection{The Tuple Constructor Expression}

\Rule{tuple constructor}{
  \Lit{mk\_}, \Lit{(}, expression, expression list, \Lit{)}
  }

\subsubsection{Record Expressions}

\Rule{record constructor}{
  \Lop{mk\_},\nmk name, \Lit{(}, \OptPt{expression list}, \Lit{)}
  }
\ntext{no delimiter is allowed}


\Rule{record modification}{
  identifier, \Lit{|->}, expression
  }

\subsubsection{Apply Expressions}

\Rule{apply}{
  expression, \Lit{(}, \OptPt{expression list}, \Lit{)}
  }

\Rule{field select}{
  expression, \Lit{.}, identifier
  }

  
\subsubsection{The Is Expression}

\Rule{is expression}{
  \Lit{is\_},\nmk name, \Lit{(}, expression, \Lit{)} \dsep
  is basic type, \Lit{(}, expression, \Lit{)}
  }
\ntext{no delimiter is allowed}


\subsubsection{Names}

\Rule{name}{
  identifier
  }

\Rule{name list}{
  name, \SeqPt{\Lit{,}, name}
  }

\Rule{old name}{
  identifier, \Lit{\char'176}
  }



\subsection{Patterns and Bindings}

\subsubsection{Patterns}\label{patterns2}

\Rule{pattern}{
  pattern identifier \dsep
  match value \dsep
  set enum pattern \dsep
  set union pattern \dsep
  seq enum pattern \dsep
  seq conc pattern \dsep
  tuple pattern \dsep
  record pattern
  }

\Rule{pattern identifier}{
  identifier \dsepl \Lit{-}
  }

\Rule{match value}{
  \Lit{(}, expression, \Lit{)} \dsep
  symbolic literal
  }

\Rule{set enum pattern}{
  \Lit{\{}, \OptPt{pattern list}, \Lit{\}}
  }\nonstandard{1}

\Rule{set union pattern}{
  pattern, \Lit{union}, pattern
  }

\Rule{seq enum pattern}{
  \Lit{[}, \OptPt{pattern list}, \Lit{]}
  }\nonstandard{1}

\Rule{seq conc pattern}{
  pattern, \Lit{\char'136}, pattern
  }

\Rule{tuple pattern}{
  \Lit{mk\_}, \Lit{(}, pattern, \Lit{,}, pattern list, \Lit{)}
  } 

\Rule{record pattern}{
  \Lit{mk\_},\nmk name, \Lit{(}, \OptPt{pattern list}, \Lit{)}
  }
\ntext{no delimiter is allowed}

\Rule{pattern list}{
  pattern, \SeqPt{\Lit{,}, pattern}
  }


\subsubsection{Bindings}


\Rule{bind}{
  set bind \dsepl type bind
  }

\Rule{set bind}{
  pattern, \Lop{in set}, expression
  }

\Rule{type bind}{
  pattern, \Lit{:}, type
  }

\Rule{bind list}{
  multiple bind, \SeqPt{\Lit{,}, multiple bind}
  }

\Rule{multiple bind}{
  multiple set bind \dsep
  multiple type bind
  }

\Rule{multiple set bind}{
  pattern list, \Lop{in set}, expression
  }

\Rule{multiple type bind}{
  pattern list, \Lit{:}, type
  }





\end{document}
