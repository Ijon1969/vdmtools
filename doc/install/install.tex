%%
%% Toolbox Language Manual
%% $Id: install.tex,v 1.3 2006/04/19 10:30:10 vdmtools Exp $
%% 

%%%%%%%%%%%%%%%%%%%%%%%%%%%%%%%%%%%%%%%%
% PDF compatibility code. 

\makeatletter
\newif\ifpdflatex@
\ifx\pdftexversion\@undefined
\pdflatex@false
%\message{Not using pdf}
\else
\pdflatex@true
%\message{Using pdf}
\fi

\newcommand{\latexpdf}[2]{
  \ifpdflatex@ #1
  \else #2
  \fi
}

\newcommand{\latexorpdf}[2]{
  \ifpdflatex@ #2
  \else #1
  \fi
}

#ifdef A4Format
\newcommand{\pformat}{a4paper}
#endif A4Format
#ifdef LetterFormat
\newcommand{\pformat}{letterpaper}
#endif LetterFormat

\makeatother

%%%%%%%%%%%%%%%%%%%%%%%%%%%%%%%%%%%%%%%%

\latexorpdf{
\documentclass[\pformat,12pt]{article}
}{
% pdftex option is used by graphic[sx],hyperref,toolbox.sty
\documentclass[\pformat,pdftex,12pt]{article}
}

\usepackage{toolbox}
\usepackage{vdmsl-2e}
\usepackage{alltt}
\usepackage{graphics}
%\usepackage{path}
%\usepackage{palatino}
\usepackage{ifthen}
\usepackage{verbatimfiles}

#ifdef VDMPP
\usepackage{vpp}
#endif VDMPP

% Ueki change start
#ifdef JPN
\AtBeginDvi{\special{pdf:tounicode 90ms-RKSJ-UCS2}}
#endif JPN

\usepackage[dvipdfm,bookmarks=true,bookmarksnumbered=true,colorlinks,plainpages=true]{hyperref}

\def\seename{$\Rightarrow$}
% Ueki change end

% Ueki delete start
%\latexorpdf{
%\usepackage[plainpages=true,colorlinks,linkcolor=black,citecolor=black,pagecolor=black, urlcolor=black]{hyperref}
%}{
%\usepackage[plainpages=true,colorlinks]{hyperref}
%}
% Ueki delete end

\makeindex

\def\vdmsl{{\small VDM-SL}}
\def\vdmpp{{\small VDM++}}
#ifdef VDMSL
\newcommand{\ifSLPP}[2]{#1-$<$version$>$}
\newcommand{\vdmslpp}{VDM-SL}
\newcommand{\vdmslppEm}{VDM-SL}
\newcommand{\Toolbox}{VDMTools}
\newcommand{\toolbox}{Toolbox}
\newcommand{\vdmde}{vdmde}
\newcommand{\vdmgde}{vdmgde}
\newcommand{\vdmhome}{vdmhome}
\newcommand{\vdmdeNineteenEl}{vdmde.el}
\newcommand{\VdmSlPp}{\VdmSl}
#endif VDMSL
#ifdef VDMPP
\newcommand{\ifSLPP}[2]{#2-<version>}
\newcommand{\vdmslpp}{VDM++}
\newcommand{\vdmslppEm}{VDM\/++}
\newcommand{\Toolbox}{VDMTools}
\newcommand{\toolbox}{toolbox}
\newcommand{\vdmde}{vppde}
\newcommand{\vdmgde}{vppgde}
\newcommand{\vdmhome}{vpphome}
\newcommand{\vdmdeNineteenEl}{vppde.el}
\DeclareRobustCommand{\VdmSlPp}{VDM++-\VdmSl}
#endif VDMPP
\newcommand{\cg}{\vdmslpp\raisebox{-0.6ex}{to}C++Code Generator}
%\newcommand{\licensedat}{\texttt{license.dat}}

% The use of VDMSL/VDMPP ifdef's have basicly been exchanged with the
% use of LaTeX ifthenelse's.  For this two LaTeX boolean value VDMSL and
% VDMpp have been defined  (Lowercase p's are used to avoid conflict with
% the VDMPP environment variable.  The typical use are:
%   \ifthenelse{\boolean{VDMsl}}{vdmsl-text}{vdmpp-text}
%   \ifthenelse{\boolean{VDMsl}}{vdmsl-text}{}
%   \ifthenelse{\boolean{VDMpp}}{vdmpp-text}{}
% The advantage of this as opposed to ifdef's is that within a general
% paragraph specific VDM-SL and VDM++ parts can be distinguished without
% problematic empty lines.
% 
% The values are initialised such that exactly one of the values is true
% and the other is false.  This should hopefully avoid strange behaviour
% due to possible preprossing errors.  The default case is VDM-SL.
\newboolean{VDMsl}
\setboolean{VDMsl}{true}
\newboolean{VDMpp}
\setboolean{VDMpp}{false}
#ifdef VDMSL
\setboolean{VDMsl}{true}
\setboolean{VDMpp}{false}
#endif VDMSL
#ifdef VDMPP
\setboolean{VDMpp}{true}
\setboolean{VDMsl}{false}
#endif VDMPP

\newcommand{\AfterInit}[1]{}
%Initialisation must be performed before
%  {\tt #1} can be called.}

% This macro can be used in `description' lists where
% the item given to `meti' is put on its own line,
% thereby giving proper (nicer) identation to the
% explanation.
\newcommand{\meti}[1]{\item[#1]\mbox{}\\}

\newcommand{\Index}[1]{#1\index{#1}}

\newcommand{\Lit}[1]{`#1\Quote}
\newcommand{\Rule}[2]{
  \begin{quote}\begin{tabbing}
    #1\ \ \= = \ \ \= #2  ; %    Adds production rule to index
  \end{tabbing}\end{quote}
  }
\newcommand{\SeqPt}[1]{\{\ #1\ \}}
\newcommand{\lfeed}{\\ \> \>}
\newcommand{\OptPt}[1]{[\ #1\ ]}
\newcommand{\dsepl}{\ $|$\ }
\newcommand{\dsep}{\\ \> $|$ \>}
\newcommand{\Lop}[1]{\Lit{\kw{#1}}}
\newcommand{\Sig}[1]{\Lit{{\tt #1}}}
\newcommand{\blankline}{\vspace{\baselineskip}}
\newcommand{\Brack}[1]{(\ #1\ )}
\newcommand{\nmk}{\footnotemark}
\newcommand{\ntext}[1]{\footnotetext{{\bf Note: } #1}}


%\usepackage[dvips]{color}
%\usepackage{longtable}
%\usepackage{float}
%\definecolor{covered}{rgb}{0,0,0}     %black
%\definecolor{not_covered}{gray}{0.5}  %gray

%\restylefloat{figure}
\setcounter{topnumber}{3}
\def\topfraction{1.0}
\setcounter{bottomnumber}{3}
\def\bottomfraction{1.0}
\setcounter{totalnumber}{3}
\def\textfraction{.1}



\newlength{\keywwidth}

\newcommand{\xfigpicture}[4]{
\begin{figure}[hbt]
\setlength{\unitlength}{1mm}
\begin{center}
\mbox{
\begin{picture}(#1,#2)
\put(0,0){\special{psfile=#3 hscale=70 vscale=55}}
\end{picture} }
\end{center}
\caption{#4}
\end{figure}
}

\newcommand{\qq}{\marginpar{\bf ???}}
\newcommand{\aaa}{\tt }
\newcommand{\cmd}{\tt }
%\newcommand{\id}[1]{%
%  \settowidth{\keywwidth}{\tt #1}%
%  \protect\makebox[\keywwidth][l]{{\it #1}}}
%\nolinenumbering

\begin{document}
\vdmtoolsmanualcsk{\vdmslpp\ Installation Guide}
         {\ifthenelse{\boolean{VDMsl}}{v8.2}{v8.2}}
         {2016}
         {\vdmslpp}
         {1.0}

%%\pagenumbering{roman}
%%\tableofcontents\label{endtofc}
%%\ifthenelse{\isodd{\value{page}}}{\mbox{}\newpage}{}
%%\mbox{}
%%\newpage
%%\pagenumbering{arabic}

%%\parskip2ex
%%\parindent0mm

\section{Introduction} \label{introduction}

This document describes how to install \Toolbox.  \Toolbox\ can be
installed on the following platforms:
\begin{itemize}
%\item Sun Sparc running Solaris 2.6.
%\item HP9000/700 architecture running HP-UX 10.0.
\item PC/586, 686, running Windows 2000 or Windows XP. (Throughout this guide we use the term ``Windows'' to refer
collectively to Windows 98, Windows 2000 and Windows NT).
\item PC/586, 686, running Linux ReadHat /Mandrake /SuSE with Qt 3.2 or later
\item Macintosh Mac OS X 10.3.x 
\end{itemize}


%An installation of \Toolbox\ consists of these three components: 
An installation of \Toolbox\ consists of the component: 
\begin{itemize}
\item the \Toolbox\ application, 
%\item the license file, and
%\item a license server\index{license!server}. 
\end{itemize}

%The license file (typically named \licensedat) determines how you are
%allowed to use \Toolbox\ (how many users can use it, and when the
%license expires).  With an evaluation license file it is possible to
%run any number of simultaneous sessions whereas a purchased copy of
%\Toolbox\ will be installed with a license file that only allows a
%limited number of simultaneous sessions. To maintain this purchased
%copies must be installed with a license server\index{license!server}
%program which will monitor that the license conditions are not
%violated. The license server is not needed for an evaluation license
%file.

The installation description is divided into two parts: one for the
Windows platform, and one for the Unix platform.


\newpage
%\section{How to get a License File}

%\subsection{A Free Evaluation License}

%You can get a free evaluation license to \Toolbox, just contact CSK
%(see how on the coverpage).  CSK will then send you a free one-month
%evaluation license.

%\subsection{A Purchased License}

%A purchased license needs a server to run the license manager,
%therefore specific information about this server must be provided.
%This information is the IP address for Windows platforms and the
%hostid for Unix platforms. The information can be accessed in the
%following way:

%\begin{description}
%\item[IP address on the Windows Platform:] You can find the IP address
%  this way:
%  \begin{enumerate}
%  \item From the {\em Start\/} menu open {\em Settings\/} and choose
%    the Control Panel folder,
%  \item then open the {\em Network} folder.
%  \item then choose the {\em Protocols Tab},
 % \item and press the {\em Properties\/} button.
 % \item Here you will find the IP address.
 % \end{enumerate}
  
%\item[Hostid on the Unix Platform:] You can find the hostid by typing
%  {\tt hostid} on the machine that you want to run as the license
%  server.
%\end{description}

%Send either the IP address for your Windows machine or the hostid for
%your Unix machine to CSK (see how on the coverpage), and we will send
%you a purchased license file.



\newpage
\section{Installing VDMTools on a Windows Platform}
\label{install-win32}


% Remember \Index{install-nt}!!!!!!

This section includes steps to install \Toolbox\ on a Windows
  platform. \emph{Note: for the time being \Toolbox\ must be 
  installed on each machine where it should be used, it cannot be
  installed on a server and run from there.}

% The \licensedat\ file is used both by the \Toolbox\ appplication and
% by the license server, therefore it must be stored on the file system
% such that it is always accessible by these applications.

% The \Toolbox\ installation consists of the following two steps.
The \Toolbox\ installation consists of the following step.

\begin{enumerate}
\item Install \Toolbox\ on the destination machine, by running the
  installation program (named \texttt{setup\ifSLPP{sl}{pp}.exe}).  The
  installation program will prompt for the installation directory.
  % and
 % if you are not already using FLEXlm it will prompt for the full file
 % name of the \licensedat\ file. 
  
 % The license file need not exist at
 % the time of installation, but it must exist before \Toolbox\ can be
 % started. The value (the full file name of the \licensedat\ file)
 % will be stored in the environment variable LM\_LICENSE\_FILE. On the
%  Windows 98 platform this will be set in the \texttt{autoexec.bat}
%  file, and on the Windows 2000/NT platform this will be set in the
%  Control Panel. When uninstalling \Toolbox\ the LM\_LICENSE\_FILE
%  variable will not be unset, in this way it is possible to use same
%  license file for several products.


%\item For an evaluation license file: Copy the license file to the path
%  you specified above. 

%\item For a purchased license file: 
  
%  \begin{itemize}
%  \item In the subdirectory \texttt{lm} of the \Toolbox\ installation
%    directory you will find an executable \texttt{flexlminstall.exe}.
%    Copy this executable to the license server machine and run it. It
%    will ask for the destination directory (default \verb-c:\flexlm-)
%    and then install the files into this directory. Copy the license
%    file to the same directory.
  
%  \item From the {\em Start\/} menu open {\em Settings\/} and choose
%    the Control Panel folder and from here start {\em FLEXlm
%      License Manager}.
  
%    Choose {\em Setup\/} and fill out the paths of \texttt{lmgrd.exe},
%    \licensedat, and the log file, \texttt{Debug.log}, to be the
%    destination installation directory. \\
%    {\em Do not choose the \texttt{lmgrd95.exe} file instead of
%      \texttt{lmgrd.exe}, on any Windows platform.}
    
%  \item On Windows 98 you should select {\em Start server at Power
%      Up\/} and on Windows 2000 and NT you should choose {\em Use NT Service}.
  
%    Then choose the {\em Licenses\/} tab and press {\em Show License
%      File\/}; please check that the path to \texttt{ifad.exe} is the
%    same as the path you installed FLEXlm in.
  
%    Choose the {\em Control\/} tab and press the {\em Start\/} button.
%    It will say ``server started''; then verify that the server
%    started properly by pressing the {\em Status\/} button. It should
%    answer something like ``host: license server UP (MASTER)''.
%    Otherwise check the \texttt{Debug.log} file to see what went
%    wrong.
    
%  \item Finally copy the \licensedat\ file to the directory on the
%    client machines that were designated to contain it.
%  \end{itemize}
\end{enumerate}

#ifdef VDMPP

\subsection{Installing the VDM++ Add-In for Rational Rose}

If Rational Rose (Rose 98 or Rose 2000) is already installed on your
Windows system, the VDM++ add-in will automatically be added to the Rose
environment when installing the \Toolbox{}. The VDM++ add-in will
automatically be registered in the Windows registry for Rational
Rose.

If the Toolbox is installed before Rose, the user has to
use the following steps to install the VDM++ add-in:
\begin{itemize}
\item Install Rose.
\item Run the installation program named {\tt inst\_addin.exe}.
This can be found in the {\tt uml} subdirectory of the VDM++ Toolbox installation.
The VDM++ add-in will then
automatically be registered in the Windows registry for Rational Rose.
\end{itemize}

Restart Rose. Confirm that your add-in is activated via the Add-In
Manager menu of Rose.

#endif VDMPP

%\subsection{Trouble Shooting on Windows}

%This section enumerates the currently known problems and limitations
%specific to the installation of \Toolbox\ on Windows systems:

%\begin{itemize}
%\item UNC network paths (\verb-\\host\share\path\...-) are not
%  supported and will cause error messages from the GUI, like ``Error:
%  ``\verb-\\host\share-'' must be an absolute pathname''. 

%  A workaround for this problem is to map the UNC path to a drive
%  letter. 
%#ifdef VDMPP
%\item An error looking like the one in Figure~\ref{fig:corbaerror} is
%  caused by the absence of a TCP/IP connection and/or network adapter
%  on the client machine.
%  \begin{figure}[htbp]
%    \begin{center}
%      \resizebox{10cm}{!}{\includegraphics{corbaerror}}
%      \caption{Error caused by absence of TCP/IP}\label{fig:corbaerror}
%    \end{center}
%  \end{figure}

%  A workaround for this problem is as follows
%  \begin{enumerate}
%  \item Enable TCP/IP:
%    \begin{itemize}
%    \item Open Control Panel - Double click on Network;
%    \item Select Protocol tab;
%    \item Select TCP/IP Protocol (Windows installation CD might be
%          required).
 %   \end{itemize}

%  \item Set the environment variable \verb+OMNIORB_USEHOSTNAME+ to 127.0.0.1:
%    \begin{itemize}
%    \item Open Control Panel - Double click on System;
%    \item Select Environment tab;
%    \item Create variable \verb+OMNIORB_USEHOSTNAME+;
 %   \item Set value to 127.0.0.1
 %   \end{itemize}
 % \end{enumerate}

%#endif // VDMPP
  
%\item An error looking like the one in Figure~\ref{fig:licpatherror}
%  can be caused by two things:
%  \begin{figure}[htbp]
%    \begin{center}
 %     \resizebox{10cm}{!}{\includegraphics{licenseerror}}
 %     \caption{Error caused by wrong path of the LM\_LICENSE\_FILE variable 
 %       or that the variable is not accessible yet.\label{fig:licpatherror}}
 %   \end{center}
 % \end{figure}
 % \begin{enumerate}
 % \item The filename of the LM\_LICENSE\_FILE variable in the
 %   Control Panel (on Windows 2000/NT) or \texttt{autoexec.bat} (on
 %   Windows 98) is either wrong or its path is wrong.
    
%    In this case make sure that the filename in LM\_LICENSE\_FILE
%    variable is \licensedat\ containing the entire path to where it is
%    stored on the file system.

%  \item The LM\_LICENSE\_FILE is not accessible yet.
    
 %   In this case either make sure that the LM\_LICENSE\_FILE variable
%    is accessible or reboot the machine. On Windows NT the
%    LM\_LICENSE\_ FILE %%Note the extra space to allow meaningful line break.
%    variable can be made accessible from within {\em
%      Start$\rightarrow$Settings$\rightarrow$
%    Control %%Note the extra space to allow meaningful line break.
%    Panel$\rightarrow$System\/} by pressing the {\em Apply\/} button.
%  \end{enumerate}
  
%\item An error looking like the one in Figure~\ref{fig:licdirerror}
%  can be caused by the name of the LM\_LICENSE\_FILE variable in the
%  Control Panel (on Windows 2000/NT) or \texttt{autoexec.bat} (on
%  Windows 98) is the name of the directory, where the \licensedat\ file
%  is stored.

 % \begin{figure}[htbp]
%    \begin{center}
%      \resizebox{10cm}{!}{\includegraphics{licensedirerror}}
%      \caption{Error caused by path alone given to the LM\_LICENSE\_FILE 
%        variable.\label{fig:licdirerror}}
%    \end{center}
%  \end{figure}
  
%  In this case add the \licensedat\ filename to the pathname in the
%  LM\_LICE NSE\_FILE variable. %%Note the extra space to allow meaningful line break.

%\end{itemize}


\newpage
\section{Installing VDMTools on a Unix Platform}

%The description of how to \Index{install} \Toolbox\ is separated into
%two sections, one for installing \Index{evaluation copies}
%(Section~\ref{evalinstall}) and one for installing \Index{purchased
%  copies} (Section~\ref{purchinstall}).  Furthermore,
%Section~\ref{sec:emacsinstall} describes how to install \Toolbox\ 
%within Emacs, and Section~\ref{sec:unixtroubleshoot}.


%\subsection{Installing an Evaluation Copy of VDMTools}\label{evalinstall}
%
%\begin{enumerate}
%
%\item Create a directory {\tt \vdmhome} to hold \Toolbox.
%  
%\item Read in the discs or the files downloaded from CSK's WWW or
%  ftp server in the {\tt \vdmhome} directory with the {\tt tar}
%  command.  This will create a number of files which must be installed
%  explicitly as described in the following steps.
%  
%\item Install the executables {\tt bin/\vdmde} and {\tt bin/\vdmgde}
%  either by moving them to your standard binary directory or by
%  including the directory {\tt \vdmhome/bin} in your search-path.
%  
%\item In order to use the graphical user interface, {\tt \vdmgde}, you
%  must set the environment variables:
%
%#ifdef VDMSL
%  \begin{quote}
%    {\tt \Index{VDMSLROOT}} to point to {\tt \vdmhome}
%  \end{quote} 
%#endif VDMSL 
%#ifdef VDMPP 
%  \begin{quote} 
%    {\tt \Index{VDMPPROOT}} to point to {\tt \vdmhome}
%  \end{quote}
%#endif VDMPP
%
%  This can be done in the following way:
%
%  Using csh shell syntax:
%#ifdef VDMSL
%\begin{verbatim}
%  setenv VDMSLROOT=/path/to/vdmhome
%\end{verbatim}
%#endif VDMSL 
%#ifdef VDMPP 
%\begin{verbatim}
%  setenv VDMPPROOT=/path/to/vpphome
%\end{verbatim}
%#endif VDMPP
  
%  Using bash shell syntax:
%#ifdef VDMSL
%\begin{verbatim}
%  VDMSLROOT=/path/to/vdmhome
%  export VDMSLROOT
%\end{verbatim}
%#endif VDMSL 
%#ifdef VDMPP 
%\begin{verbatim}
%  VDMPPROOT=/path/to/vpphome
%  export VDMPPROOT
%\end{verbatim}
%#endif VDMPP
  
%\item Installing the license file.\index{license!file}
  
%  {\bf If you are not already running the FlexLm license server
%    program} {\tt lmgrd} on the license manager host, you can install
%  the \licensedat\ file by setting the environment variable {\tt
%    \Index{LM\_LICENSE\_FILE}} to point to the license file
%  \Index{\licensedat}. All users of \Toolbox\ must set this
%  environment variable.
  
%  {\bf If you are already running the \Index{FlexLm} license server
%    program} {\tt lmgrd} on the license manager host, and the {\tt
%    SERVER} line in the \licensedat\ files of all your applications
%  using FlexLm are identical, you can merge these files.  You will
%  need to use only one license server program and use only one
%  communications port. In this case you should copy the {\tt FEATURE}
%  lines from the \licensedat\ supplied with \Toolbox\ into your
%  current master \licensedat.
  
%  {\bf Notice} that you should not execute the {\tt lmgrd} program
%  when installing an evaluation license.
%\end{enumerate}

%You should now be able to call the tools from the command line.


%\subsection{Installing a Purchased Copy of VDMTools}\label{purchinstall}
\subsection{Installing a Copy of VDMTools}\label{purchinstall}

\begin{enumerate}
\item Create a directory {\tt \vdmhome} to hold \Toolbox.
  
\item Read in the discs or the files down-loaded from CSK's WWW or
  ftp server in the {\tt \vdmhome} directory with the {\tt tar}
  command.  This will create a number of files which must be installed
  explicitly as described in the following steps.
  
\item Install the executables {\tt bin/\Index{\vdmde}} and {\tt
    bin/\Index{\vdmgde}} either by moving them to your standard binary
  directory or by including the directory {\tt \vdmhome/bin} in your
  path.
\end{enumerate}

%\item In order to use the graphical user interface, {\tt \vdmgde}, you
%  must set the environment variable:

%#ifdef VDMSL
%  \begin{quote}
%    {\tt \Index{VDMSLROOT}} to point to {\tt \vdmhome}
%  \end{quote} 
%#endif VDMSL 
%#ifdef VDMPP 
%  \begin{quote} 
%    {\tt \Index{VDMPPROOT}} to point to {\tt \vdmhome}
%  \end{quote}
%#endif VDMPP

%  This can be done in the following way:

%  Using csh shell syntax:
%#ifdef VDMSL
%\begin{verbatim}
%  setenv VDMSLROOT=/path/to/vdmhome
%\end{verbatim}
%#endif VDMSL 
%#ifdef VDMPP 
%\begin{verbatim}
%  setenv VDMPPROOT=/path/to/vpphome
%\end{verbatim}
%#endif VDMPP
  
%  Using bash shell syntax:
%#ifdef VDMSL
%\begin{verbatim}
%  VDMSLROOT=/path/to/vdmhome
%  export VDMSLROOT
%\end{verbatim}
%#endif VDMSL 
%#ifdef VDMPP 
%\begin{verbatim}
%  VDMPPROOT=/path/to/vpphome
%  export VDMPPROOT
%\end{verbatim}
%#endif VDMPP
 
%\item Installing the license file.
  
%  {\bf If you are not already running the FlexLm license server
%    program} {\tt lmgrd} on the license manager host, install the
%  \licensedat\ file by setting the environment variable {\tt
%    \Index{LM\_LICENSE\_FILE}} to point to the license
%  file\index{license!file} \Index{\licensedat}.\\
%  {\em All users of \Toolbox\ must set this environment variable}.
  
%  {\bf If you are already running the FlexLm license server program}
%  {\tt lmgrd} on the license manager host, and the {\tt SERVER} line
%  in the \licensedat\ files of all your applications using FlexLm are
%  identical, you can merge these files. You will need to use only one
%  license server program and use only one communications port. The
%  steps are:

%  \begin{enumerate}
%  \item Copy the {\tt DAEMON} and {\tt FEATURE} lines, but not the
%    {\tt SERVER} line, from the \licensedat\ supplied with \Toolbox\ 
%    into your current master \licensedat.
    
%  \item Let the license manager re-read the master \licensedat\ file.
%    In order to do this you must login on the server machine and run
%    the program {\tt lm/lmreread}:
    
%#ifdef VDMSL
%    \begin{alltt}
%  hermes:> rlogin server
%  hermes:> vdmhome/lm/lmreread
%  hermes:>
%    \end{alltt}
%#endif VDMSL
%#ifdef VDMPP
%    \begin{alltt}
%  hermes:> rlogin server
%  hermes:> vpphome/lm/lmreread
%  hermes:>
%    \end{alltt}
%#endif VDMPP

%  \end{enumerate}

%\item Installing the vendor daemon.
  
%  You must move the vendor daemon executable {\tt lm/IFAD} to the {\tt
%    /etc} directory. If you prefer to install the vendor daemon in
%  another directory you must change the path in the DAEMON line of
%  \licensedat\ accordingly.

%\item Starting the license server.
  
%  You should now start the license server program on the host machine.
%  This is done by typing:

%#ifdef VDMSL
%  \begin{alltt}
%  hermes:> rlogin server
%  hermes:> vdmhome/lm/lmgrd >& vdmhome/lm/ifad.log &
%  hermes:>
%  \end{alltt}
%#endif VDMSL
%#ifdef VDMPP
%  \begin{alltt}
%  hermes:> rlogin server
%  hermes:> vpphome/lm/lmgrd >& vpphome/lm/ifad.log &
%  hermes:>
%  \end{alltt}
%#endif VDMPP

%  Login as root and add the following two lines to the license
%  manager host's {\tt etc/rc.local} file. This enables the license
%  server program to be restarted whenever the host is booted up.
%  Remember to specify the full pathname for {\tt \vdmhome/lm}.

%  \begin{alltt}
%  vhome=vdmhome/lm
%  $vhome/lmgrd > $vhome/ifad.log 2>&1 &
%  \end{alltt}
  
%  Now run the {\tt lm/lmstat} program\index{lmstat program} to verify
%  the installation. It should look something like this:

%  \begin{alltt}
%  hermes:> lm/lmstat -a
%  License server status (License file: /local/share/licenses/
%      license.dat): hermes: license server UP (MASTER)
%  Vendor daemon status (on hermes):
%        IFAD (v3.x): UP
%  Feature usage info:
%  Users of vdmsl: 
%  Users of vdmcg: 
%  Users of vdmpp: 
%  Users of vppcg: 
%  Users of vdmdl:  (Total of 4 licenses available)
%  hermes:>
%  \end{alltt}
  
%  It should respond that both the license server and the vendor daemon
%  are up, as in the example.

%\end{enumerate}

You should now be able to call the tools from the command line.


\subsection{Installing VDMTools for Emacs}
\label{sec:emacsinstall}

If you wish to use \Toolbox\ within Emacs you must perform the
following steps that show how to set up Emacs to support the tools in
Emacs.

\begin{enumerate}
\item Installing the Emacs interface.
  
  Install the Emacs interface {\tt emacs/\Index{\vdmdeNineteenEl}}
  either by placing it with your existing Emacs utilities according to
  your own site procedures, or by moving it to one of your local
  directories.

\item Updating your {\tt .emacs} file\index{.emacs file}
  
  Set up your {\tt .emacs} file to read in the {\tt \vdmdeNineteenEl}
  interface. If you have located {\tt emacs/\vdmdeNineteenEl} with
  your standard Emacs utilities you must add the following line to
  your {\tt .emacs} file

#ifdef VDMSL
  \begin{verbatim}
  (autoload 'vdmde "vdmde" "" t)
  \end{verbatim}
#endif VDMSL
#ifdef VDMPP
  \begin{verbatim}
  (autoload 'vppde "vppde" "" t)
  \end{verbatim}
#endif VDMPP

  otherwise you must add the full path to {\tt emacs/\vdmdeNineteenEl}
  like

#ifdef VDMSL
  \begin{verbatim}
  (autoload 'vdmde "/path/vdmde" "" t)
  \end{verbatim}
#endif VDMSL
#ifdef VDMPP
  \begin{verbatim}
  (autoload 'vppde "/path/vppde" "" t)
  \end{verbatim}
#endif VDMPP

%\item Copy the Toolbox manual pages into your {\tt man} directory.
\end{enumerate}

You should now be able to use the \Toolbox.



%\subsection{Trouble Shooting on UNIX}
%\label{sec:unixtroubleshoot}

%This sections enumerates a number of known mistakes performed during
%installation together with solutions and workarounds.

%\begin{itemize}
%\item When starting the license server an error message is reported
%  (in the \path+ifad.log+ if output is piped into that file) looking
%  like:
%\begin{verbatim}
%  license manager: can't initialize: No SERVER lines in
%  license file (-13,66)
%\end{verbatim}
%  This occurs when you try to start the license server with an
%  evaluation license, which is not needed. \Toolbox\ will still run
%  despite this error.

%\end{itemize}


\newpage
\section{How to get New Releases}
When a new version of \Toolbox\ is released it will be made available
%at CSK's \Index{WWW site}:
under VDMTools information \Index{Web site}:

\begin{verbatim}
#ifdef ENG
    http://fmvdm.org/
#endif ENG
#ifdef JPN
    http://fmvdm.org/index-ja.html
#endif JPN
\end{verbatim}

%and \Index{ftp site}:

%#ifdef VDMSL
%\begin{verbatim}
%    Site:      ftp.ifad.dk 
%    Directory: /pub/toolbox
%\end{verbatim}
%#endif VDMSL
%#ifdef VDMPP
%\begin{verbatim}
%    Site:      ftp.ifad.dk 
%    Directory: /pub/vdm++_toolbox
%\end{verbatim}
%#endif VDMPP

%Everyone who has a \Index{license} will be informed about new releases.


\end{document}
