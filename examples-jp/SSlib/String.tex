% LaTeX 2e Document.
% 
% $Id: String.tex,v 1.2 2006/01/10 10:45:26 vdmtools Exp $
% 

%%%%%%%%%%%%%%%%%%%%%%%%%%%%%%%%%%%%%%%%
% PDF compatibility code. 
\makeatletter
\newif\ifpdflatex@
\ifx\pdftexversion\@undefined
\pdflatex@false
%\message{Not using pdf}
\else
\pdflatex@true
%\message{Using pdf}
\fi

\newcommand{\latexorpdf}[2]{
  \ifpdflatex@ #2
  \else #1
  \fi
}

\newcommand{\pformat}{a4paper}

\makeatother
%%%%%%%%%%%%%%%%%%%%%%%%%%%%%%%%%%%%%%%%

\latexorpdf{
\documentclass[\pformat,12pt]{jsarticle}
}{
\documentclass[\pformat,pdftex,12pt]{jsarticle}
}


\usepackage[dvips]{color}
\usepackage{longtable}
\usepackage{alltt}
\usepackage{graphics}
\usepackage{vpp}
\usepackage{makeidx}
\makeindex

\definecolor{covered}{rgb}{0,0,0}      %black
%\definecolor{not-covered}{gray}{0.5}   %gray for previewing
%\definecolor{not-covered}{gray}{0.6}   %gray for printing
\definecolor{not-covered}{rgb}{1,0,0}  %red

\newcommand{\InstVarDef}[1]{{\bf #1}}
\newcommand{\TypeDef}[1]{{\bf #1}}
\newcommand{\TypeOcc}[1]{{\it #1}}
\newcommand{\FuncDef}[1]{{\bf #1}}
\newcommand{\FuncOcc}[1]{#1}
\newcommand{\MethodDef}[1]{{\bf #1}}
\newcommand{\MethodOcc}[1]{#1}
\newcommand{\ClassDef}[1]{{\sf #1}}
\newcommand{\ClassOcc}[1]{#1}
\newcommand{\ModDef}[1]{{\sf #1}}
\newcommand{\ModOcc}[1]{#1}


\title{VDM++ Sorting Algorithms}
\author{IFAD}
\date{August, 1997}

\begin{document}
\maketitle

\section{Introduction}

This document contains a sorting example. The class diagram can be
seen in Figure \ref{inh}.  The structure of the example is known as
the \textit{strategy} pattern. This pattern defines a family of
algorithms, encapsulates each one and make them interchangeable. The
\textit{strategy} pattern lets the algorithm vary independently from
clients that use it. The \texttt{SortMachine} class is the client that uses the
different sorting algorithms. The \texttt{Sorter} class is an abstract class
that defines a common interface to all supported algorithms.

% LaTeX 2e Document.
% 
% $Id: String.tex,v 1.2 2006/01/10 10:46:35 vdmtools Exp $
% 

%%%%%%%%%%%%%%%%%%%%%%%%%%%%%%%%%%%%%%%%
% PDF compatibility code. 
\makeatletter
\newif\ifpdflatex@
\ifx\pdftexversion\@undefined
\pdflatex@false
%\message{Not using pdf}
\else
\pdflatex@true
%\message{Using pdf}
\fi

\newcommand{\latexorpdf}[2]{
  \ifpdflatex@ #2
  \else #1
  \fi
}

\newcommand{\pformat}{a4paper}

\makeatother
%%%%%%%%%%%%%%%%%%%%%%%%%%%%%%%%%%%%%%%%

%\latexorpdf{
%\documentclass[\pformat,12pt]{article}
%}{
%\documentclass[\pformat,pdftex,12pt]{article}
%}
\documentclass[]{jarticle}

\usepackage[dvips]{color}
\usepackage{array}
\usepackage{longtable}
\usepackage{alltt}
\usepackage{graphics}
\usepackage{vpp}
\usepackage{makeidx}
\makeindex

\definecolor{covered}{rgb}{0,0,0}      %black
%\definecolor{not-covered}{gray}{0.5}   %gray for previewing
%\definecolor{not-covered}{gray}{0.6}   %gray for printing
\definecolor{not-covered}{rgb}{1,0,0}  %red

\newcommand{\InstVarDef}[1]{{\bf #1}}
\newcommand{\TypeDef}[1]{{\bf #1}}
\newcommand{\TypeOcc}[1]{{\it #1}}
\newcommand{\FuncDef}[1]{{\bf #1}}
\newcommand{\FuncOcc}[1]{#1}
\newcommand{\MethodDef}[1]{{\bf #1}}
\newcommand{\MethodOcc}[1]{#1}
\newcommand{\ClassDef}[1]{{\sf #1}}
\newcommand{\ClassOcc}[1]{#1}
\newcommand{\ModDef}[1]{{\sf #1}}
\newcommand{\ModOcc}[1]{#1}

%\nolinenumbering
%\setindent{outer}{\parindent}
%\setindent{inner}{0.0em}

\title{FString���C�u�����[}
\author{
�����L
���{�t�B�b�c�������\\
���Z�p������\\
TEL : 03-3623-4683\\
shin.sahara@jfits.co.jp\\
}
%\date{2004�N2��16��}

\begin{document}
\setlength{\baselineskip}{12pt plus .1pt}
\tolerance 10000
\maketitle

\begin{abstract}
\setlength{\baselineskip}{12pt plus .1pt}
������iseq of char�j�Ɋւ��֐���񋟂��郂�W���[���ł���B
\end{abstract}
%\vspace{-1cm}

% LaTeX 2e Document.
% 
% $Id: String.tex,v 1.2 2006/01/10 10:46:35 vdmtools Exp $
% 

%%%%%%%%%%%%%%%%%%%%%%%%%%%%%%%%%%%%%%%%
% PDF compatibility code. 
\makeatletter
\newif\ifpdflatex@
\ifx\pdftexversion\@undefined
\pdflatex@false
%\message{Not using pdf}
\else
\pdflatex@true
%\message{Using pdf}
\fi

\newcommand{\latexorpdf}[2]{
  \ifpdflatex@ #2
  \else #1
  \fi
}

\newcommand{\pformat}{a4paper}

\makeatother
%%%%%%%%%%%%%%%%%%%%%%%%%%%%%%%%%%%%%%%%

%\latexorpdf{
%\documentclass[\pformat,12pt]{article}
%}{
%\documentclass[\pformat,pdftex,12pt]{article}
%}
\documentclass[]{jarticle}

\usepackage[dvips]{color}
\usepackage{array}
\usepackage{longtable}
\usepackage{alltt}
\usepackage{graphics}
\usepackage{vpp}
\usepackage{makeidx}
\makeindex

\definecolor{covered}{rgb}{0,0,0}      %black
%\definecolor{not-covered}{gray}{0.5}   %gray for previewing
%\definecolor{not-covered}{gray}{0.6}   %gray for printing
\definecolor{not-covered}{rgb}{1,0,0}  %red

\newcommand{\InstVarDef}[1]{{\bf #1}}
\newcommand{\TypeDef}[1]{{\bf #1}}
\newcommand{\TypeOcc}[1]{{\it #1}}
\newcommand{\FuncDef}[1]{{\bf #1}}
\newcommand{\FuncOcc}[1]{#1}
\newcommand{\MethodDef}[1]{{\bf #1}}
\newcommand{\MethodOcc}[1]{#1}
\newcommand{\ClassDef}[1]{{\sf #1}}
\newcommand{\ClassOcc}[1]{#1}
\newcommand{\ModDef}[1]{{\sf #1}}
\newcommand{\ModOcc}[1]{#1}

%\nolinenumbering
%\setindent{outer}{\parindent}
%\setindent{inner}{0.0em}

\title{FString���C�u�����[}
\author{
�����L
���{�t�B�b�c�������\\
���Z�p������\\
TEL : 03-3623-4683\\
shin.sahara@jfits.co.jp\\
}
%\date{2004�N2��16��}

\begin{document}
\setlength{\baselineskip}{12pt plus .1pt}
\tolerance 10000
\maketitle

\begin{abstract}
\setlength{\baselineskip}{12pt plus .1pt}
������iseq of char�j�Ɋւ��֐���񋟂��郂�W���[���ł���B
\end{abstract}
%\vspace{-1cm}

% LaTeX 2e Document.
% 
% $Id: String.tex,v 1.2 2006/01/10 10:46:35 vdmtools Exp $
% 

%%%%%%%%%%%%%%%%%%%%%%%%%%%%%%%%%%%%%%%%
% PDF compatibility code. 
\makeatletter
\newif\ifpdflatex@
\ifx\pdftexversion\@undefined
\pdflatex@false
%\message{Not using pdf}
\else
\pdflatex@true
%\message{Using pdf}
\fi

\newcommand{\latexorpdf}[2]{
  \ifpdflatex@ #2
  \else #1
  \fi
}

\newcommand{\pformat}{a4paper}

\makeatother
%%%%%%%%%%%%%%%%%%%%%%%%%%%%%%%%%%%%%%%%

%\latexorpdf{
%\documentclass[\pformat,12pt]{article}
%}{
%\documentclass[\pformat,pdftex,12pt]{article}
%}
\documentclass[]{jarticle}

\usepackage[dvips]{color}
\usepackage{array}
\usepackage{longtable}
\usepackage{alltt}
\usepackage{graphics}
\usepackage{vpp}
\usepackage{makeidx}
\makeindex

\definecolor{covered}{rgb}{0,0,0}      %black
%\definecolor{not-covered}{gray}{0.5}   %gray for previewing
%\definecolor{not-covered}{gray}{0.6}   %gray for printing
\definecolor{not-covered}{rgb}{1,0,0}  %red

\newcommand{\InstVarDef}[1]{{\bf #1}}
\newcommand{\TypeDef}[1]{{\bf #1}}
\newcommand{\TypeOcc}[1]{{\it #1}}
\newcommand{\FuncDef}[1]{{\bf #1}}
\newcommand{\FuncOcc}[1]{#1}
\newcommand{\MethodDef}[1]{{\bf #1}}
\newcommand{\MethodOcc}[1]{#1}
\newcommand{\ClassDef}[1]{{\sf #1}}
\newcommand{\ClassOcc}[1]{#1}
\newcommand{\ModDef}[1]{{\sf #1}}
\newcommand{\ModOcc}[1]{#1}

%\nolinenumbering
%\setindent{outer}{\parindent}
%\setindent{inner}{0.0em}

\title{FString���C�u�����[}
\author{
�����L
���{�t�B�b�c�������\\
���Z�p������\\
TEL : 03-3623-4683\\
shin.sahara@jfits.co.jp\\
}
%\date{2004�N2��16��}

\begin{document}
\setlength{\baselineskip}{12pt plus .1pt}
\tolerance 10000
\maketitle

\begin{abstract}
\setlength{\baselineskip}{12pt plus .1pt}
������iseq of char�j�Ɋւ��֐���񋟂��郂�W���[���ł���B
\end{abstract}
%\vspace{-1cm}

\include{test/String.vpp}

\include{test/StringT.vpp}

%\newpage
%\addcontentsline{toc}{section}{Index}
%\printindex

\end{document}


\include{test/StringT.vpp}

%\newpage
%\addcontentsline{toc}{section}{Index}
%\printindex

\end{document}


\include{test/StringT.vpp}

%\newpage
%\addcontentsline{toc}{section}{Index}
%\printindex

\end{document}


\newpage
\addcontentsline{toc}{section}{Index}
\printindex


\end{document}
